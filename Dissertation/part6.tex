\chapter{Асимптотика собственных значений для ветвящихся многообразий из трех плоских листов, ограниченных софокусными квадриками.}\label{ch:ch6}

\section{Введение.}\label{sec:ch6/sect1}
Рассмотрим  $\Gamma = \cup_{j=1}^3 \Gamma_j$, где каждый $\Gamma_j$ с четырех сторон ограничен горизонтальной осью $\{y=0\}$, вертикальной осью $\{x=0\}$, эллипсом $\frac{x^2}{a^2} + \frac{y^2}{b^2} =1$ и софокусным эллипсом $\frac{x^2}{a^2 - \lambda_j} + \frac{y^2}{b^2-\lambda_j} =1$, где $b^2 < \lambda_j < a^2, \ j=1,2,3$. Пусть все $\Gamma_j$ имеют общий отрезок $\{x=0\}$.
\textcolor{red}{а могли бы соединяться и по y=0 тоже}

Без ограничения общности будем считать, что фокусы эллипса находятся в точках $(\pm \delta, 0)$. Введем эллиптическую систему координат $(x,y) = (\delta \cosh \rho \cos \phi, \Delta \sinh \rho \sin \phi)$. Каждая область задается как  
$$\Gamma_j = \{(\rho, \phi) \in [0, \rho_0] \times [\phi_j, \frac{\pi}{2}]\},$$
где $\rho_0  = \text{arccosh} \frac{a}{\delta}$ -- граница эллипса, $\phi_0$ определяются для гипербол как в разделе \ref{sec:ch2/sec4}.


\section{Собственные функции.}\label{sec:ch6/sect2}
Рассмотрим для начала случай, когда каждая область $\Gamma_j$ целиком находится в замыкании первого квадранта, т.е. $0 \leq \phi_j$. 
Здесь немного повествование будет поспешное: из предыдущих глав мы знаем как устроены общие решения уравнений Гельмгольца в ограниченных квадриками областях. В каждой $\Gamma_j$ решение $\Delta u_j +\varkappa^2 u_j = 0$ записываем в виде произведения $u_j(x,y) = \Phi_j(\phi)R_j(\rho)$, причем из примера \ref{sec:ch5/sect3/subsect1} заключаем, что $R_j(\rho) = R(\rho), \ j=1,2,3$.  
$R(\rho_0)=R(0)=0$, следовательно $R(\rho) = Se_\nu(\rho, q)$ для некоторого $\nu$, который определяется из решений углового уравнения $\Phi_j(\phi)$. 

Потребуем от угловых функций Матьё выполнения \textit{условий Кирхгофа} в точке $\frac{\pi}{2}$:
\begin{equation}
\begin{pmatrix}
0	&	0	&	0	\\
0	&	0	&	0	\\
1	&	1	&	1
\end{pmatrix} 
\left.
\begin{pmatrix}
\alpha_1 \Phi_1'(\phi)		\\
\alpha_2 \Phi_2'(\phi)		\\
\alpha_3 \Phi_3'(\phi)	
\end{pmatrix}
\right|_{\phi = \frac{\pi}{2}} + 
\begin{pmatrix}
1	&	-1	&	0	\\
0	&	1	&	-1	\\
0	&	0	&	0
\end{pmatrix} 
\left.
\begin{pmatrix}
\alpha_1 \Phi_1(\phi)		\\
\alpha_2 \Phi_2(\phi)		\\
\alpha_3 \Phi_3(\phi)	
\end{pmatrix} 
\right|_{\phi = \frac{\pi}{2}} = 
\begin{pmatrix}
0	\\	0	\\	0
\end{pmatrix} 
\label{eq:ellipsesKirchhoffCondition}
\end{equation}

Нетрудно заметить, что из равенств $\alpha_1 \Phi_1( \frac{\pi}{2})=\alpha_2 \Phi_2( \frac{\pi}{2}) = \alpha_3 \Phi_3( \frac{\pi}{2})	$ следует, что равенство для производных после его домножения на $\Phi_2(\frac{\pi}{2}) \Phi_3(\frac{\pi}{2})$ может быть записано в виде
\begin{multline}
\left. \alpha_1 \Phi_1'( \phi ) \Phi_2(\phi) \Phi_3(\phi) + \alpha_2 \Phi_2(\phi) \Phi_2'(\phi) \Phi_3(\phi) + \alpha_3 \Phi_3(\phi) \Phi_2(\phi) \Phi_3'(\phi) \right|_{\phi = \frac{\pi}{2}} = \\
\left. \alpha_1 \Phi_1'( \phi ) \Phi_2(\phi) \Phi_3(\phi) + \alpha_1 \Phi_1(\phi) \Phi_2'(\phi) \Phi_3(\phi) + \alpha_1 \Phi_1(\phi) \Phi_2(\phi) \Phi_3'(\phi) \right|_{\phi = \frac{\pi}{2}} = \\
\alpha_1 \frac{\partial}{\partial \phi} \left. \bigg(
\Phi_1( \phi )\Phi_2(\phi) \Phi_3(\phi) \bigg)  \right|_{\phi = \frac{\pi}{2}} = 0,
\end{multline}
для удобства положим $\alpha_1 = 1$. 

\section{Спектр оператора Лапласа и его асимптотика при малых эксцентриситетах.}\label{sec:ch6/sect3}
Из граничного условия $\Phi_j(\phi_j) = 0$ можем заключить, что с точностью до умножения на константу $\Phi_j(\phi)  = se_\nu(\phi) ce_\nu(\phi_j) - ce_\nu(\phi) se_\nu(\phi_j)$ для некоторого $\nu$. Отсюда с учетом разложений угловых функций Матьё в ряды Фурье можно определить $f(\nu)$ и искать нужные $\nu = \nu_0+q \nu_1 + o(q)$, пользуясь на соображения, похожие на \ref{sec:ch2/sec4/subs2}. После этого поиск собственных значений оператора Лапласа аналогичен рассмотрению эллиптического сектора  $B_\varepsilon$ из \ref{sec:ch2/sec4/subs2/subs2}.

Однако, даже для $\nu_0$ получается задача, сравнимая по сложности с примером \ref{sec:ch4/sect3/subsect1} для одномерного графа с одной точкой ветвления, для которой  затруднительно получить ответ в явном виде. Рассмотрим случай одинаковых листов: $\phi_1=\phi_2=\phi_3=\phi_0$. В этом случае  нужные нам значения параметра $\nu$ можно найти как нули функции 
$$f(\nu) = \frac{\partial}{\partial \phi} \left. \bigg(
\Phi^3( \phi ) \bigg)  \right|_{\phi = \frac{\pi}{2}} = \left. 3 \Phi^2(\phi) \Phi'(\phi)  \right|_{\phi = \frac{\pi}{2}},$$
где $\Phi(\phi)  = se_\nu(\phi) ce_\nu(\phi_0) - ce_\nu(\phi) se_\nu(\phi_0)$. 
Нули первого множителя $\Phi(\frac{\pi}{2})$ рассматривались в \ref{sec:ch2/sec4/subs2/subs2}. Рассмотрим подробнее нули функции $\Phi'(\frac{\pi}{2})$ как функции от $\nu$. 

Запишем $\Phi'(\frac{\pi}{2})$, используя приведенные в \ref{sec:ch2/sec4/subs2} разложения для \eqref{eq:se} и \eqref{eq:сe} по степеням $q$. Для краткости сразу подставим $\nu = \nu_0 + q \nu_1 + o(q)$. Тогда коэффициентом ряда $\Phi'(\frac{\pi}{2})$ при нулевой степени $q$ является
$$ \nu_0 \left( \cos \frac{\nu_0 \pi}{2} \cos \nu_0 \phi_1 + \sin \frac{\pi \nu_0}{2} \sin \nu \phi_1 \right) = \nu_0 \cos \nu_0 ( \frac{\pi}{2} - \phi_1),$$
то есть $\nu_0 = \frac{\pi(1 + 2m)}{\pi-2 \phi_1}$ для $m \in \mathbb{N}$. 

Коэффициент при первой степени $q$ тогда может быть упрощен до вида
$$
\frac{\nu_0^2 \sin 2 \phi_1}{2(1-\nu_0^2)} - \nu_1 \nu_0 (\frac{\pi}{2} - \phi_1),
$$
причем коэффициент равен нулю при $\nu_1 =  \dfrac{\frac{\nu_0^2 \sin 2 \phi_1}{2(1-\nu_0^2)}}{\nu_0 (\frac{\pi}{2} - \phi_1)} = \dfrac{\nu_0 \sin 2\phi_1}{(1-\nu_0^2)(\pi-2\phi_1)}.$
Таким образом, для этого $\nu = \nu(q) = \nu_0 + q \nu_1 + o(q)$ остается найти нули радиальной функции Матьё $Se_{\nu(q)}(\rho_0, q)$ как функции от $q$. Необходимая работа уже выполнена в  \ref{eq:oddSe_nueigenvalues}, тогда подставляя полученные $\nu_0$ и $\nu_1$ в выражения \ref{eq:u0}, \ref{eq:u1}, получим 

\begin{align}
  & \nu_0 = \frac{\pi(1 + 2m)}{\pi-2 \phi_1},  \  \nu_1 = \dfrac{\nu_0 \sin 2\phi_1}{(1-\nu_0^2)(\pi-2\phi_1)},\\
   & \varkappa_{k, m}^2 = \frac{\alpha_{\nu_0, m}^2}{r_0^2} 
+  \delta^2 \frac{\alpha_{\nu_0, m}^3}{2 r_0^4}\frac{1}{ \left.\frac{\partial J_{\nu_0}(u)}{\partial u}\right|_{u=\alpha_{\nu_0, m}} } \biggl(
\frac{(\nu_0 - 2)J_{\nu_0-2}(\alpha_{\nu_0, m})   }{4\nu_0 (\nu_0-1)} - \notag \\
&\qquad\qquad{}- \frac{(\nu_0 + 2)J_{\nu_0+2}(\alpha_{\nu_0, m})}{4\nu_0 (\nu_0+1)} - \nu_1 \left.\frac{\partial J_\nu}{\partial \nu}\right|_{\nu = \nu_0}(\alpha_{\nu_0, m})
    \biggr) + o(\delta^2),
 \end{align}
где $\alpha_{\nu_0, n}$ -- $n$-ый нуль функции Бесселя $J_{\nu_0}(x)$, $m, n \in \mathbb{N}$.

Нули сомножителя $\Phi(\frac{\pi}{2})$, как следует из  \ref{sec:ch2/sec4/subs2/subs2}, добавят еще одну серию собственных значений:
$$ \nu_0 = \frac{\pi 2m}{\pi-2 \phi_1},  \  \nu_1 = \dfrac{2 \nu_0  \sin 2\phi_1}{(1-\nu_0^2)(\pi-2\phi_1)}, \ m \in \mathbb{N}$$

\section{Дополнительные случаи.}\label{sec:ch6/sect4}
Ранее мы предположили, что $0 \leq \phi_j$. Если же допустить, что хотя бы для одного $j$ выполняется $\phi_j < 0$, тогда внутренность соответствующего $\Gamma_j$ содержит часть соединяющего фокусы отрезка, которую назовем $J$. На отрезке $J$ потребуется наложить условия непрерывности функции и производной, следствием чего будет условие на одинаковую четность угловой и радиальной функций $R_j(\rho), \Phi_j(\phi)$. Если $\phi_1 \neq \frac{-\pi}{2}$, то есть соединяющий фокусы отрезок  и граница $\partial \Gamma_j$ пересекаются по множеству ненулевой меры, тогда функция $R_j(\rho)$ должна быть нечетной, но тогда с точностью до постоянного множителя $\Phi_j(\phi) = se_\nu(\phi)$. Эта особенность потребует отдельного рассмотрения собственных значений.

Если же  $\phi_j < 0$ выполняется для каждого $j=1,2,3$ и углы разные, то мы получим, что $\Phi_j(\phi)$ являются одной и той же функцией $se_\nu(\phi)$, причем из условий Кирхгофа $se'_\nu(\frac{\pi}{2}) = 0$, то есть $\nu \in \mathbb{Z}$, но тогда условия $se_\nu(\phi_j) = 0, \ j=1,2,3$ могут усложнить задачу вплоть до полной невозможности найти решения.

Рассмотрим случай $\phi_j = -\frac{\pi}{2}$. Тогда  каждый $\Gamma_j$ определяется в эллиптической системе координат как $(\rho, \phi) \in [0,\rho_0] \times [-\frac{\pi}{2}, \frac{\pi}{2}]$, и решения $R_j(\rho)\Phi_j(\phi)$ принимают вид \eqref{eq:fun}.
Тогда для собственных значений справедливы проведенные в  \ref{sec:ch2/sec4/subs2/subs1} и \ref{sec:ch2/sec4/subs3/subs1} рассуждения.

\clearpage
