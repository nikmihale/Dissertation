\chapter{Самосопряженные расширения оператора Лапласа на графах с одной точкой ветвления.}\label{ch:ch4}

\section{Введение.}\label{sec:ch4/sect1}
Рассмотрим граф $\Gamma$, определенный как объединение $n$ отрезков $\Gamma_j=[0,l_j], j \in \{1, 2, \ldots, n\}$: $\Gamma = \cup_{j=1}^n \Gamma_j$. Точку $Q$ будем называть точкой ветвления  $\Gamma$, если она является граничной точкой как минимум двух  $\Gamma_i, \Gamma_j$, $i \neq j$.

Пусть на $\Gamma$ задана Борелевская мера. Предположим, что ее ограничение на каждую  $\Gamma_j$ совпадает со стандартной Лебеговской мерой на $\mathbb{R}_{d_j}$. Тогда пространство комплекснозначных квадратично интегрируемых на $\Gamma$ по Лебеговской мере допускает представление $L_2(\Gamma) = \bigoplus L_2(\Gamma_j)$.

Положим $C_{0,0}^\infty(\Gamma)$ -- векторное пространство бесконечно дифференцируемых косплекснозначных функций на $\Gamma$ с компактным носителем, не содержащим точки ветвления многообразия $\Gamma$ и пусть $\textbf{L}_0 = \bigoplus \textbf{L}_0^j$ -- линейный оператор, определенный на $C_0^\infty(\Gamma)$, определенный как $\textbf{L}_0 u = \bigoplus \textbf{L}_0^j u_j $, где  $\textbf{L}_0 u = \Delta u$  и $\{u_j, j=1,\ldots, n\}$ -- ограничение функции $u$ на область $\Gamma_j$.

\section{Условия самосопряженности оператора Лапласа.}\label{sec:ch4/sect2}
Теорема Фон Неймана  (\cite{reed1980methods, kato2013perturbation}) позволяет описать множество самосопряженных расширений симметрических операторов. Мы хотим получить явное описание множества самосопряженных расширений оператора $\textbf{L}_0$ в терминах условий на линейном подпространстве в пространстве граничных значений
$$G = D(\textbf{L}_0^*) / \overline{D(\textbf{L}_0)} = \{ (u(0), u'(0) ) \}  =\mathbf{C}^{2n}.$$

\begin{theorem}
Оператор $\textbf{L}$ с областью определения 
$$D(\textbf{L}) = \left\{ u \in W_2^2(\Gamma): u'(0) = A u(0) \right\},$$
самосопряжен тогда и только тогда, когда матрица $A$ удовлетворяет равенству $A = A^*$.
\end{theorem} 
\begin{proof}
Пусть $u \in D(\textbf{L})$ и $v \in D(\textbf{L}_0^*)$, тогда справедливо равенство 
$$\left(\textbf{L} u, v \right)_H - \left(u, \textbf{L}_0^*v \right)_H = \left(u(0), v'(0)\right)_{\mathbf{C}^{2n}} - \left(u'(0), v(0)\right)_{\mathbf{C}^{2n}}.$$
Следовательно, $\left(\textbf{L} u, v \right)_H - \left(u, \textbf{L}_0^*v \right)_H = \left(u(0), v'(0) - A^* v(0)\right)_{\mathbf{C}^{2n}}$.
В силу произвольности $u(0)$ получаем, что условие $v'(0) = A^* v(0)$ необходимо и достаточно для включения $v \in D(\textbf{L}^*)$.
\end{proof}

\begin{theorem}
Оператор $\mathbf{L}$ самосопряжен тогда и только тогда, когда область определения $D(\mathbf{L})$ состоит из функций пространства $W_2^2(\Gamma)$, граничные значения удовлетворяют равенству $Z u'(0) + A u(0) = 0$, где ранг матрицы $(Z|A)$ равен $n$ и для матриц $A, Z$ справедливо $Z A^* = A Z^*$.
\end{theorem}
\begin{proof}
Обозначим $\left\{ \left( 
\begin{pmatrix}
           u(0) \\
           u'(0) 
\end{pmatrix}
\right)_{2n \times 1} = \Phi_{2n \times n} h_{n \times 1}
\right\}$ набор решений линейных уравнений $Z u'(0) + A u(0) = 0$, где $\Phi_{2n \times n}$ -- фундаментальная матрица и $h_{n \times 1}$ -- матрица независимых постоянных. Тогда систему линейных уравнений можно переписать в виде 
$$ 0 = Z u'(0) + A u(0) = \left( A \  Z \right) \begin{pmatrix}
           u(0) \\
           u'(0) 
\end{pmatrix} = 
 \left( A \  Z \right) \Phi_{2n \times n} h_{n \times 1}.$$ В силу произвольности независимых постоянных констант $h_{n \times 1}$ справедливо равенство $ \left( A \  Z \right) \Phi_{2n \times n} = 0_{n \times n}$, из которого можно получить соотношение 
\begin{equation}\Phi^T \begin{pmatrix}
         A^T \\
	Z^T 
\end{pmatrix}_{2n \times n} = 0_{n\times n}.
\label{eq:ch4:eq1}
\end{equation}
Если $u \in D(\mathbf{L})$ и область определения оператора $\mathbf{L}$ определяется системой $Z u'(0) + A u(0) = 0$, тогда для любого $v \in D(\mathbf{L}_0^*)$ справедливо равенство 
$$\left( \mathbf{L} u, v\right)_H - \left( u, \mathbf{L}_0^* v\right)_H = \left( h, \Phi^T 
\begin{pmatrix}
           0 & 1 \\
           -1 & 0
\end{pmatrix}_{2n \times 2n} 
\begin{pmatrix}
           \overline{v(0)} \\
           \overline{v'(0)}
\end{pmatrix}_{2n \times 1} \right).$$
Для $v \in D(\mathbf{L}^*)$ должно выполняться равенство 
\begin{equation}
\left( \mathbf{L} u, v\right)_H - \left( u, \mathbf{L}_0^* v\right)_H =0.
\label{eq:ch4:eq2}
\end{equation}
Положим $\overline{V} = \begin{pmatrix}
           \overline{v_1} & \ldots & \overline{v_n} \\
           \overline{v'_1} & \ldots & \overline{v'_n} \\
\end{pmatrix}_{2n \times n}$ -- базис линейного пространства $D(\mathbf{L^*}) / \overline{D(\mathbf{L}_0)}$, тогда каждый столбец матрицы $\overline{V}$ удовлетворяет равенству \eqref{eq:ch4:eq2}, следовательно 
\begin{equation}
\Phi^T \begin{pmatrix}
           0 & 1 \\
           -1 & 0
\end{pmatrix}_{2n \times 2n} \overline{V}_{2n \times n} = 0_{n \times n}.
\label{eq:ch4:eq3}
\end{equation}

Из \eqref{eq:ch4:eq1} и \eqref{eq:ch4:eq3} следует, что матрица $V$ может быть выбрана как $V=
\begin{pmatrix}
           -Z^* \\
           A^*
\end{pmatrix}. $

Оператор $\mathbf{L}$ самосопряжен тогда и только тогда, когда $D(\mathbf{L})=D(\mathbf{L}^*)$, тогда для матрицы $V$, в которой по столбцам записаны базисные векторы подпространства  $D(\mathbf{L}^*)/\overline{D(\mathbf{L}_0)}$ условие $D(\mathbf{L})=D(\mathbf{L}^*)$ выполняется тогда и только тогда, когда в матрице $V$ по столцам записаны базисные векторы пространства $D(\mathbf{L})/\overline{D(\mathbf{L}_0)}$, то есть, если каждый ее столбец удовлетворяет системе $Z u'(0) + A u(0) =0$. Это эквивалентно системе уравнений 
$\left( A \  Z \right) \begin{pmatrix}
           -Z^* \\
           A^*
\end{pmatrix} = 0$.
\end{proof}

\section{Спектр оператора Лапласа на графах с одной точкой ветвления.}\label{sec:ch4/sect3}
Пусть метрический граф $\Gamma = \cup_{j=1}^3 \Gamma_j$ состоит из трех отрезков $\Gamma_j = [0, l_j]$, соединенных в граничной точке $x=0$. 

Рассмотрим собственные функции $u$ самосопряженного расширения оператора $\Delta$, обращающиеся в нуль на трех граничных точках. В точке ветвления потребуем выполнение равенства $Z u'(0) + A u(0) =0$, и пусть $Z A^* = A Z^*$.

Решениями уравнений $u_j''(x) +k^2 u(x) = 0$ на каждом $\Gamma_j$ являются функции $A_j e^{ i  k x} + B_j  e^{ -i  k x}$, где из граничного условия $u_j(l_j)=0 = $ следует $B_j = -A_j e^{ 2 i  k l_j}$.
Тогда условие в точке ветвления $x=0$ запишется с учетом равенств $u_j(0) = A_j + B_j, u'_j(0) = (i k) (A_j-B_j)$ в следующем виде:
\begin{multline*}
Z \begin{pmatrix}
(i k) (A_1-B_1) \\ (i k) (A_2-B_2) \\ (i k) (A_3-B_3)
\end{pmatrix}
+A \begin{pmatrix}
A_1+B_1 \\ A_2+B_2 \\ A_3+B_3
\end{pmatrix}
= 
(i k Z + A) \begin{pmatrix}
 A_1 \\  A_2 \\  A_3
\end{pmatrix}+ ((-i k) Z + A) 
\begin{pmatrix}
B_1 \\ B_2  \\ B_3
\end{pmatrix}
 = \\ 
\left(
i k Z + A 
\right)
\begin{pmatrix}
 A_1 \\  A_2 \\  A_3
\end{pmatrix} + 
(i k Z - A )
\begin{pmatrix}
e^{ 2i  k l_1} 	& 0 			& 0 \\ 
0 		& e^{ i  k l_2}  	& 0 \\ 
0		& 0			& e^{ i  k l_3}
\end{pmatrix}
\begin{pmatrix}
 A_1 \\  A_2 \\  A_3
\end{pmatrix} = \begin{pmatrix}
0  \\ 0 \\ 0 
\end{pmatrix}.
\end{multline*}
Эта система линейных уравнений относительно констант $\begin{pmatrix}
 A_1 \\  A_2 \\  A_3
\end{pmatrix}$ имеет решения тогда и только тогда, когда $k$ обращает в нуль детерминант матрицы $\left(
i k Z + A 
\right)
 + 
(i k Z - A )
\begin{pmatrix}
e^{ 2i  k l_1} 	& 0 			& 0 \\ 
0 		& e^{ i  k l_2}  	& 0 \\ 
0		& 0			& e^{ i  k l_3}
\end{pmatrix}.
$
Домножим оба слагаемых справа на невырожденную матрицу $\begin{pmatrix}
e^{ -i  k l_1} 	& 0 			& 0 \\ 
0 		& e^{ -i  k l_2}  	& 0 \\ 
0		& 0			& e^{ -i  k l_3}
\end{pmatrix}$, после чего с учетом тождества $e^{i  k x} = \cos k x + i \sin k x$ перепишем условие на собственные значения $k$ в виде

\begin{equation}
0=F(k)=\det  \left(
k Z 
\begin{pmatrix}
\cos k l_1 	& 0 			& 0 \\ 
0 		& \cos k l_2  	& 0 \\ 
0		& 0			& \cos k l_3
\end{pmatrix}
 - A \begin{pmatrix}
\sin k l_1 	& 0 			& 0 \\ 
0 		& \sin k l_2  	& 0 \\ 
0		& 0			& \sin k l_3
\end{pmatrix} \right).
 \label{eq:ch4:eigenF}
 \end{equation}


\section{Пример}\label{sec:ch4/sect3/subsect1}
Пусть $u$ является решением уравнения $\Delta u + k^2 u =0 $ на  $\Gamma$ и в точке $x=0$ выполняется $\left. u \right|_{\Gamma_1} = \left. u \right|_{\Gamma_2} = \left. u \right|_{\Gamma_3}$, т.е. $u_1(0) = u_2(0) = u_3(0)$ . Для внешних производных функций $u_j$ потребуем равенства $u'_1(0) + u'_2(0) + u'_3(0)=0$. Эти  условия известны как \textit{ условия Кирхгофа}, и им соответствует система уравнений $Z u'(0) + A u(0) =0$, где 
\[
\begin{array}{cc}
A = \begin{pmatrix}
1		& -1 			& 0 \\ 
0 		& 1		  	& -1 \\ 
0		& 0			& 0
\end{pmatrix} , \quad
Z = \begin{pmatrix}
0		& 0			& 0 \\
0		& 0			& 0 \\
1		& 1			& 1
\end{pmatrix} .
\end{array}
\]
Согласно предыдущему пункту, собственные значения $k^2$ связаны с нулями выражения
\eqref{eq:ch4:eigenF}. Подставим в выражение $A$ и $Z$, чтобы получить равенство
\begin{multline*}
F(k) = \det \begin{pmatrix}
-\sin{k l_1}			& \sin{k l_2} 			& 0 \\ 
0 				& -\sin{k l_2}		  	& \sin{k l_3} \\ 
k \cos{k l_1}		& k \cos{k l_2}			& k \cos{k l_3}
\end{pmatrix} = \\ k \left( \sin {k l_1} \sin{k l_2} \cos{k l_3} + \sin{k l_1} \cos{k l_2} \sin{k l_3} + \cos{k l_1} \sin{k l_2}  \sin{k l_3} \right) = 0.
\end{multline*}
Собственные значения $k^2$ получаются из нулей выражения $F(k)$ возведением в квадрат.

В случае $l_1=l_2=l_3=l$ выражение для $F(k)$ принимает вид $F(k) = 3 k \cos{k l} \sin^2 k l$, тогда собственные значения получаются как 
$ k^2 = \frac{ \pi^2 m^2}{4l^2},   \ m \in \mathbb{N}.$

\section{Пример для частного случая вложения трилистника}\label{sec:ch4/sect3/subsect2}
Потребуем, чтобы касательная плоскость к графику функции $u(x)$ на $\Gamma$ была определена в том числе в точке ветвления $x=0$. 
Такая плоскость существует только при условии непрерывности $u$: $u_1(0)=u_2(0)=u_3(0)$, следовательно матрица $A$ имеет тот же вид, что и в предыдущем примере.
Для существования касательной плоскости в точке ветвления потребуем, чтобы касательные векторы к графику $u(\Gamma)$ в точке $x=0$ находились в одной плоскости.

Пусть $\Gamma$ вложен в $\mathbb{R}^2$ таким образом, что 
$$
\angle( \Gamma_1, \Gamma_2) = \alpha, \quad
\angle( \Gamma_2, \Gamma_3) = \beta, \quad
\angle( \Gamma_3, \Gamma_1) = \gamma,
$$
где $\alpha + \beta + \gamma = 2 \pi$. Без ограничения общности можем считать, что для $\Gamma_j$ справедлива параметризация 
$\Gamma_1 = \{(-t , 0), \ t \in [0, l_1]\}$,
$\Gamma_2 = \{(-t \cos \alpha ,-t \sin \alpha ), \ t \in [0, l_2]\}$,
$\Gamma_3 = \{(-t\cos(\alpha+\beta) ,-t\sin(\alpha+\beta)), \ t \in [0, l_3]\}$.
Тогда график $u(\Gamma) \in \mathbb{R}^3$ представляет собой объединение 
\[
\begin{array}{ll}
u(\Gamma_1) = \{(-t , 0, u_1(t)), t \in [0, l_1]\}, \\
u(\Gamma_2) = \{(-t \cos \alpha ,-t \sin \alpha, u_2(t) ), t \in [0, l_2]\}, \\
u(\Gamma_3) =\{(-t\cos(\alpha+\beta) ,-t\sin(\alpha+\beta), u_3(t)), t \in [0, l_3]\}.
\end{array}
\]
Внешние относительно точки ветвления  $x=0$ касательные векторы $a_1, a_2, a_3$ тогда определены как
\[
\begin{array}{ll}
a_1 = (1 , 0, u'_1(0)), \\
a_2 = (\cos \alpha , \sin \alpha, u'_2(0) ),\\
a_3 =(\cos(\alpha+\beta) ,\sin(\alpha+\beta), u'_3(0)).
\end{array}
\]
Условие принадлежности векторов $a_1, a_2, a_3$ одной плоскости можно записать в виде 
\begin{multline*}
\det (a_1, \ a_2, \ a_3) = \det 
\begin{pmatrix}
1 		&	\cos \alpha 	& 	\cos(\alpha+\beta) 	\\
0		&	\sin \alpha		&	\sin(\alpha+\beta)	\\
u'_1(0)	&	u'_2(0) 		&	u'_3(0)
\end{pmatrix} = \\ u'_1(0) \sin \beta + u'_3(0) \sin \alpha - u'_2(0) \sin (\alpha+\beta) = \\
 u'_1(0) \sin \beta + u'_2(0) \sin \gamma + u'_3(0) \sin \alpha = 0.
\end{multline*}
Таким образом, граничные условия в точке ветвления могут быть записаны в виде $Z u'(0) + A u(0) =0$ при
\[
\begin{array}{cc}
A = \begin{pmatrix}
1		& -1 			& 0 \\ 
0 		& 1		  	& -1 \\ 
0		& 0			& 0
\end{pmatrix} , \quad
Z = \begin{pmatrix}
0			& 0				& 0 \\
0			& 0				& 0 \\
\sin \beta		& \sin \gamma		& \sin \alpha
\end{pmatrix} .
\end{array}
\]
Подставляя выражения матриц в \eqref{eq:ch4:eigenF} получим следующее
\begin{statement}
Решения уравнения $\Delta u + k^2 u =0 $ на  $\Gamma$, для которых определена в точке ветвления $x=0$ касательная плоскость, соответствуют собственным числам $k^2$ таким, что $k$ является нулем функции
\begin{multline*}
F(k)=k \left(
\sin{\alpha} \sin{k l_1} \sin{k l_2} \cos{k l_3} + 
\sin{\gamma} \sin{k l_1} \cos{k l_2} \sin{k l_3} 
\right. \\ \left. + 
\sin{\beta} \cos{k l_1} \sin{k l_2} \sin{k l_3} 
 \right)
 \end{multline*}
 
 В случае $l_1=l_2=l_3=l$ выражение для $F(k)$ принимает вид $F(k) = k \cos{k l} \sin^2 (k l) \left( \sin \alpha + \sin \beta + \sin \gamma \right)$, тогда для собственных значений справедливо выражение $ k^2 = \frac{ \pi^2 m^2}{4l^2},   \ m \in \mathbb{N}$.
\label{stat:1dimAnglesSpectrum}
\end{statement}

\clearpage
