\chapter*{Введение}                         % Заголовок
\addcontentsline{toc}{chapter}{Введение}    % Добавляем его в оглавление

%\newcommand{\actuality}{{\textbf\actualityTXT}}
%\newcommand{\progress}{\textbf{\progressTXT}}
%\newcommand{\aim}{{\textbf\aimTXT}}
%\newcommand{\tasks}{\textbf{\tasksTXT}}
%\newcommand{\novelty}{\textbf{\noveltyTXT}}
%\newcommand{\influence}{\textbf{\influenceTXT}}
%\newcommand{\methods}{\textbf{\methodsTXT}}
%\newcommand{\defpositions}{\textbf{\defpositionsTXT}}
%\newcommand{\reliability}{\textbf{\reliabilityTXT}}
%\newcommand{\probation}{\textbf{\probationTXT}}
%\newcommand{\contribution}{\textbf{\contributionTXT}}
%\newcommand{\publications}{\textbf{\publicationsTXT}}

\newcommand{\actuality}{\section*{\actualityTXT}}
\newcommand{\progress}{\section*{\progressTXT}}
\newcommand{\aim}{\section*{\aimTXT}}
\newcommand{\tasks}{\section*{\tasksTXT}}
\newcommand{\novelty}{\section*{\noveltyTXT}}
\newcommand{\influence}{\section*{\influenceTXT}}
\newcommand{\methods}{\section*{\methodsTXT}}
\newcommand{\defpositions}{\section*{\defpositionsTXT}}
\newcommand{\reliability}{\section*{\reliabilityTXT}}
\newcommand{\probation}{\section*{\probationTXT}}
\newcommand{\valueT}{\section*{\valueTXT}}
\newcommand{\volumeAndStructure}{\section*{\volumeAndStructureTXT}}
\newcommand{\contents}{\section*{\contentsTXT}}
\newcommand{\gratitude}{\section*{\gratitudeTXT}}

\newcommand{\contribution}{\section*{\contributionTXT}}
\newcommand{\publications}{\section*{\publicationsTXT}}


{\actuality} 
Диссертация посвящена исследованию квантовых и классических бильярдов в областях, ограниченных софокусными квадриками.

Первая тема, которую мы рассмотрим, --- это асимптотика собственных значений оператора Шрёдингера для свободной квантовой частицы. 

В последние годы был достигнут значительный прогресс в понимании гипотезы Биркгофа об интегрируемых плоских бильярдах, например, см.~\cite{bm2022}.
С другой стороны, были найдены строгие доказательства неинтегрируемости некоторых бильярдов, таких как эллиптические бильярды в сильном магнитном поле~\cite{bm2020, bm2019}. Неинтегрируемые бильярды, как и бильярды в магнитном поле, исследуются давно, см.~\cite{berry1985, berry1986}.
Изменение бильярдной области с эллипса на, например, стадион, немедленно приводит к хаотической динамике, \cite{bunimovich1974, stockmann2000}.

Хорошо известно, что классический бильярд в ограниченной софокусными квадриками области интегрируем. Интегрируемость  этой системы следует из того, что помимо сохранения полной энергии существует другая постоянная величина, а именно произведение угловых моментов относительно общих фокусов граничных квадрик.
В случае, если фокусы совпадают, степень этого дополнительного интеграла можно понизить, поскольку угловой момент относительно центра круговой области сохраняется.
Однако, в работе \cite{wref13} было показано, что для бильярда в круговом секторе сохраняемой величиной является не  угловой момент, а квадрат углового момента относительно вершины угла.


Недавние работы А.\,Т.~Фоменко и В.\,В.~Ведюшкиной (см. \cite{wref6,wref7,wref8} , а также другие публикации этих авторов) вновь привлекли к этой теме внимание специалистов. 
В частности, в работе \cite{wref6} изучались особенности бильярда в кольце, ограниченном софокусными эллипсами (`эллиптическое кольцо'). 
В работе \cite{Fok15} в качестве плоских биллиардов рассматриваются также области (`эллиптический сектор'), ограниченные эллипсом с большой полуосью $0 <r_0$ с фокальным расстоянием $\delta$ и 
одной ветвью софокусной гиперболы  (область $A_\delta$), обозначенная как  $A_1$ в \cite{Fok15},
или двумя ветвями софокусных гипербол (область $B_\delta$), которая обозначалась как  $A_0'$ в \cite{Fok15},
Все упомянутые эллипсы и гиперболы имеют общие фокусы в точках $(\pm \delta, 0)$.

В настоящей работе рассматривается соответствующая квантовая система, а именно изучается спектр оператора Шрёдингера в этой области и в ее накрытиях. 
При стремлении $\delta$ к $0$, области $A_\delta$ и $B_\delta$  стремятся к круговому сектору с некоторым центральным углом. Аналогично, при стремлении $\delta$ к $0$ эллиптическое кольцо стремится к круговому кольцу. 
Возникает естественный вопрос об установлении асимптотики уровней энергии при $\delta \to 0$. 
Для приведенных типов областей получены точные решения стационарного уравнения Шрёдингера, а именно собственные функции и соответствующие уровни энергии. Наше исследование направлено на вычисление асимптотики последних при устремлении фокального расстояния $\delta$ к нулю в областях $A_\delta$, $B_\delta$ и в эллиптическом кольце.

Вторая тема, которую мы рассмотрим, --- это развитие теории математических бильярдов.
Одним из возможных направлений являются рассмотрение бильярда с неплоской метрикой. Здесь мы упомянем интегрируемые биллиарды на софокусных столах на плоскости с метрикой Минковского для свободной частицы, рассмотренные В. Драговичем, М. Раднович [127] и Е.Е.Каргиновой [128, 129]. В присутствии центрального потенциала типа Гука бильярд на софокусном столе также сохраняет интегрируемость: один из примеров был разобран в работе А.И.Скворцова и В.В.Ведюшкиной [130].

К ряду других содержательных результатов приводит ослабление условия на выпусклость углов: если предположить, что граница бильярдного стола может иметь углы $\frac{3\pi}{2}$, тогда результирующий псевдоинтегрируемый бильярд имеет совместные поверхности уровня первых интегралов, негомеоморфные торам. Упомянем здесь работы В.Драговича и М.Раднович [131-133], а также В.А.Москвина [134,135]. 

Перспективными направлениями также являются класс бильярдов с проскальзыванием вдоль границы стола, который был предложен А.Т.Фоменко в работе [63] и так называемые бильярдные игры [100]. Отметим также бильярды на склеенных из плоских столов $CW$- комплексах с перестановками [Ведюшкина].

Автором совместно с Ф.\,Ю.\,Попеленским был обнаружен новый класс интегрируемых бильярдов.
Пусть область $\Omega$ ограничивается набором софокусных квадрик и разбивается дугами квадрик того же семейства на области $\Omega_i$. Припишем каждой области $\Omega_i$ коэффициент $n_i$, имеющий смысл показателя преломления.
Рассмотрим движение материальной точки в области $\Omega$: будем считать, что на внешней границе $\Omega$ движение подчиняется закону `угол падения равен углу отражения', а на общей границе областей $\Omega_i$ и $\Omega_j$ выполняется соотношение $n_i \cos \theta_i = n_j \cos \theta_j$. Гдесь $\theta_i, \theta_j$ -- углы, которые образуют отрезки траектории с нормалью к кривой $C \subset (\partial \Omega_i \cap \partial \Omega_j)$, если $\theta_i$ и $\theta_j$ корректно определены. В работе для такой системы исследуется интеграл движения $\Xi$. Дополнительно в работе описаны слоения изоэнергетического многообразия на поверхности уровня интеграла $\Xi$ для двух разбиений областей $\Omega$. 

{\aim} 
Диссертационная работа преследует следующие цели:
\begin{enumerate}[beginpenalty=10000] % https://tex.stackexchange.com/a/476052/104425
  \item Вычисление асимптотики собственных значений оператора Шрёдингера в зависимости от расстояния между фокусами для:
  \begin{itemize}[beginpenalty=10000] % https://tex.stackexchange.com/a/476052/104425
  \item конечно-листного накрытия области, ограниченной двумя софокусными эллипсами (<<эллиптическое кольцо>>)
  \item симметричной относительно горизонтальной оси области, ограниченной дугой эллипса и ветвью софокусной  гиперболы (<<эллиптический сектор>> вида $A_\delta$)  (см. рис. \ref{fig:intro_quantum_domains}).
  \item области, ограниченной отрезком горизонтальной оси, дугой эллипса и ветвями двух софокусных с эллипсом гипербол (<<эллиптический сектор>> вида $B_\delta$)  (см. рис. \ref{fig:intro_quantum_domains}).
  \end{itemize}
   \begin{figure}[ht]
    \centerfloat{
        \hfill
        \subcaptionbox{<<эллиптический сектор>> вида $A_\delta$}{%
\includegraphics[width=3.5cm]{right1.pdf}}
        \hfill
        \subcaptionbox{<<эллиптический сектор>> вида $B_\delta$}{%
\includegraphics[width=3.5cm]{up1.pdf}}
        \hfill
    }
    \caption{Софокусные столы для квантовой задачи.}\label{fig:intro_quantum_domains}
\end{figure}
  
  \item Для бильярда с косинусным законом преломления на софокусных столах:
    \begin{itemize}[beginpenalty=10000]
	  \item Исследовать динамическую систему на интегрируемость. Доказать существование дополнительного интеграла движения $\Xi$.
	  \item Исследовать поверхности постоянного уровня первого интеграла $\Xi$ для двух разбиений бильярдного стола (см. рис. \ref{fig:intro_classical_domains}).
  \begin{figure}[ht]
    \centerfloat{
        \hfill
%        \subcaptionbox{Область для задачи А}
        \hfill
%        \subcaptionbox{Область для задачи Б}{%
\includegraphics[width=1.7cm]{images/ch4/section3_circular/domain.pdf}
%{
        \hfill
    }
    \caption{Разбиения софокусных столов для классической задачи.}\label{fig:intro_classical_domains}
\end{figure}
    \end{itemize}
\end{enumerate}

{\defpositions}
\begin{enumerate}[beginpenalty=10000] % https://tex.stackexchange.com/a/476052/104425
  \item Вычислены и приведены коэффициенты разложения собственных значений $E_{k,m}$ стационарного оператора Шрёдингера по степеням половины фокального расстояния $\delta$ для:
   \begin{itemize}[beginpenalty=10000] % https://tex.stackexchange.com/a/476052/104425
  \item конечно-листного накрытия области, ограниченной двумя софокусными эллипсами (<<эллиптическое кольцо>>) 
  \item симметричной относительно горизонтальной оси области, ограниченной дугой эллипса и ветвью софокусной  гиперболы (<<эллиптический сектор>> вида $A_\delta$, см. рис. \ref{fig:intro_quantum_domains})
  \item области, ограниченной отрезком горизонтальной оси, дугой эллипса и ветвями двух софокусных с эллипсом гипербол (<<эллиптический сектор>> вида $B_\delta$, см. рис. \ref{fig:intro_quantum_domains})
  \end{itemize}
  Коэффициенты получены для всех натуральных $k$ и $m$ с точностью до второго порядка включительно.

  \item Вычислена явная формула для интеграла движения $\Xi$ для бильярда в области, ограниченной эллипсом, подчиняющегося преломлениям согласно косинусному закону на дугах софокусных квадрик.
  \item Построены поверхности постоянного уровня дополнительного интеграла $\Xi$ для свободной частицы в эллипсе при преломлении траектории согласно косинусному закону на софокусном эллипсе (см. рис. \ref{fig:intro_classical_domains}). Поверхности построены для регулярных и критических значений интеграла $\Xi$.
   \item Построены поверхности постоянного уровня дополнительного интеграла $\Xi$ для свободной частицы в круге при преломлении траектории согласно косинусному закону на окружности меньшего радиуса и сегменте радиальной прямой (см. рис. \ref{fig:intro_classical_domains}). Поверхности построены для регулярных и критических значений интеграла $\Xi$.
\end{enumerate}
%В папке Documents можно ознакомиться с решением совета из Томского~ГУ
%(в~файле \verb+Def_positions.pdf+), где обоснованно даются рекомендации
%по~формулировкам защищаемых положений.


%настоящей работы является получение этой асимптотики с точностью до второго порядка. Ожидается, что в нулевом порядке уровни энергии будут совпадать с результатами~\cite{wref13} для кругового сектора и~\cite[\S~207, с.~276]{wref11} для накрытия кругового кольца кратности $p=1$.

% из первой статьи про косинусное преломление:
%Интерес к составным бильярдам возникает в связи с активно разрабатываемой  в школе А.~Т.~Фоменко теорией бильярдов со сложной топологией, таких как бильярдные книжки. Современное состояние этой теории см. в обзоре [1].  
%
%Рассмотрим ограниченную эллипсом область, разбитую дугой софокусной квадрики на две области с разными плотностями, но постоянными внутри каждой из областей. Тогда мы можем рассмотреть бильярдную траекторию, которая при пересечении границы раздела двух сред меняет направление по закону Снеллиуса [2]: отношение синусов углов падения и преломления равно обратному отношению плотностей сред. 
%Экспериментальная компьютерная проверка демонстрирует, что такой бильярд неинтегрируем. 
%
%С другой стороны, в физике известны законы преломления другого вида. 
%В задачах теплопроводности направление векторов плотностей теплового потока на границе раздела двух сред определяется отношением тангенсов [3, 4].
%В некоторых задачах оптики встречаются законы, формулируемые в терминах отношения косинусов углов падения и преломления [5, 6].
%
%В настоящей работе мы рассматриваем бильярд в эллипсе, разделенном дугами софокусных квадрик на несколько областей $\Omega_i$ с постоянными в них плотностями $n_i$, при этом закон преломления задан равенством $n_1 \cos{\theta_1} = n_2 \cos{\theta_2}$. Мы покажем, что в полученная система будет интегрируемой (см. пункты 5 и 6) и предъявим дополнительный интеграл. В некоторых случаях значения этого дополнительного интеграла принадлежат не прямой, а окружности, см. пункт 7.
%Для~достижения поставленной цели необходимо было решить следующие задачи:

%В настоящей работе  эта асимптотика получена с точностью до второго порядка. В нулевом порядке уровни энергии совпадают с результатами~\cite{wref13} для кругового сектора и~\cite[\S~207, с.~276]{wref11} для накрытия кругового кольца кратности $p=1$.


{\novelty} 
Все положения диссертации, выносимые на защиту, являются оригинальными и получены автором самостоятельно  или при равноценном вкладе с соавторами. Кроме того, диссертация содержит следующие вспомогательные результаты, которые также являются новыми:
  \begin{itemize}[beginpenalty=10000] % https://tex.stackexchange.com/a/476052/104425
  \item впервые исследован косинусный закон преломления бильярдной траектории на софокусных столах
  \item для этого закона приведена методика построения бифуркационных диаграмм нового типа, одновременно учитывающих все возможные значения <<оптических>> параметров областей
  \item впервые получены особые поверхности, соответствующие одновременным бифуркациям разных типов в разных частях бильярдного стола
  \end{itemize}

%Для уровней энергии получены явные асимптотические выражения, в частности, формулы ($\ref{eq:funcRing}$) и ($\ref{eq:valRing}$) для $p$-листного накрытия эллиптического кольца, а также формулы ($\ref{eq:fun}$) и ($\ref{eq:val}$) для области $A_{\fixme{\varepsilon}}$ и формулы  ($\ref{eq:funB}$) и ($\ref{eq:valB}$) для области  $B_{\fixme{\varepsilon}}$. Для особых случаев для областей $A_{\fixme{\varepsilon}}$ и $B_{\fixme{\varepsilon}}$ справедливы формулы (\ref{eq:valS1}) и (\ref{eq:valS2}). 
%Приведенные асимптотики для собственных значений в зависимости от расстояния между фокусами справедливы с точностью до второго порядка включительно и подтверждаются численными экспериментами.

%Интегрируемость классических бильярдов в областях $A_{\fixme{\varepsilon}}$ и $B_{\fixme{\varepsilon}}$ следует из существования сохраняющейся величины в дополнение к полной энергии.
%В приложении~\ref{app:A} приведен квантовый аналог этой величины, сохраняющейся для квантовых бильярдов в этих же областях.
%
%\begin{enumerate}[beginpenalty=10000] % https://tex.stackexchange.com/a/476052/104425
%  \item Впервые \ldots
%  \item Впервые \ldots
%  \item Было выполнено оригинальное исследование \ldots
%\end{enumerate}
%
{\methods} В работе используются элементы теории Штурма и теории специальных функций, методы теории краевых задач и математического анализа. В исследовании бильярда с косинусным законом преломления на софокусных квадриках применяются методы теории топологической классификации интегрируемых
гамильтоновых систем с одной и двумя степенями свободы, построенной А.Т. Фоменко, Х. Цишангом, А.В. Болсиновым и многими другими.

{\influence} 
Диссертация имеет теоретический характер.
Полученные результаты могут быть использованы при исследовании собственных функций и собственных значений оператор Лапласа в областях, ограниченных софокусными квадриками.
Разработанные методы позволяют получить асимптотику собственных значений в том числе для областей, деформация которых не удовлетворяет необходимым условиям вариационной формулы Адамара (например, если рассматривать для <<эллиптических секторов>> $A_\delta$ и $B_\delta$ в качестве параметра деформации величину $\delta^2$).

Ценность исследования бильярда с косинусным законом преломления заключается в расширении класса бильярдных задач, что позволяет строить новые интересные примеры интегрируемых систем, топология слоений Лиувилля для которых весьма нетривиальна. 
В рамках нового научного направления в теории биллиардов рассмотрена возможность неодновременных перестроек в областях бильярдного стола, соответствующих разным оптическим плотностям, а также одновременные неодинаковые перестройки в них.
Кроме того, получен динамический бильярд с первым интегралом, множество значений которого не является интервалом.

{\probation}
Основные результаты диссертации обоснованы в виде строгих математических доказательств и прошли апробацию на следующих научных конференциях и семинарах:

\begin{enumerate}%[beginpenalty=10000]
\item Студенческая школа-конференция <<Математическая весна --  2023>>, Нижний Новгород, Россия, 27-30 марта 2023;

\item Ломоносовские чтения 2023, Россия, 4-14 апреля 2023;

\item XXX Международная научная конференция студентов, аспирантов и молодых учёных <<Ломоносов-2023>>,  Москва, Россия, 10-21 апреля 2023;

\item Воронежская зимняя математическая школа С.Г. Крейна, Воронеж, Россия, 26-30 января 2024;

\item Студенческая школа-конференция <<Математическая весна -- 2024>>, Нижний Новгород, Россия, 25-28 марта 2024;

\item XXXI Международная конференция студентов, аспирантов и молодых ученых <<Ломоносов--2024>>, Москва, Россия, 12-26 апреля 2024;

%\item Современные геометрические и топологические методы, Сириус, Россия, 14-19 мая 2024;	%там не по диссеру был доклад
\item Семинар “Дифференциальная геометрия и приложения” под руководством акад. А.Т. Фоменко на механико-математическом факультете МГУ имени М.В. Ломоносова, 25 ноября 2024;


\item XXXII Международная научная конференция студентов, аспирантов и молодых ученых <<Ломоносов--2025>>, Москва, Россия, 11-25 апреля 2025.
\end{enumerate}
%{\contribution} Автор принимал активное участие \ldots


\ifnumequal{\value{bibliosel}}{0}
{%%% Встроенная реализация с загрузкой файла через движок bibtex8. (При желании, внутри можно использовать обычные ссылки, наподобие `\cite{vakbib1,vakbib2}`).
    {\publications} Основные результаты по теме диссертации изложены
    в четырех работах \nocite{nikulin2023spektr611484954, nikulin2024asymptotic617844539, vestnikLatest, sbornikLatest}
    X из которых изданы в журналах, рекомендованных ВАК,
    X "--- в тезисах докладов.
}%
{%%% Реализация пакетом biblatex через движок biber
    \begin{refsection}[bl-author, bl-registered]
        % Это refsection=1.
        % Процитированные здесь работы:
        %  * подсчитываются, для автоматического составления фразы "Основные результаты ..."
        %  * попадают в авторскую библиографию, при usefootcite==0 и стиле `\insertbiblioauthor` или `\insertbiblioauthorgrouped`
        %  * нумеруются там в зависимости от порядка команд `\printbibliography` в этом разделе.
        %  * при использовании `\insertbiblioauthorgrouped`, порядок команд `\printbibliography` в нём должен быть тем же (см. biblio/biblatex.tex)
        %
        % Невидимый библиографический список для подсчёта количества публикаций:
        \phantom{\printbibliography[heading=nobibheading, section=1, env=countauthorvak,          keyword=biblioauthorvak]%
        \printbibliography[heading=nobibheading, section=1, env=countauthorwos,          keyword=biblioauthorwos]%
        \printbibliography[heading=nobibheading, section=1, env=countauthorscopus,       keyword=biblioauthorscopus]%
        \printbibliography[heading=nobibheading, section=1, env=countauthorconf,         keyword=biblioauthorconf]%
        \printbibliography[heading=nobibheading, section=1, env=countauthorother,        keyword=biblioauthorother]%
        \printbibliography[heading=nobibheading, section=1, env=countregistered,         keyword=biblioregistered]%
        \printbibliography[heading=nobibheading, section=1, env=countauthorpatent,       keyword=biblioauthorpatent]%
        \printbibliography[heading=nobibheading, section=1, env=countauthorprogram,      keyword=biblioauthorprogram]%
        \printbibliography[heading=nobibheading, section=1, env=countauthor,             keyword=biblioauthor]%
        \printbibliography[heading=nobibheading, section=1, env=countauthorvakscopuswos, filter=vakscopuswos]%
        \printbibliography[heading=nobibheading, section=1, env=countauthorscopuswos,    filter=scopuswos]}%
        %
        \nocite{*}%
        %
        {\publications} Основные результаты по теме диссертации изложены в~\arabic{citeauthor}~печатных работах, 
        \arabic{citeauthorvak} из которых изданы в журналах, рекомендованных ВАК%
        \ifnum \value{citeauthorscopuswos}>0%
            , \arabic{citeauthorscopuswos} "--- в~периодических научных журналах, индексируемых Web of~Science и Scopus%
        \fi%
        \ifnum \value{citeauthorconf}>0%
            , \arabic{citeauthorconf} "--- в~тезисах докладов.
        \else%
            .
        \fi%
        \ifnum \value{citeregistered}=1%
            \ifnum \value{citeauthorpatent}=1%
                Зарегистрирован \arabic{citeauthorpatent} патент.
            \fi%
            \ifnum \value{citeauthorprogram}=1%
                Зарегистрирована \arabic{citeauthorprogram} программа для ЭВМ.
            \fi%
        \fi%
        \ifnum \value{citeregistered}>1%
            Зарегистрированы\ %
            \ifnum \value{citeauthorpatent}>0%
            \formbytotal{citeauthorpatent}{патент}{}{а}{}%
            \ifnum \value{citeauthorprogram}=0 . \else \ и~\fi%
            \fi%
            \ifnum \value{citeauthorprogram}>0%
            \formbytotal{citeauthorprogram}{программ}{а}{ы}{} для ЭВМ.
            \fi%
        \fi%
        % К публикациям, в которых излагаются основные научные результаты диссертации на соискание учёной
        % степени, в рецензируемых изданиях приравниваются патенты на изобретения, патенты (свидетельства) на
        % полезную модель, патенты на промышленный образец, патенты на селекционные достижения, свидетельства
        % на программу для электронных вычислительных машин, базу данных, топологию интегральных микросхем,
        % зарегистрированные в установленном порядке.(в ред. Постановления Правительства РФ от 21.04.2016 N 335)
    \end{refsection}%
    \begin{refsection}[bl-author, bl-registered]
        % Это refsection=2.
        % Процитированные здесь работы:
        %  * попадают в авторскую библиографию, при usefootcite==0 и стиле `\insertbiblioauthorimportant`.
        %  * ни на что не влияют в противном случае
        \nocite{vakbib2}%vak
        \nocite{patbib1}%patent
        \nocite{progbib1}%program
        \nocite{bib1}%other
        \nocite{confbib1}%conf
        \nocite{nikulin2023spektr611484954}
	\nocite{nikulin2024asymptotic617844539}
	\nocite{vestnikLatest}
	\nocite{sbornikLatest}

    \end{refsection}%
        %
        % Всё, что вне этих двух refsection, это refsection=0,
        %  * для диссертации - это нормальные ссылки, попадающие в обычную библиографию
        %  * для автореферата:
        %     * при usefootcite==0, ссылка корректно сработает только для источника из `external.bib`. Для своих работ --- напечатает "[0]" (и даже Warning не вылезет).
        %     * при usefootcite==1, ссылка сработает нормально. В авторской библиографии будут только процитированные в refsection=0 работы.
        \nocite{nikulin2023spektr611484954}
	\nocite{nikulin2024asymptotic617844539}
	\nocite{vestnikLatest}
	\nocite{sbornikLatest}
}


\ifsynopsis
{\volumeAndStructure} Диссертация состоит из~введения,
\fixme{XX} глав, заключения и~приложения. Полный объем диссертации
\fixme{ХХХ}~страниц с~\fixme{ХХ}~рисунками и~\fixme{5}~таблицами. Список
литературы содержит \fixme{ХХX}~наименование.
\else
\begin{refsection}[bl-author, bl-registered]
	% Это refsection=3.
	% Для подсчёта позиций списка литературы при группировке работ автора
	%
	% Невидимый библиографический список для подсчёта количества публикаций:
	\printbibliography[heading=nobibheading, section=0, env=counter, keyword=bibliofull]%
	%
	\nocite{*}%
	%% authorother
	%		\nocite{bib1}%
	%		\nocite{bib2}%
	%
	{\volumeAndStructure} Диссертация состоит из~введения,
	\formbytotal{totalchapter}{глав}{ы}{}{} и заключения.
	Полный объём диссертации составляет
	\formbytotal{TotPages}{страниц}{у}{ы}{}, включая
	\formbytotal{totalcount@figure}{рисун}{ок}{ка}{ков} и
	\formbytotal{totalcount@table}{таблиц}{у}{ы}{}.
	Список литературы содержит
	\formbytotal{citenum}{наименован}{ие}{ия}{ий}.
\end{refsection}%
\fi
 % Характеристика работы по структуре во введении и в автореферате не отличается (ГОСТ Р 7.0.11, пункты 5.3.1 и 9.2.1), потому её загружаем из одного и того же внешнего файла, предварительно задав форму выделения некоторым параметрам

\section*{Содержание работы}
Во \underline{\textbf{введении}} формулируется цель работы, кратко излагаются ее результаты и содержание. 
Условно диссертация может быть разделена на две части. Главы 1--2 посвящены исследованию асимптотики собственных значений для квантовых бильярдов на софокусных столах. Главы 3--5 целиком посвящены развитию новой динамической системы, в основе которой лежит косинусный закон преломления траектории.

% актуальность
%исследований, проводимых в~рамках данной диссертационной работы,
%приводится обзор научной литературы по~изучаемой проблеме,
%формулируется цель, ставятся задачи работы, излагается научная новизна
%и практическая значимость представляемой работы. В~последующих главах
%сначала описывается общий принцип, позволяющий \dots, а~потом идёт
%апробация на частных примерах: \dots  и~\dots.

В \underline{\textbf{первой главе}} приведены сведения, необходимые для исследования асимптотики уровней энергии оператора Шрёдингера на софокусных столах при близком к нулю расстоянии между фокусами.
Рассматривается уравнение свободной квантовой частицы с потенциалом <<бесконечной ямы>>. Такой подход устанавливает соответствие между  решениями стационарного уравнения Шрёдингера и собственными  функциями $\psi$ оператора Лапласа с условием Дирихле на границе области $\Omega \subset \mathbb{R}^2$.

Напомним, что в полярной системе координат $(x,y) = (r \cos \phi, r \sin \phi)$ в предположении $\psi(r, \phi) = R(r) \Phi(\phi)$ уравнение Гельмгольца расщепляется в систему дифференциальных уравнений на функции $R(r)$ и $\Phi(\phi)$. В частности, радиальная составляющая $R(r)$ удовлетворяет дифференциальному уравнению Бесселя. 
Ввиду значительной роли функций Бесселя в дальнейших соображениях, в главе приводится основная теория функций Бесселя.
Связь собственных значений оператора Лапласа и нулей функций Бесселя для задачи в полярной системе координат проиллюстрирована на примере свободной квантовой частицы в круге. 
Также приводится анализ задачи в круговом кольце и секторе, поскольку эти области являются предельными для аналогичных областей в эллиптической системе координат $(x,y)=(c \cosh \rho \cos \phi, c \sinh \rho \sin \phi)$ с фокусами в точках $(\pm c,0)$ при $c \to 0$.
Уравнение Гельмгольца в новой системе координат также расщепляется: если $\psi(\rho, \phi) = R(\rho) \Phi(\phi)$, то функции $\Phi(\phi)$ и $R(\rho)$ удовлетворяют \textit{угловому} и \textit{радиальному уравнению Матьё}, соответственно.
Ключевым свойствам решений этих уравнений, \textit{функциям Матьё}, посвящен отдельный раздел первой главы. 
%Приводятся понятия \textit{характеристической экспоненты} $\nu$ из теории Флоке и \textit{характеристических значений} $a_n(q), b_n(q)$ для целого порядка $\nu=n$ и $\lambda_\nu(q)$ для нецелого $\nu$. 
Приводятся элементы теории Флоке и теории Штурма. 
%Глава продолжается приведением необходимой теории функций Матьё. А именно, решение углового уравнения Матьё рассматривается с точки зрения теории Флоке, что позволяет записать решение в виде $F_\nu(\phi) = e^{i \nu \phi} P(\phi)$, где функция $P$ имеет тот же период $\pi$, что и коэффициенты углового уравнения Матьё.
%Из теории Штурма следует ограничение на разделяющий параметр $\zeta$, поскольку при $q\neq 0$ существует не более чем одно периодическое решение с периодом $\pi$ или $2\pi$. А именно, имеет место следующая классификация функций Матьё целого порядка
%\begin{table} [htbp]%
%    \centering
%    \caption{Периодические функции Матьё целого порядка}%
%    \label{tab:table1}% label всегда желательно идти после caption
%	%    \renewcommand{\arraystretch}{1.5}%% Увеличение расстояния между рядами, для улучшения восприятия.
%    \begin{SingleSpace}
%	\begin{tabular}{||c | c | c | c||} 
%            \toprule     %%% верхняя линейка
%            $\zeta$	&   \begin{tabular}{c}Периодическое решение\\ углового уравнения Матьё\footnotemark[3]\end{tabular} &   Период  & Четность функции \\
%            \midrule 
%		$a_{2n}(q)$                   &   $ce_{2n}(z, q)$               & период $\pi$     & четная \\ \hline
%		$a_{2n+1}(q)$                 &   $ce_{2n+1}(z, q)$             & антипериод\footnotemark[4] $\pi$ & четная \\ \hline
%		$b_{2n+1}(q)$                 &   $se_{2n+1}(z, q)$             & антипериод $\pi$ & нечетная \\ \hline          
%		$b_{2n+2}(q)$                 &   $se_{2n+2}(z, q)$             & период $\pi$     & нечетная \\ 
%		            \bottomrule %%% нижняя линейка
%        \end{tabular}%
%    \end{SingleSpace}
%\end{table}
%\footnotetext[3]{В табл. 1 приведены только собственные функции периода $\pi$ или $2\pi$.}
%\footnotetext[4]{Антипериод $\pi$: $f(x+\pi) = -f(x)$.}
%Похожая классификация существует для функций Матьё нецелого порядка, и она также включена в первую главу работы.

%Для нашей задачи основной интерес представляют разложения угловых функций Матьё в ряды Фурье и связь радиальных функций Матьё с угловыми. Как оказывается, радиальные функции допускают запись в виде бесконечной суммы функций Бесселя, при этом коэффициенты в этой сумме будут связаны с коэффициентами Фурье угловой функции Матьё с теми же параметрами $\zeta$ и $q$. 

%Рассматривается также предел характеристических значений $\lambda_\nu(q)$ и собственных функций при $q \to 0$. Аналогичные соображения приведены для непериодических функций Матьё.
%Для радиальных функций Матьё справедливо, что замена $\phi \mapsto i \phi$ превращает угловые уравнения Матьё к радиальным, однако аналогичный ряд по гиперболическим функциям сходится медленно. Но существует разложение радиальных функций Матьё в бесконечную сумму функций Бесселя, где коэффициенты связаны с коэффициентами Фурье угловой функции с теми же параметрами.
%
В отличие от квантового бильярда в круге, в эллипсе появляется дополнительное условие на решения стационарного уравнения Шрёдингера. 
Пусть $\Omega \in \mathbb{R}^2$ и  $J$ -- это часть соединяющего фокусы $(\pm c,0)$ отрезка. Предположим, что $J$ содержится во внутренности области $\Omega$.
Тогда гладкое решение $\psi(\rho,\phi)$ удовлетворяет следующим условиям:
\begin{align}
& \psi(0,\phi) = \psi(0,-\phi)\notag \\
 &   \qquad\qquad\qquad\qquad     \text{\em (непрерывность сдвига через  $J$)} \label{eq:intro_disp}, \\[10pt]
 &   \frac{\partial}{\partial \rho} \psi( \rho, \phi)|_{\rho \to 0} = - \frac{\partial}{\partial \rho} \psi(\rho, - \phi)|_{\rho \to 0} \notag \\
 &\qquad\qquad\qquad\qquad   \text{\em (непрерывность производной через $J$)}\label{eq:intro_grad}, 
\end{align}

Один из разделов первой главы посвящен задаче об асимптотике уровня энергии свободной квантовой частицы в эллипсе с точки зрения теории специальных функций. Эта задача призвана сформировать базовый инструментарий, который получит дальнейшее развитие и применение при рассмотрении задач из следующей главы. В частности, показано, что при $c\to0$ собственные значения в эллипсе имеют своим пределом значения в круге. Этот же характер асимптотического поведения сохраняется и для иных областей, рассмотренных во второй главе.
В завершение первой главы приводится общий вид собственных функций для <<эллиптического кольца>>, анализ задачи продолжен в первом разделе следующей главы.

Асимптотика собственных значений приводится в начале \underline{\textbf{второй главы}}. 
Доказана следующая теорема:
\begin{theorem}
	Значение $\varkappa^2_{k,m}(\delta), k, m \in \mathbb{N}$, зависит от половины фокусного расстояния $\delta$ с точностью до $o(\delta^2)$ следующим образом:
{\small
\begin{equation}
\varkappa^2_{k,m}(\delta) = \left[
\begin{array}{cc}
\dfrac{\alpha_{\nu, m}^2}{r_0^2} + \delta^2 \dfrac{\alpha_{\nu, m}^3}{8 \nu r_0^4} \left. \frac{
\frac{\nu-2}{\nu-1}
\left(
W_{\nu-2, \nu}(u) + W_{\nu, \nu-2}(u)
\right)- 
\frac{\nu+2}{\nu+1}
\left(
W_{\nu+2, \nu}(u) + W_{\nu, \nu+2}(u)
\right)
}{ \frac{\partial W_{\nu,\nu}(u)}{\partial u} }\right|_{u=\alpha_{\nu, m}},  & -\nu \in \mathbb{N} \setminus \{1, 2\}; \\

\dfrac{\alpha_{2, m}^2}{r_0^2} - \delta^2 \dfrac{\alpha_{2, m}^3}{12  r_0^4} \left. \frac{
	\left(
	W_{4, 2}(u) + W_{2, 4}(u)
	\right)
}{ \frac{\partial W_{2,2}(u)}{\partial u} }\right|_{u=\alpha_{2, m}}, & -\nu=2; \\

\dfrac{\alpha_{1, m}^2}{r_0^2} - \delta^2 \dfrac{3 \alpha_{1, m}^3 }{16 r_0^4}
\left. \frac{
	\left(
	W_{3, 1}(u) + W_{1, 3}(u)
	\right)
}{ \frac{\partial W_{1,1}(u)}{\partial u} }\right|_{u=\alpha_{1, m}}, & -\nu=1; \\


\dfrac{\alpha_{0, m}^2}{r_0^2} - \delta^2 \dfrac{\alpha_{0, m}^3}{4r_0^4} \left. \frac{
 \left( W_{2, 0}(u) + W_{0, 2}(u) \right)
}{ \frac{\partial W_{0,0}(u)}{\partial u} }\right|_{u=\alpha_{0, m}}, & \nu=0;   \\

\dfrac{\alpha_{1, m}^2}{r_0^2} - \delta^2 \dfrac{\alpha_{1, m}^3}{16 r_0^4} \left. \frac{ \left( W_{3, 1}(u) + W_{1, 3}(u) \right)
}{ \frac{\partial W_{1,1}(u)}{\partial u} }\right|_{u=\alpha_{1, m}},    & \nu=1;\\

\dfrac{\alpha_{\nu, m}^2}{r_0^2} + \delta^2 \dfrac{\alpha_{\nu, m}^3}{2 r_0^4} \left. \frac{
\frac{1}{4(\nu-1)} \left( W_{\nu-2, \nu}(u) + W_{\nu, \nu-2}(u) \right) - \frac{1}{4(\nu+1)} \left( W_{\nu+2, \nu}(u) + W_{\nu, \nu+2}(u) \right)
}{ \frac{\partial W_{\nu,\nu}(u)}{\partial u} }\right|_{u=\alpha_{\nu, m}},
&   \left[
\begin{aligned}
\nu &\in \mathbb{N} \setminus \{1\},\\
\nu &\notin \mathbb{Z}.
\end{aligned}
\right.
\end{array}
\right.,
%\tag{6}
\label{eq:valRing}
\end{equation}
}
где $\alpha_{\nu, m}$ --- $m$-й нуль функции $W_{\nu, \nu}(u)$. Последняя определяется как $W_{a, b}(u) = Y_a(u)J_b(\lambda u) - Y_a(\lambda u)J_b(u)$ для $\lambda = \frac{r_1}{r_0}$.
\label{th:sect2_th4}
\end{theorem}
Отметим, что при $\delta \to 0$ область приближается к круговому кольцу, а собственные значения совпадают с теми, что были получены для предельной области. 
Большой раздел второй главы посвящен рассмотрению квантовой свободной частицы в областях двух типов: симметричной $A_\varepsilon$ и несимметричной $B_\varepsilon$ (см. рис. \ref{fig:intro_AB_examples}), где симметрия подразумевается относительно горизонтальной оси $Ox$. 
\begin{figure}[ht]
    \begin{minipage}[b][][b]{0.49\linewidth}\centering
        \includegraphics[width=0.8\linewidth]{right1.pdf} \\ 
        Множество $A_\varepsilon$, $\phi_0<\pi/2$
    \end{minipage}
    \hfill
    \begin{minipage}[b][][b]{0.49\linewidth}\centering
        \includegraphics[width=0.8\linewidth]{up1.pdf}  \\
        Множество $B_\varepsilon$, $\phi_0<\phi_1<\pi/2$
    \end{minipage}
\caption{Примеры множеств $A_\varepsilon$ и $B_\varepsilon$.}
\label{fig:intro_AB_examples}
\end{figure}
Обе области при $\delta \to 0$ имеют своим пределом круговой сектор, результат для которого получен в работе \cite{wref13}.

Для симметричной области $A_\varepsilon$ вычислена следующая асимптотика собственных значений:
\begin{theorem} 
Для собственных функций $\psi_{k, m}(\rho, \phi)$ оператора   $\hat{H}$ справедливы выражения
\begin{equation}
\psi_{k, m}(\rho, \phi) = 
\left[
\begin{array}{ll}
    Ce_\nu(\rho, q) ce_\nu(\phi, q) ,   &    \text{нечетный $k \geq 1$}\\
    Se_\nu(\rho, q) se_\nu(\phi, q) ,   &    \text{четный $k \geq 2$}\\
\end{array}
\right.\label{eq:fun}
\end{equation}
с параметрами
\[
\nu = \nu_{k,m} = \nu_0+ \varepsilon^2 \nu_1 + o(\varepsilon^2),  \quad q=q_{k,m} = \dfrac{\varkappa_{k,m}^2 r_0^2 \varepsilon^2}{4},
\]
где
$$\nu_0 = \frac{\pi k}{2\phi_0}\text{\ \  и\  \  }\nu_1=\frac{\alpha_{\nu_0, m}^2}{4} \frac{\pi k \sin 2\phi_0}{\pi^2 k^2 - 4\phi_0^2}  .$$
Для соответствующих собственных  значений  $\varkappa^2_{k, m}$ справедливы равенства
\begin{equation}
\varkappa^2_{k, m} = \dfrac{\alpha_{\nu_0, m}^2}{r_0^2} +
\varepsilon^2 \dfrac{\alpha_{\nu_0, m}^3}{2 r_0^2}\dfrac{\varkappa_1 }{ \left.\frac{\partial J_{\nu_0}(u)}{\partial u}\right|_{u=\alpha_{\nu_0, m}} } + o(\varepsilon^2),\label{eq:val}
\end{equation}
где
\begin{equation*}
    \varkappa_1 = 
    \left[
\begin{array}{ll}
\frac{J_{\nu_0-2}(\alpha_{\nu_0, m})}{4(\nu_0-1)} - \frac{J_{\nu_0+2}(\alpha_{\nu_0, m})}{4(\nu_0+1)} 
  - \nu_1 \left.\frac{\partial J_\nu}{\partial \nu}\right|_{\nu=\nu_0}(\alpha_{\nu_0, m}),\qquad \text{для нечетных $k\geq 3$ ;} \\[10pt]
\frac{(\nu_0 - 2)J_{\nu_0-2}(\alpha_{\nu_0, m})   }{4\nu_0 (\nu_0-1)} -\\
\qquad - \frac{(\nu_0 + 2)J_{\nu_0+2}(\alpha_{\nu_0, m})}{4\nu_0 (\nu_0+1)}  
- \nu_1 \left.\frac{\partial J_\nu}{\partial \nu}\right|_{\nu = \nu_0}(\alpha_{\nu_0, m}), \qquad \ \ \!    \text{для четных $k \geq 2$}.        
\end{array}
\right.
\end{equation*}
Здесь $\alpha_{\nu_0,m}$ --  $m$-й нуль функции Бесселя первого рода $J_{\nu_0}(x)$.
\end{theorem} 
Случай $k=1$ для области $A_\varepsilon$ при $\phi_0=\pi/2$ рассматривается отдельно. Для этого случая формула имеет следующий вид:
\begin{align}
   & \varkappa_{k, m}^2 = \frac{\alpha_{1, m}^2}{r_0^2} - \varepsilon^2 \frac{\alpha_{1, m}^3}{16r_0^2} 
    \frac{J_3(\alpha_{1, m})}{\left.\frac{\partial J_1 (u)}{\partial u}\right|_{u=\alpha_{1, m}}} 
    + o(\varepsilon^2), \label{eq:valS1}
    \end{align}

Для несимметричной области $B_\varepsilon$  аналогичная асимптотика подчиняется следующей формуле:
\begin{theorem}
В области  $B_\varepsilon$ для собственных функций $\psi_{k, m}(\rho, \phi)$ и собственных значений $\varkappa^2_{k, m}$ оператора $\hat{H}$ справедливы выражения

\begin{align}
&\psi_{k, m}(\rho, \phi) = 
    Se_\nu(\rho, q) \biggl( ce_\nu(\phi_0, q) se_\nu(\phi, q) -ce_\nu(\phi, q) se_\nu(\phi_0, q) \biggr) ,  \label{eq:funB}
\end{align}
с параметрами
\begin{align*}    
    & q=q_{k,m} = \frac{\varkappa_{k,m}^2 r_0^2 \varepsilon^2}{4}, \\ 
&\nu = \nu_{k,m} = \frac{\pi k}{\phi_1-\phi_0} +\varepsilon^2 \frac{\alpha_{\nu_0, m}^2}{4} \frac{\pi k (\sin 2\phi_1 - \sin 2 \phi_0)}{\pi^2k^2-(\phi_1-\phi_0)^2} + o(\varepsilon^2) ,  
\end{align*}
где
\begin{align*}
& \nu_0 = \frac{\pi k}{\phi_1-\phi_0},\text{\ \ и \ \ }
\nu_1= \frac{\pi k (\sin 2\phi_1 - \sin 2 \phi_0)}{\pi^2k^2-(\phi_1-\phi_0)^2} .
\end{align*}
Для соответствующих собственных значений справедливы равенства
\begin{align}
\varkappa_{k, m}^2 ={}& \frac{\alpha_{\nu_0, m}^2}{r_0^2} +  \varepsilon^2 \frac{\alpha_{\nu_0, m}^3}{2 r_0^2}\frac{1}{ \left.\frac{\partial J_{\nu_0}(u)}{\partial u}\right|_{u=\alpha_{\nu_0, m}} }  
 \biggl(\frac{(\nu_0 - 2)J_{\nu_0-2}(\alpha_{\nu_0, m})   }{4\nu_0 (\nu_0-1)} -
\notag \\ 
&{}- \frac{(\nu_0 + 2)J_{\nu_0+2}(\alpha_{\nu_0, m})}{4\nu_0 (\nu_0+1)} 
- \nu_1 \left.\frac{\partial J_\nu}{\partial \nu}\right|_{\nu = \nu_0}(\alpha_{\nu_0, m})
    \biggr) + o(\varepsilon^2).\label{eq:valB}
\end{align}
Здесь $\alpha_{\nu_0,m}$ -- $m$-й нуль функции Бесселя первого рода $J_{\nu_0}(x)$.
\end{theorem}

Случай $k=1$ для области $B_\varepsilon$ для $(\phi_0, \phi_1)=(0,\pi)$  рассматривается отдельно. Для нее имеет место следующая формула:
\begin{align}
    \varkappa_{1, m}^2& = \frac{\alpha_{1, m}^2}{r_0^2} - \varepsilon^2 \frac{3\alpha_{1, m}^3}{16r_0^2} 
    \frac{J_3(\alpha_{1, m})}{\left.\frac{\partial J_1 (u)}{\partial u}\right|_{u=\alpha_{1, m}}} 
    + o(\varepsilon^2).  \label{eq:valS2}
\end{align}

В завершение главы приведена постоянная наблюдаемая величина для квантовой свободной частицы на софокусных столах.


\underline{\textbf{Третья глава}} содержит основные понятия теории интегрируемого математического бильярда, а также формулировку косинусного закона преломления.

Приведем косинусный закон преломления как изложено в работе. 
Пусть две области $\Omega_i$ и $\Omega_j$ граничат по кривой $C$. 
Показатели преломления для этих областей равны $n_i$ и $n_j$, соответственно.
Будем считать, что движение материальной точки при достижении кривой $C$ подчиняется следующим правилам (далее мы будем ссылаться на них как на модифицированный закон преломления $(\ast)$).
\begin{itemize}
\item[1.] Выполнено соотношение $n_i \cos \theta_i = n_j \cos \theta_j $, где $\theta_i, \theta_j \in [0,\frac{\pi}{2} ]$ -- углы, которые образуют отрезки траектории   в соответствующих областях с нормалью к кривой $C$, если $\theta_i$ и $\theta_j$ корректно определены.
\item[2.] Если $n_i > n_j$ и материальная точка, двигаясь в области $\Omega_i$, достигает кривой $C$, причем в точке пересечения траектории с кривой $C$ выполнено неравенство $\cos \theta_i > \frac{n_j}{n_i}$, то происходит полное внутреннее отражение траектории в область $\Omega_i$ по закону <<угол падения равен углу отражения>> (в этом случае угол $\theta_j$ не определен, поскольку $\frac{n_i}{n_j} \cos \theta_i > 1$). 
\item[3.] В предыдущих двух пунктах два соседних отрезка траектории с общей точкой на кривой $C$ лежат по разные стороны от нормали к кривой $C$ в этой точке.
\item[4.] Если $n_i > n_j$ и материальная точка, двигаясь в области $\Omega_j$, достигает кривой $C$, причем $\theta_j = 0$, тогда $\cos \theta_i = \frac{n_j}{n_i}$ и материальная точка продолжает движение в области $\Omega_i$ вдоль любого из двух возможных направлений, образующих угол $\theta_i$ с нормалью к кривой $C$.
\item[5.] Аналогично при $n_i < n_j$.
\end{itemize}

\begin{figure}[!htb]
\minipage{0.32\textwidth}
\includegraphics[width=\linewidth]{images/ch4/section1/example1.pdf}
    \caption{Иллюстрация к пункту 1.}
    \label{fig:intro_example1}
\endminipage\hfill
\minipage{0.32\textwidth}
\includegraphics[width=\linewidth]{images/ch4/section1/example2.pdf}
    \caption{Иллюстрация к пункту 2.}
    \label{fig:intro_example2}
\endminipage\hfill
\minipage{0.32\textwidth}
\includegraphics[width=\linewidth]{images/ch4/section1/example3.pdf}
    \caption{Иллюстрация к пункту 4.}
    \label{fig:intro_example3}
\endminipage\hfill
\end{figure}

Под \textit{софокусным столом} будем понимать область, ограниченную дугами эллипсов и гипербол с общими фокусами в точках $(\pm c, 0)$, $c > 0$. В частности каждая из \textit{софокусных квадрик} $Q_\lambda$ удовлетворяет уравнению $\dfrac{x^2}{a^2-\lambda} +\dfrac{y^2}{b^2-\lambda} = 1$ для некоторого параметра  $\lambda \in (0, a^2)$.
В теории классического математического бильярда в эллипсе в качестве постоянной движения часто рассматривают параметр \textit{каустики} -- такой софокусной квадрики, которая является касательной к каждому звену бильярдной траектории. Этот параметр может быть вычислен явно как функция координат точки и компонент вектора скорости:
$$\Lambda(x, y, v_x, v_y) =  \dfrac{a^2 v_y^2 + b^2v_x^2 - (x v_y-y v_x)^2}{v_x^2 + v_y^2}.$$

Для бильярдов на софокусных столах, подчиняющихся этому закону, приводится выражение для постоянной движения.
А именно, если преломление по закону $(\ast)$ происходит на дугах непересекающихся квадрик $Q_1, \ldots, Q_k$, тогда справедливо следующее утверждение:

\begin{theorem}
Пусть внутренность эллипса разбита попарно непересекающимися дугами софокусных квадрик на области $\Omega_1, \ldots, \Omega_k$. Перенумеруем области так, чтобы общие границы имели только области с соседними номерами. Пусть $\lambda_j$ -- параметр софокусной квадрики, разделяющей $\Omega_j$ и $\Omega_{j+1}$, $j=1, \ldots, k-1$. Здесь и далее показатель преломления для области $\Omega_j$ обозначается через $n_j$.
Определим функцию $\Xi(x, y, v_x, v_y)$ по формуле: 
\begin{equation*}
\Xi(x, y, v_x, v_y) = \left[
\begin{array}{ll}
    \Lambda(x, y, v_x, v_y) n_1^2, \qquad  \ \ \qquad   \text{ если } (x,y) \in \Omega_1 ;
    \\
    \Lambda(x, y, v_x, v_y) n_p^2 + \sum_{j=1}^{p-1} \lambda_j(n_j^2-n_{j+1}^2), \\
     \qquad \qquad \qquad \qquad \qquad \qquad  \text{ если } (x,y) \in \Omega_p \text{ для } 1 < p \leq k. 
\end{array}
\right.
\end{equation*}
Функция $\Xi(x, y, v_x, v_y)$ является константой на траекториях бильярда с косинусным законом преломления $(\ast)$.
\label{th:intro_non_intersecting}
\end{theorem}

Два возможных варианта таких разбиений показаны на рис. \ref{fig:intro_example4} и \ref{fig:intro_example5}. 
\begin{figure}[!htb]
\minipage{0.45\textwidth}
   \includegraphics[width=1\textwidth]{images/ch4/section1/multiple ellipses.pdf}
    \caption{Взаимное расположение областей $\Omega_1, \ldots, \Omega_k$.}
    \label{fig:intro_example4}
\endminipage\hfill
\minipage{0.45\textwidth}
    \includegraphics[width=0.9\textwidth]{images/ch4/section1/multiple hyperbolas.pdf}   
    \caption{Взаимное расположение областей $\Omega_1, \ldots, \Omega_k$.}
    \label{fig:intro_example5}
\endminipage\hfill
\end{figure}
Возникает \textbf{задача А}:  описать слоение изоэнергитического многообразия на поверхности уровня первого интеграла $\Xi$ для случаев, показанных на рис. \ref{fig:intro_example4} и \ref{fig:intro_example5}. В следующей главе подробно рассматривается случай двух областей, разделенных одним софокусным эллипсом (см. рис. \ref{fig:intro_example4} при $k=2$). Динамика этой системы и перестройки поверхностей постоянного значения интеграла $\Xi$ уже в этом случае очень нетривиальны.

Случай пересекающихся квадрик оказывается гораздо сложнее. Каждой точке $A$ пересечения разделяющих среды квадрик требуется поставить в соответствие коэффициент $\gamma_A$, имеющий смысл \textit{коэффициента ветвления}. Величина $\gamma_A$ явно зависит от параметров пересекаюшихся в точке $A$ квадрик и параметров $n_j$ примыкающих областей $\Omega_j$.

%В частности, дополнительный интеграл принимает значения не в $\mathbb{R}$, а в фактор-группе $\mathbb{R}$ по  аддитивной подгруппе, допускающей явное описание.
%Если существует единственная точка пересечения преломляющих квадрик, первый интеграл $\Xi$ из теоремы формулируется иначе.
\begin{figure}[!htb]
\centering
     \includegraphics[width=0.35\textwidth]{images/ch4/section1/img2.pdf}
\caption{Взаимное расположение областей $\Omega_1,\ldots,\Omega_4$.}
    \label{fig:intro_example6}
\end{figure}

К примеру, для области с изображенным на рис. \ref{fig:intro_example6} разбиением введем коэффициент $\gamma_A$ в точке $A$, имеющий смысл коэффициента ветвления, по формуле $$\gamma_A = (\lambda_1 - \lambda_2) ( n_1^2 - n_2^2 + n_3^2 - n_4^2).$$

Определим вспомогательную функцию $\widetilde{\Xi}(x, y, v_x, v_y)$ 

\begin{equation*}
\widetilde{\Xi}(x, y, v_x, v_y) = \left[
\begin{array}{ll}
    \Lambda(x, y, v_x, v_y) n_1^2, \qquad  \  \ \qquad   \text{ если } (x,y) \in \Omega_1 
    \\
    \Lambda(x, y, v_x, v_y) n_p^2 + \sum_{j=1}^{p-1} \lambda_{\sigma(j)}(n_j^2-n_{j+1}^2), \\
     \qquad \qquad \qquad \qquad \qquad \qquad  \text{ если } (x,y) \in \Omega_p \text{ для } 1 < p \leq 4,
\end{array}
\right.
\end{equation*}
где $\sigma(j)$ -- номер квадрики, разделяющей $\Omega_j$ и $\Omega_{j+1}$. 
Неформально говоря, величина $\widetilde{\Xi}$ почти подходит на роль дополнительного интеграла, но имеет разрыв на дуге, разделяющей области $\Omega_1$ и $\Omega_4$. Можно проверить, что на любой бильярдной траектории, пересекающей эту дугу, функция $\widetilde{\Xi}$ испытывает один и тот же скачок, равный  $\pm \gamma_A$. 
Поэтому мы определим \textit{первый интеграл $\Xi(x, y, v_x, v_y)$ со значениями в $S^1= \mathbb{R}/\gamma_A \mathbb{Z}$ }по формуле $$\Xi(x, y, v_x, v_y) = \widetilde{\Xi}(x, y, v_x, v_y) \mod \gamma_A.$$
Эта величина на траекториях бильярда сохраняется.
Аналогичный подход справедлив и в других случаях единственной точки пересечения квадрик, на которых происходит преломление по правилу $(\ast)$.

Если границы раздела областей пересекаются по двум и более точкам, то имеет место общая закономерность: 
\textit{ 
\noindent Для каждой точки пересечения $A_i, i=1,\ldots,m$, определен коэффициент $\gamma_{A_i}$. Дополнительный интеграл \  $\Xi(x, y, v_x, v_y)$ принимает значения в $\mathbb{R}/(\gamma_{A_1} \mathbb{Z}+ \ldots + \gamma_{A_m} \mathbb{Z})$. Если $\gamma_{A_i}$ соизмеримы, т. е. всевозможные дроби $\dfrac{\gamma_{A_i}}{\gamma_{A_j}}$ --- рациональные числа (или бесконечность), то $\mathbb{R}/(\gamma_{A_1} \mathbb{Z}+ \ldots + \gamma_{A_m} \mathbb{Z}) = S^1$. Если же среди $\gamma_{A_i}$ есть пара с иррациональным отношением $\dfrac{\gamma_{A_i}}{\gamma_{A_j}}$, то подгруппа $\gamma_{A_1} \mathbb{Z}+ \ldots + \gamma_{A_m} \mathbb{Z}$ всюду плотна в $\mathbb{R}$. В этом случае дополнительный интеграл $\Xi(x, y, v_x, v_y)$ корректно определен, но использовать его для топологического анализа структуры траекторий представляется весьма затруднительным.}

%\begin{figure}[!htb]
%\centering
%   \includegraphics[width=0.2\textwidth]{images/ch4/section1/imgB.pdf}   
%    \caption{Взаимное расположение областей $\Omega_1, \Omega_2$.}
%    \label{fig:intro_example9}
%\end{figure}
%
%В рамках задачи Б будет рассмотрен в качестве бильярдной области $\Omega$  <<прямоугольник>>, изображенный на рис. \ref{fig:intro_example9}. Требуется описать как изоэнергетическое многообразие расслаивается на поверхности уровня первого интеграла $\Xi$.
%
%Можно сослаться на свои работы в автореферате. Для этого в файле
%\verb!Synopsis/setup.tex! необходимо присвоить положительное значение
%счётчику \verb!\setcounter{usefootcite}{1}!. В таком случае ссылки на
%работы других авторов будут подстрочными.
%Изложенные в третьей главе результаты опубликованы в~\cite{vakbib1, vakbib2}.
%Использование подстрочных ссылок внутри таблиц может вызывать проблемы.
%


\underline{\textbf{Четвертая глава}} посвящена рассмотрению задачи А: описать слоение изоэнергетического многообразия на поверхности уровня первого интеграла $\Xi$ для подчиняющейся закону $(\ast)$ бильярдной системы в области $\Omega = \Omega_{in} \cup \Omega_{out}$ (см. рис. \ref{fig:intro_problemA}).
\begin{figure}[!htb]
\centering
\includegraphics[scale=0.3]{images/ch4/section2/domain_problemA.pdf}
    \caption{Область $\Omega$ для задачи А.}
    \label{fig:intro_problemA}
\end{figure}

Для этой системы отрезки произвольной траектории, лежащие в области $\Omega_{in}$, касаются квадрики с параметром $\alpha_{in} \in (\lambda_1, a^2)$, а ее отрезки, лежащие в $\Omega_{out}$ -- вообще говоря, другой квадрики с параметром $\alpha_{out} \in (0, a^2)$. При этом параметры связаны соотношением $(\alpha_{out} - \lambda_1) n_{out}^2 = (\alpha_{in} - \lambda_1) n_{in}^2$.
Это тождество позволяет построить отображение $\alpha: \Xi \mapsto  L \in \mathbb{R}^2$ -- на прямую в плоскости с декартовыми координатами $(\alpha_{in}, \alpha_{out})$. 
А именно, для фиксированных $\lambda_1, n_{in}, n_{out}$ точка $\alpha(\Xi)$ лежит на прямой $L$, которая в декартовых координатах $(\alpha_{in}, \alpha_{out})$ задается уравнением
\begin{equation}
\alpha_{out} = \alpha_{in} \left(\frac{n_{in}}{n_{out}}\right)^2 + \lambda_1 \frac{n_{out}^2 - n_{in}^2}{n_{out}^2}.
\label{eq:L_line}
\end{equation}
Отметим, что прямая $L$ проходит через точку $(\alpha_{in}, \alpha_{out}) = (\lambda_1, \lambda_1)$.
Мы введем структурную диаграмму критических значений первого интеграла $\Xi$. 
\begin{figure}[!htb]
\centering
\includegraphics[scale=0.1]{images/ch4/section2/problemA_subdivisions.pdf}
    \caption{Подразбиение областей $D_1, \ldots, D_4$.}
    \label{fig:intro_problemA_subdivisions}
\end{figure}

Анализ слоения на поверхности уровня интеграла $\Xi$ проходит по следующей схеме. Сначала фиксируются параметры $\lambda_1, n_{in}, n_{out}$. Они определяют прямую $L$, при этом значение интерграла $\Xi$ однозначно определяет точку на этой прямой. Прямая $L$ пересекает некоторые области, изображенные на рис. \ref{fig:intro_problemA_subdivisions}. Разбиение на рисунке связано с различными типами траекторий (например, возможны траектории, для которых не определен $\alpha_{in}$ или $\alpha_{out}$).
Структура слоения определяется тем, как прямая $L$ пересекает изображенные области.
При этом нерегулярные значения интеграла $\Xi$ соответствуют точкам пересечения прямой $L$ с координатными линиями $\alpha_{in}, \alpha_{out} = b^2, a^2$. 
В главе доказана следующая теорема:
%Для удобства занумеруем регулярные области значений интеграла $\Xi$ на рис. \ref{fig:intro_causticTypesDiagram}.
%В ходе их описания мы узнаем как выглядят соответствующие области $\Omega$. Потом, когда будем описывать бифуркации, 
\begin{theorem} 
Областям $D_i^j$ в плоскости $(\alpha_{in}, \alpha_{out})$ соответствуют следующие поверхности $\Xi = \const$
\medskip
\begin{center}
\begin{tabular}{|c|c|c|}
\hline 
$D_1^1$  	& 	$\alpha_{in} \in (\lambda_1, b^2), \ \alpha_{out} \in (b^2, a^2)$			& сфера с 5 ручками; \\ \hline 
$D_1^2$  	& 	$\alpha_{in} \in (b^2, a^2), \ \alpha_{out} \in (\alpha_{in}, a^2)$				& сфера с 5 ручками; \\ \hline 
$D_1^3$  	& 	$\alpha_{in} \in (b^2, a^2), \ \alpha_{out} \in (b^2, \alpha_{in})$				& сфера с 5 ручками; \\ \hline 
$D_1^4$ 	& 	$\alpha_{in} \in (\lambda_1, b^2), \ \alpha_{out} \in (\alpha_{in}, b^2)$	& 2 дизъюнктных тора; \\ \hline 
$D_1^5$  	& 	$\alpha_{in} \in (\lambda_1, b^2), \ \alpha_{out} \in (\lambda_1, \alpha_{in})$	& 2 дизъюнктных тора; \\ \hline 
$D_1^6$  	& 	$\alpha_{in} \in (b^2, a^2), \ \alpha_{out} \in (\lambda_1, b^2)$			& сфера с 5 ручками; \\ \hline 
\hline
$D_2$  	& 	$(\alpha_{out} \in (0, \lambda_1), \ \alpha_{in} < \lambda_1)$ или & \\
		&  $(\alpha_{in} \in (0, \lambda_1), \ \alpha_{out} < \lambda_1)$				& 2 дизъюнктных тора; \\ \hline
 \hline
$D_3^1$  	& 	$\alpha_{in} \in (\lambda_1, b^2), \ \alpha_{out} > a^2$				& 2 дизъюнктных тора; \\ \hline 
$D_3^2$  	& 	$\alpha_{in} \in (b^2, a^2), \ \alpha_{out} > a^2 $					& 1 тор; \\ \hline 
\hline 
$D_4^1$  	& 	$\alpha_{in} > a^2, \ \alpha_{out} \in (b^2, a^2)$						& 2 дизъюнктных тора; \\ \hline 
$D_4^2$  	& 	$\alpha_{in} > a^2, \alpha_{out} \in (\lambda_1, b^2)$				& 2 дизъюнктных тора; \\ \hline 
\end{tabular}
\end{center}
\label{st:intron1_n2_surfaces}
\end{theorem} 
В завершение приведены особые поверхности, соответствующие пересечению прямой $L$ и выделенных на рис. \ref{fig:intro_problemA_subdivisions} сегментов координатных линий $\alpha_{in}, \alpha_{out} = b^2, a^2$. Приведенные 13 бифуркаций включают в себя <<двойные перестройки>>, соответствующие пересечению прямой $L$ с точками  $(b^2, a^2)$ и $(a^2, b^2)$. Для нетривиальных особых поверхностей приведены иллюстрации.



\underline{\textbf{Пятая глава}} посвящена рассмотрению задачи Б: описать слоения изоэнергитического многообразия на поверхности уровня первого интеграла $\Xi$ для подчиняющейся закону $(\ast)$ бильярдной системы в области $\Omega = \Omega_1 \cup \Omega_2$ (см. рис. \ref{fig:intro_domain}). Показатели преломлений $n_1$ для $\Omega_1$ и $n_2$ для $\Omega_2$ предполагаем фиксированными.
\begin{figure}[!htb]
\centering
\includegraphics[scale=0.3]{images/ch4/section3_circular/domain.pdf}
    \caption{Область $\Omega$ представляет собой объединение 2 областей.}
    \label{fig:intro_domain}
\end{figure}
Для этой системы в качестве постоянной движения, сохраняющейся в областях $\Omega_1$ и $\Omega_2$ до пересечения дуг $EF$ или $FG$ рассматривается величина $\rho^2 =  \frac{(x v_y - y v_x)^2}{v_x^2 + v_y^2}$. При этом на двух границах раздела сред $EF$ и $FG$ параметр каустики $\rho^2$ преобразуется по-разному:
\begin{statement}
Имеют место следующие соотношения для параметров $\rho_1, \rho_2$ в точке преломления:
\begin{align}
(\rho_1^2 - r_1^2)n_1^2 = (\rho_2^2 - r_1^2)n_2^2 \qquad & 
			\text{ при } (x,y) \in EF, 
			\label{eq:st1_eq1}
			\\
\rho_1^2 n_1^2 = \rho_2^2 n_2^2 \qquad  & \text{ при } (x,y) \in FG.
			\label{eq:st1_eq2}
\end{align}
\label{st:across_EF}
\end{statement}

Разобьем бильярдную траекторию в точках пересечения дуги $EF$ на фрагменты $T_k, k \geq 1$. 
Каждый фрагмент бильярдной траектории $T_k$ образует ломаную кривую в $\Omega \setminus EF$, где $EF  \subset \partial \Omega_1 \cap \partial \Omega_2$.
Введем на $\Omega \setminus EF$ функцию $\Xi(x, y, v_x, v_y)$ по формуле
\begin{equation}
\Xi(x, y, v_x, v_y) = \left[
\begin{array}{ll}
    \rho^2(x,y,v_x,v_y) n_1^2, &  \text{ если } (x,y) \in \Omega_1 \\
    \rho^2(x,y,v_x,v_y) n_2^2, & \text{ если } (x,y) \in \Omega_2.
\end{array}
\right.
\label{eq:XiDefinition}
\end{equation}
Эта функция постоянна в каждой точке фрагмента бильярдной траектории $T_k$, но на разных фрагментах значения могут различаться. 

Направления роста и убывания интеграла $\Xi$ можно проиллюстрировать на примере рис. \ref{fig:n1gtn2_Xi_growth} и \ref{fig:n1ltn2_Xi_growth}.
\begin{figure}[!htb]
\minipage{0.5\textwidth}
\centering
\includegraphics[width=5cm]{images/ch4/section3_circular/n1gtn2.png}
    \caption{Направление роста $\Xi$ при $n_1 > n_2$.}
    \label{fig:n1gtn2_Xi_growth}
\endminipage\hfill
\minipage{0.5\textwidth}
\centering
\includegraphics[width=5cm]{images/ch4/section3_circular/n1ltn2.png}
    \caption{Направление роста $\Xi$ при $n_1 < n_2$.}
    \label{fig:n1ltn2_Xi_growth}
\endminipage\hfill
\end{figure}

Рассмотрим значения интеграла $\Xi$ для двух последовательных фрагментов $T_k, T_{k+1}$. Выясним, как они связаны, или, что то же самое, как меняется интеграл $\Xi$ при преломлении на дуге $EF$. 

\begin{statement}
Значения $\Xi_k$ и $\Xi_{k+1}$  интеграла $\Xi$, соответствующие фрагментам траектории $T_k$ и $T_{k+1}$, различаются на величину $r_1^2(n_1^2-n_2^2)$.
\end{statement}
Значению интеграла $\Xi$ поставим в соответствие точку плоскости $\mathbb{R}^2$ по формуле 
\begin{equation}
\Xi \mapsto P(\Xi) = (\rho_1^2, \rho_2^2) = \left( \frac{\Xi}{n_1^2}, \frac{\Xi}{n_2^2} \right).
% = \left( \left. \rho^2 \right|_{\Omega_1} ,  \left. \rho^2 \right|_{\Omega_2}  \right).
\label{eq:XiMap}
\end{equation}
Тогда для фиксированных параметров $n_1, n_2$ отображением $P$ значения величины $\Xi$ отображаются на прямую $L \subset \mathbb{R}^2$. 
Параметризуем прямую $L: \xi \mapsto P(\xi) = (\xi \cos \theta, \xi \sin \theta)$. На ней можно выделить точку $L_1$, которая разделяет прямую на две части относительно параметра $\xi$ на прямой $L$ на две части: $\{ \xi < L_1\}$ и $\{ \xi > L_1\}$. Объединяя эти части по всевозможным прямым $L$, получим области в $\mathbb{R}^2$, изображенные на рис. \ref{fig:pt10:_lineDomains_simple}.
\begin{figure}[!htb]
\centering
\includegraphics[scale=0.07]{images/ch4/section3_circular/line_domains_simple.pdf}
    \caption{Области $\left\{\xi< L_1\right\}$ и $\left\{\xi > L_1\right\}$.}
    \label{fig:pt10:_lineDomains_simple}
\end{figure}
При этом показано, что эти множества являются непересекающимися в смысле бильярдных траекторий: если траектория пересекает $EF$ хотя бы однажды, тогда точки, соответствующие фрагментам $T_k$ такой траектории, находятся только на части прямой $L$, попадающей внутрь области $\{ \xi < L_1\}$, изображенной на рис. \ref{fig:pt10:_lineDomains_simple}.
Эти области рассматриваются отдельно в соответствующих разделах главы.


Для области $\{\xi < L_1\}$, соответствующей <<ветвящемуся интегралу>>, получена бифуркационная диаграмма (см. рис. \ref{fig:pt10:_B1_lattice_straight}), на которой фрагмент $L \cap \{\xi < L_1\}$ выглядит как горизонтальная прямая. 
\begin{figure}[!htb]
\centering
\includegraphics[width=10cm]{images/ch4/section3_circular/B1_lattice_straight.pdf}
    \caption{}
    \label{fig:pt10:_B1_lattice_straight}
\end{figure}
При этом для любой фиксированной тройки параметров $(r_1, n_1, n_2)$ точки $P(\Xi_m)$ и $P(\Xi_{m+1})$, соответствующие двум последовательным фрагментам бильярдной траектории $T_m$ и $T_{m+1}$, соединяются вектором $(1, 0)$. 
Структура слоения определяется тем, как прямая $L \cap \{\xi < L_1\}$ пересекает области, обозначенные $C_k, k \geq 1$ на рис. \ref{fig:pt10:_B1_lattice_straight}. 
При этом в склейке регулярных поверхностей участвуют все $C_m$ одного и того же индекса $m$, лежащие на одной горизонтали. Нерегулярным поверхностям соответствуют точки пересечения прямой $L \cap \{\xi < L_1\}$ с прямыми одного из трех семейств, отмеченных римскими цифрами.

Приводится классификация заметаемых областей для фрагментов бильярдных траекторий $T_m$ в зависимости от расположения соответствующей точки $P(\Xi_m)$ на диаграмме (см. рис. \ref{fig:pt10:_B1Prime_lattice}).
\begin{figure}[!htb]
\centering
\includegraphics[width=7cm]{images/ch4/section3_circular/B1Prime_lattice.pdf}
    \caption{Прямая, соответствующая прямой $\rho_1^2 = r_2^2$.}
    \label{fig:pt10:_B1Prime_lattice}
\end{figure}
Для неособых поверхностей доказаны следующие теоремы:
\textit{При $n_1^2<n_2^2$}
\begin{theorem}
В области $\left\{\xi < L_1\right\}$ поверхности $S_\Xi$ являются сферами с $2m$ ручками, если $P(\Xi) \in C_m$ и сферами с $2m-1$ ручками, если $P(\Xi) \in C_m'$ и $m \leq \left\lfloor \frac{r_2^2}{r_2^2-r_1^2} \right\rfloor$. 
\label{th:pt10:th1}
\end{theorem}
\textit{При $n_1^2 > n_2^2$}
\begin{theorem}
В области $\left\{\xi < L_1\right\}$ поверхности $S_\Xi$ являются сферами с $2m+1$ ручками, если $P(\Xi) \in C_m$ и сферами с $2m$ ручками, если $P(\Xi) \in C_m'$ и $m \leq \left\lfloor \frac{r_2^2}{r_2^2-r_1^2} \right\rfloor$. 
\label{th:pt10:th2}
\end{theorem}

Случай $\{\xi > L_1\}$ существенно проще предыдущего, поскольку в этом случае дополнительный интеграл $\Xi$ однозначен: бильярдные траектории лежат на поверхностях уровня дополнительного интеграла $\Xi=\const$.
Структура слоения определяется тем, как прямая $L \cap  \{\xi > L_1\}$ пересекает четыре области, изображенные на рис. \ref{fig:pt10:_xiL1_subdivision}.
\begin{figure}[!htb]
\centering
\includegraphics[width=6cm]{images/ch4/section3_circular/sect3_xiL1_subdivision.pdf}
    \caption{Разбиение области $\left\{\xi > L_1\right\}$.}
    \label{fig:pt10:_xiL1_subdivision}
\end{figure}
Доказано следующее:
\begin{theorem}
Поверхность уровня $\Xi = \const$ является сферой с двумя ручками для случаев
\begin{itemize}[beginpenalty=10000]
\item $r_2^2 > \rho_1^2 > \rho_2^2 > r_1^2$
\item $r_2^2 > \rho_2^2 > \rho_1^2 > r_1^2$
\end{itemize}
И тором для случаев 
\begin{itemize}[beginpenalty=10000]
\item $\rho_2^2 > r_2^2 > \rho_1^2 > r_1^2$
\item $\rho_1^2 > r_2^2 > \rho_2^2 > r_1^2$
\end{itemize}
\end{theorem}

В завершение главы приводятся описания особых поверхностей для случаев $\{\xi < L_1\}$ и $\{\xi > L_1\}$ с иллюстрациями для ключевых перестроек.

\FloatBarrier
\pdfbookmark{Заключение}{conclusion}                                  % Закладка pdf
В \underline{\textbf{заключении}} перечислены основные результаты диссертации, а также предложены
возможные направления дальнейших исследований в рамках тематики работы.

{\gratitude}
Автор выражает глубокую благодарность своему научному руководителю Ф.Ю.Попеленскому за постановку задачи, постоянную поддержку и внимание к работе на всех этапах её подготовки. 
Автор благодарит коллектив кафедры дифференциальной геометрии и приложений механико-математического факультета МГУ за вдохновляющую, доброжелательную атмосферу и поддержку.

%{\contents}
%{\gratitude}

