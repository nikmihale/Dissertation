\chapter*{Заключение}                       % Заголовок
\addcontentsline{toc}{chapter}{Заключение}  % Добавляем его в оглавление

%% Согласно ГОСТ Р 7.0.11-2011:
%% 5.3.3 В заключении диссертации излагают итоги выполненного исследования, рекомендации, перспективы дальнейшей разработки темы.
%% 9.2.3 В заключении автореферата диссертации излагают итоги данного исследования, рекомендации и перспективы дальнейшей разработки темы.
%% Поэтому имеет смысл сделать эту часть общей и загрузить из одного файла в автореферат и в диссертацию:
В работе были получены новые важные сведения о классических и квантовых бильярдах на софокусных столах.

Во главе 2 получены точные решения стационарного уравнения Шрёдингера в областях $A_\delta$ и $B_\delta$, а также в накрытии эллиптического кольца. Предполагается что фокусы граничных квадрик всех рассматриваемых областей расположены в точках $(\pm \delta,0)$. Вычислена асимптотика собственных значений для  близких к нулю значений $\delta$ с точностью до $\delta^2$ включительно. 
Любопытно заметить, что предложенный в диссертации метод опирается на алгоритм, который естественным образом может быть продолжен по степеням $\delta$, то есть можно получить коэффициенты разложения для последующих степеней $\delta$. При этом коэффициенты разложения  получаются аналитическими методами. 

В главе 3 показано, что подчиняющейся косинусному закону преломления классический бильярд на софокусном столе является интегрируемой системой. 
Получена явная формула дополнительного интеграла $\Xi$. Стоит отметить что в зависимости от параметров квадрик, на которых преломляется траектория, множеством значений интеграла $\Xi$ может оказаться не только отрезок, но и окружность. 

В главе 4 приведено описание поверхностей постоянного уровня интеграла $\Xi$ для бильярда в эллипсе $\Omega = \Omega_1 \cup \Omega_2$ с преломлением согласно косинусному закону на софокусном эллипсе (задача А).
Приведена методика построения по всей видимости новых бифуркационных  диаграмм, которые учитывают одновременно все возможные значения оптических параметров областей. 
Получены ранее не встречавшиеся в теории динамических систем особые поверхности, соответствующие одновременным разным бифуркациям в фрагментах бильярдного стола $\Omega_i$. 

Аналогичная задача решена в главе 5 на другом софокусном столе $\Omega = \Omega_1 \cup \Omega_2$ (задача Б). 
Сложность задачи обусловлена тем, что бильярдная траектория может неоднократно посещать каждый фрагмент $\Omega_i$ с разными параметрами касательной квадрики. 
В работе это обстоятельство преодолено: найден однозначный интеграл $\Xi$, для которого построена бифуркационная диаграмма, с помощью которой описаны поверхности уровня этого интеграла $\Xi$ для регулярных и критических значений.

В качестве направлений дальнейших исследований можно выделить следующие.
  \begin{itemize}[beginpenalty=10000] % https://tex.stackexchange.com/a/476052/104425
  \item Построение аналогичных асимптотик уровней энергии квантового бильярда в присутствии потенциала.
  \item Изучение более широкого класса областей. Применение изложенных в диссертации методов позволяет вычислить асимптотику собственных значений для квантового бильярда в <<эллиптическом прямоугольнике>> (см. рис. \ref{fig:conclusion_quantum_domains}). Существуют иные <<эллиптические секторы>>, которые при сближении фокусов также стремятся к круговому сектору, но для анализа задач в них требуется продвинутое понимание распределения нулей  угловых функций Матьё.
    \begin{figure}[ht]
    \centerfloat{
        \hfill
        \subcaptionbox{<<эллиптический сектор>>}{%
\includegraphics[width=3.5cm]{future/future_1.pdf}}
        \hfill
        \subcaptionbox{<<эллиптический прямоугольник>>}{%
\includegraphics[width=2.5cm]{future/future_2.pdf}}
        \hfill
    }
    \caption{Софокусные столы для перспективных квантовых задач.}\label{fig:conclusion_quantum_domains}
\end{figure}  
  \item Решение задач А и Б для большего количества областей разбиения бильярдного стола. 
  \item Изучение бильярдных книжек с косинусным законом преломления. При этом преломление может происходить не только на корешках, но и внутри листов. 
  \item Вычисление инвариантов Фоменко-Цишанга для задач А и Б.
  \end{itemize}

%\input{common/concl}
