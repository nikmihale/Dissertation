\chapter{Постоянная наблюдаемая величина}\label{app:A}

Как было показано в  \cite{wref13}, наблюдаемая  $(\hat{L_z})^2=\hbar^2\frac{\partial^2}{\partial \phi^2}$ сохраняется для квантового биллиарда в круговом секторе. 
В частности, решению $\psi(r,\phi) = J_{\lambda_p}(\frac{\alpha_{\lambda_{p,n}}}{r_0}r) \sin \lambda_p \phi$ 
уравнения Шрёдингера в секторе $0\leq r \leq r_0, 0 \leq \phi \leq \theta_s$ соответствует величина $$\langle \psi, (\hat{L_z})^2\psi\rangle = \hbar^2 \lambda_p^2.$$

В наших условиях интегрируемость классического биллиарда следует из сохраняемой величины, а именно, произведение угловых моментов относительно фокусов.


Определим два оператора  $\hat L_{\pm c} = (x \pm c)\frac{\partial}{\partial y} - y\frac{\partial}{\partial x}$ угловых моментов относительно фокусов  $(x,y) = (\mp c, 0)$.


Рассмотрим оператор  $\hat A = \frac{-1}{2}(L_c L_{-c} + L_{-c} L_c + c^2 \nabla ^2)$. Пусть функция $\psi(\phi, \rho) = \Phi(\zeta, q, \phi) R(\zeta, q, \rho)$ является собственной функцией для оператора $\nabla^2$. %, причем $<\psi, \psi>=1$.
%Положим $\Phi_1(a, q, \phi), \Phi_2(a, q, \phi)$ -- независимые решения уравнения $\frac{\partial^2}{\partial \phi^2}\Phi + (a - 2q\cos{2\phi})\Phi = 0$, $R_1(a, q, \rho), R_2(a, q, \rho)$ -- независимые решения уравнения $\frac{\partial^2}{\partial \rho^2}R - (a - 2q\cosh{2\rho})R = 0$. 

Тогда
$\hat A \phi = \zeta \phi,$
в частности,
\begin{equation}
\langle  \psi, \hat A \psi\rangle = \zeta \langle\psi, \psi\rangle. \label{eq:cons}
\end{equation}


$\hat A$ в эллиптических координатах можно записать в виде 
\begin{align*}
\hat A & = \frac{2}{\cos 2\phi - \cosh 2\rho}(\sinh{\rho}^2 \dfrac{\partial^2}{\partial \phi^2}-{}\notag\\ 
&\qquad{}-\sin\phi^2\dfrac{\partial^2}{\partial \rho^2})  -\frac{c^2}{2} \nabla^2 =\notag\\
&= \frac{\cos 2\phi \dfrac{\partial^2}{\partial \rho^2} + \cosh 2\rho \dfrac{\partial^2}{\partial \phi^2}}{\cos 2 \phi - \cosh 2 \rho}.
\end{align*}
Угловая  $\Phi(\zeta, q, \phi)$ и радиальная   $R(\zeta, q, \rho)$ составляющие являются решениями соответствующих функций Матьё с параметрами  $\zeta, q$.
Таким образом,
$\frac{\partial^2}{\partial \phi ^2} \Phi(\phi) = -(\zeta-2q\cos\phi)\Phi(\phi)$ и $\frac{\partial^2}{\partial \rho ^2} R(\rho) = (\zeta-2q\cosh\rho)R(\rho)$. 
Поэтому,
\begin{align*}
A \psi &= R(\rho)\Phi(\phi)\times {}\notag\\
&{}\times\frac{\zeta(\cos 2\phi -\cosh2\rho) + (2q-2q)\cos2\phi\cosh2\rho}{\cos 2 \phi - \cosh 2 \rho} = {}\notag \\
&{}=\zeta \psi.
\end{align*}


\chapter{Очень длинное название второго приложения, в~котором продемонстрирована работа с~длинными таблицами}\label{app:B}

\section{Подраздел приложения}\label{app:B1}
Вот размещается длинная таблица:
\makeatletter
\@ifpackagelater{longtable}{2024/07/04} % hotfix of bug fixed in https://github.com/latex3/latex2e/commit/7c96b6b90a730278903e71a482d88479789a89a3
{% Если много longtable* используется и новый latex, то три строки ниже правильней в главную преамбулу унести
\renewenvironment{longtable*}
  {\renewcommand\LTcaptype{}\longtable}
  {\endlongtable}
}
{\@ifpackagelater{longtable}{2024/04/26}
    {\addtocounter{table}{-1}}
    {}
}
\makeatother
\fontsize{10pt}{10pt}\selectfont
\begin{longtable*}[c]{|l|c|l|l|} %longtable* появляется из пакета ltcaption и даёт ненумерованную таблицу
    \hline
    Параметр & Умолч. & Тип & Описание               \\ \hline
    \endfirsthead   \hline
    \multicolumn{4}{|c|}{\small\slshape (продолжение)}        \\ \hline
    Параметр & Умолч. & Тип & Описание               \\ \hline
    \endhead        \hline
    \multicolumn{4}{|r|}{\small\slshape продолжение следует}  \\ \hline
    \endfoot        \hline
    \endlastfoot
    \multicolumn{4}{|l|}{\&INP}        \\ \hline
    kick & 1 & int & 0: инициализация без шума (\(p_s = const\)) \\
    &   &     & 1: генерация белого шума                  \\
    &   &     & 2: генерация белого шума симметрично относительно \\
    & & & экватора    \\
    mars & 0 & int & 1: инициализация модели для планеты Марс     \\
    kick & 1 & int & 0: инициализация без шума (\(p_s = const\)) \\
    &   &     & 1: генерация белого шума                  \\
    &   &     & 2: генерация белого шума симметрично относительно \\
    & & & экватора    \\
    mars & 0 & int & 1: инициализация модели для планеты Марс     \\
    kick & 1 & int & 0: инициализация без шума (\(p_s = const\)) \\
    &   &     & 1: генерация белого шума                  \\
    &   &     & 2: генерация белого шума симметрично относительно \\
    & & & экватора    \\
    mars & 0 & int & 1: инициализация модели для планеты Марс     \\
    kick & 1 & int & 0: инициализация без шума (\(p_s = const\)) \\
    &   &     & 1: генерация белого шума                  \\
    &   &     & 2: генерация белого шума симметрично относительно \\
    & & & экватора    \\
    mars & 0 & int & 1: инициализация модели для планеты Марс     \\
    kick & 1 & int & 0: инициализация без шума (\(p_s = const\)) \\
    &   &     & 1: генерация белого шума                  \\
    &   &     & 2: генерация белого шума симметрично относительно \\
    & & & экватора    \\
    mars & 0 & int & 1: инициализация модели для планеты Марс     \\
    kick & 1 & int & 0: инициализация без шума (\(p_s = const\)) \\
    &   &     & 1: генерация белого шума                  \\
    &   &     & 2: генерация белого шума симметрично относительно \\
    & & & экватора    \\
    mars & 0 & int & 1: инициализация модели для планеты Марс     \\
    kick & 1 & int & 0: инициализация без шума (\(p_s = const\)) \\
    &   &     & 1: генерация белого шума                  \\
    &   &     & 2: генерация белого шума симметрично относительно \\
    & & & экватора    \\
    mars & 0 & int & 1: инициализация модели для планеты Марс     \\
    kick & 1 & int & 0: инициализация без шума (\(p_s = const\)) \\
    &   &     & 1: генерация белого шума                  \\
    &   &     & 2: генерация белого шума симметрично относительно \\
    & & & экватора    \\
    mars & 0 & int & 1: инициализация модели для планеты Марс     \\
    kick & 1 & int & 0: инициализация без шума (\(p_s = const\)) \\
    &   &     & 1: генерация белого шума                  \\
    &   &     & 2: генерация белого шума симметрично относительно \\
    & & & экватора    \\
    mars & 0 & int & 1: инициализация модели для планеты Марс     \\
    kick & 1 & int & 0: инициализация без шума (\(p_s = const\)) \\
    &   &     & 1: генерация белого шума                  \\
    &   &     & 2: генерация белого шума симметрично относительно \\
    & & & экватора    \\
    mars & 0 & int & 1: инициализация модели для планеты Марс     \\
    kick & 1 & int & 0: инициализация без шума (\(p_s = const\)) \\
    &   &     & 1: генерация белого шума                  \\
    &   &     & 2: генерация белого шума симметрично относительно \\
    & & & экватора    \\
    mars & 0 & int & 1: инициализация модели для планеты Марс     \\
    kick & 1 & int & 0: инициализация без шума (\(p_s = const\)) \\
    &   &     & 1: генерация белого шума                  \\
    &   &     & 2: генерация белого шума симметрично относительно \\
    & & & экватора    \\
    mars & 0 & int & 1: инициализация модели для планеты Марс     \\
    kick & 1 & int & 0: инициализация без шума (\(p_s = const\)) \\
    &   &     & 1: генерация белого шума                  \\
    &   &     & 2: генерация белого шума симметрично относительно \\
    & & & экватора    \\
    mars & 0 & int & 1: инициализация модели для планеты Марс     \\
    kick & 1 & int & 0: инициализация без шума (\(p_s = const\)) \\
    &   &     & 1: генерация белого шума                  \\
    &   &     & 2: генерация белого шума симметрично относительно \\
    & & & экватора    \\
    mars & 0 & int & 1: инициализация модели для планеты Марс     \\
    kick & 1 & int & 0: инициализация без шума (\(p_s = const\)) \\
    &   &     & 1: генерация белого шума                  \\
    &   &     & 2: генерация белого шума симметрично относительно \\
    & & & экватора    \\
    mars & 0 & int & 1: инициализация модели для планеты Марс     \\
    \hline
    \multicolumn{4}{|l|}{\&SURFPAR}        \\ \hline
    kick & 1 & int & 0: инициализация без шума (\(p_s = const\)) \\
    &   &     & 1: генерация белого шума                  \\
    &   &     & 2: генерация белого шума симметрично относительно \\
    & & & экватора    \\
    mars & 0 & int & 1: инициализация модели для планеты Марс     \\
    kick & 1 & int & 0: инициализация без шума (\(p_s = const\)) \\
    &   &     & 1: генерация белого шума                  \\
    &   &     & 2: генерация белого шума симметрично относительно \\
    & & & экватора    \\
    mars & 0 & int & 1: инициализация модели для планеты Марс     \\
    kick & 1 & int & 0: инициализация без шума (\(p_s = const\)) \\
    &   &     & 1: генерация белого шума                  \\
    &   &     & 2: генерация белого шума симметрично относительно \\
    & & & экватора    \\
    mars & 0 & int & 1: инициализация модели для планеты Марс     \\
    kick & 1 & int & 0: инициализация без шума (\(p_s = const\)) \\
    &   &     & 1: генерация белого шума                  \\
    &   &     & 2: генерация белого шума симметрично относительно \\
    & & & экватора    \\
    mars & 0 & int & 1: инициализация модели для планеты Марс     \\
    kick & 1 & int & 0: инициализация без шума (\(p_s = const\)) \\
    &   &     & 1: генерация белого шума                  \\
    &   &     & 2: генерация белого шума симметрично относительно \\
    & & & экватора    \\
    mars & 0 & int & 1: инициализация модели для планеты Марс     \\
    kick & 1 & int & 0: инициализация без шума (\(p_s = const\)) \\
    &   &     & 1: генерация белого шума                  \\
    &   &     & 2: генерация белого шума симметрично относительно \\
    & & & экватора    \\
    mars & 0 & int & 1: инициализация модели для планеты Марс     \\
    kick & 1 & int & 0: инициализация без шума (\(p_s = const\)) \\
    &   &     & 1: генерация белого шума                  \\
    &   &     & 2: генерация белого шума симметрично относительно \\
    & & & экватора    \\
    mars & 0 & int & 1: инициализация модели для планеты Марс     \\
    kick & 1 & int & 0: инициализация без шума (\(p_s = const\)) \\
    &   &     & 1: генерация белого шума                  \\
    &   &     & 2: генерация белого шума симметрично относительно \\
    & & & экватора    \\
    mars & 0 & int & 1: инициализация модели для планеты Марс     \\
    kick & 1 & int & 0: инициализация без шума (\(p_s = const\)) \\
    &   &     & 1: генерация белого шума                  \\
    &   &     & 2: генерация белого шума симметрично относительно \\
    & & & экватора    \\
    mars & 0 & int & 1: инициализация модели для планеты Марс     \\
    \hline
\end{longtable*}

\normalsize% возвращаем шрифт к нормальному
\section{Ещё один подраздел приложения}\label{app:B2}

Нужно больше подразделов приложения!
Конвынёры витюпырата но нам, тебиквюэ мэнтётюм позтюлант ед про. Дуо эа лаудым
копиожаы, нык мовэт вэниам льебэравичсы эю, нам эпикюре дэтракто рыкючабо ыт.

Пример длинной таблицы с записью продолжения по ГОСТ 2.105:

\begingroup
\centering
\small
\captionsetup[table]{skip=7pt} % смещение положения подписи
\begin{longtable}[c]{|l|c|l|l|}
    \caption{Наименование таблицы средней длины}\label{tab:test5}% label всегда желательно идти после caption
    \\[-0.45\onelineskip]
    \hline
    Параметр & Умолч. & Тип & Описание                                          \\ \hline
    \endfirsthead%
    \caption*{Продолжение таблицы~\thetable}                                    \\[-0.45\onelineskip]
    \hline
    Параметр & Умолч. & Тип & Описание                                          \\ \hline
    \endhead
    \hline
    \endfoot
    \hline
    \endlastfoot
    \multicolumn{4}{|l|}{\&INP}                                                 \\ \hline
    kick     & 1      & int & 0: инициализация без шума (\(p_s = const\))       \\
             &        &     & 1: генерация белого шума                          \\
             &        &     & 2: генерация белого шума симметрично относительно \\
             &        &     & экватора                                          \\
    mars     & 0      & int & 1: инициализация модели для планеты Марс          \\
    kick     & 1      & int & 0: инициализация без шума (\(p_s = const\))       \\
             &        &     & 1: генерация белого шума                          \\
             &        &     & 2: генерация белого шума симметрично относительно \\
             &        &     & экватора                                          \\
    mars     & 0      & int & 1: инициализация модели для планеты Марс          \\
    kick     & 1      & int & 0: инициализация без шума (\(p_s = const\))       \\
             &        &     & 1: генерация белого шума                          \\
             &        &     & 2: генерация белого шума симметрично относительно \\
             &        &     & экватора                                          \\
    mars     & 0      & int & 1: инициализация модели для планеты Марс          \\
    kick     & 1      & int & 0: инициализация без шума (\(p_s = const\))       \\
             &        &     & 1: генерация белого шума                          \\
             &        &     & 2: генерация белого шума симметрично относительно \\
             &        &     & экватора                                          \\
    mars     & 0      & int & 1: инициализация модели для планеты Марс          \\
    kick     & 1      & int & 0: инициализация без шума (\(p_s = const\))       \\
             &        &     & 1: генерация белого шума                          \\
             &        &     & 2: генерация белого шума симметрично относительно \\
             &        &     & экватора                                          \\
    mars     & 0      & int & 1: инициализация модели для планеты Марс          \\
    kick     & 1      & int & 0: инициализация без шума (\(p_s = const\))       \\
             &        &     & 1: генерация белого шума                          \\
             &        &     & 2: генерация белого шума симметрично относительно \\
             &        &     & экватора                                          \\
    mars     & 0      & int & 1: инициализация модели для планеты Марс          \\
    kick     & 1      & int & 0: инициализация без шума (\(p_s = const\))       \\
             &        &     & 1: генерация белого шума                          \\
             &        &     & 2: генерация белого шума симметрично относительно \\
             &        &     & экватора                                          \\
    mars     & 0      & int & 1: инициализация модели для планеты Марс          \\
    kick     & 1      & int & 0: инициализация без шума (\(p_s = const\))       \\
             &        &     & 1: генерация белого шума                          \\
             &        &     & 2: генерация белого шума симметрично относительно \\
             &        &     & экватора                                          \\
    mars     & 0      & int & 1: инициализация модели для планеты Марс          \\
    kick     & 1      & int & 0: инициализация без шума (\(p_s = const\))       \\
             &        &     & 1: генерация белого шума                          \\
             &        &     & 2: генерация белого шума симметрично относительно \\
             &        &     & экватора                                          \\
    mars     & 0      & int & 1: инициализация модели для планеты Марс          \\
    kick     & 1      & int & 0: инициализация без шума (\(p_s = const\))       \\
             &        &     & 1: генерация белого шума                          \\
             &        &     & 2: генерация белого шума симметрично относительно \\
             &        &     & экватора                                          \\
    mars     & 0      & int & 1: инициализация модели для планеты Марс          \\
    kick     & 1      & int & 0: инициализация без шума (\(p_s = const\))       \\
             &        &     & 1: генерация белого шума                          \\
             &        &     & 2: генерация белого шума симметрично относительно \\
             &        &     & экватора                                          \\
    mars     & 0      & int & 1: инициализация модели для планеты Марс          \\
    kick     & 1      & int & 0: инициализация без шума (\(p_s = const\))       \\
             &        &     & 1: генерация белого шума                          \\
             &        &     & 2: генерация белого шума симметрично относительно \\
             &        &     & экватора                                          \\
    mars     & 0      & int & 1: инициализация модели для планеты Марс          \\
    kick     & 1      & int & 0: инициализация без шума (\(p_s = const\))       \\
             &        &     & 1: генерация белого шума                          \\
             &        &     & 2: генерация белого шума симметрично относительно \\
             &        &     & экватора                                          \\
    mars     & 0      & int & 1: инициализация модели для планеты Марс          \\
    kick     & 1      & int & 0: инициализация без шума (\(p_s = const\))       \\
             &        &     & 1: генерация белого шума                          \\
             &        &     & 2: генерация белого шума симметрично относительно \\
             &        &     & экватора                                          \\
    mars     & 0      & int & 1: инициализация модели для планеты Марс          \\
    kick     & 1      & int & 0: инициализация без шума (\(p_s = const\))       \\
             &        &     & 1: генерация белого шума                          \\
             &        &     & 2: генерация белого шума симметрично относительно \\
             &        &     & экватора                                          \\
    mars     & 0      & int & 1: инициализация модели для планеты Марс          \\
    \hline
    \multicolumn{4}{|l|}{\&SURFPAR}                                             \\ \hline
    kick     & 1      & int & 0: инициализация без шума (\(p_s = const\))       \\
             &        &     & 1: генерация белого шума                          \\
             &        &     & 2: генерация белого шума симметрично относительно \\
             &        &     & экватора                                          \\
    mars     & 0      & int & 1: инициализация модели для планеты Марс          \\
    kick     & 1      & int & 0: инициализация без шума (\(p_s = const\))       \\
             &        &     & 1: генерация белого шума                          \\
             &        &     & 2: генерация белого шума симметрично относительно \\
             &        &     & экватора                                          \\
    mars     & 0      & int & 1: инициализация модели для планеты Марс          \\
    kick     & 1      & int & 0: инициализация без шума (\(p_s = const\))       \\
             &        &     & 1: генерация белого шума                          \\
             &        &     & 2: генерация белого шума симметрично относительно \\
             &        &     & экватора                                          \\
    mars     & 0      & int & 1: инициализация модели для планеты Марс          \\
    kick     & 1      & int & 0: инициализация без шума (\(p_s = const\))       \\
             &        &     & 1: генерация белого шума                          \\
             &        &     & 2: генерация белого шума симметрично относительно \\
             &        &     & экватора                                          \\
    mars     & 0      & int & 1: инициализация модели для планеты Марс          \\
    kick     & 1      & int & 0: инициализация без шума (\(p_s = const\))       \\
             &        &     & 1: генерация белого шума                          \\
             &        &     & 2: генерация белого шума симметрично относительно \\
             &        &     & экватора                                          \\
    mars     & 0      & int & 1: инициализация модели для планеты Марс          \\
    kick     & 1      & int & 0: инициализация без шума (\(p_s = const\))       \\
             &        &     & 1: генерация белого шума                          \\
             &        &     & 2: генерация белого шума симметрично относительно \\
             &        &     & экватора                                          \\
    mars     & 0      & int & 1: инициализация модели для планеты Марс          \\
    kick     & 1      & int & 0: инициализация без шума (\(p_s = const\))       \\
             &        &     & 1: генерация белого шума                          \\
             &        &     & 2: генерация белого шума симметрично относительно \\
             &        &     & экватора                                          \\
    mars     & 0      & int & 1: инициализация модели для планеты Марс          \\
    kick     & 1      & int & 0: инициализация без шума (\(p_s = const\))       \\
             &        &     & 1: генерация белого шума                          \\
             &        &     & 2: генерация белого шума симметрично относительно \\
             &        &     & экватора                                          \\
    mars     & 0      & int & 1: инициализация модели для планеты Марс          \\
    kick     & 1      & int & 0: инициализация без шума (\(p_s = const\))       \\
             &        &     & 1: генерация белого шума                          \\
             &        &     & 2: генерация белого шума симметрично относительно \\
             &        &     & экватора                                          \\
    mars     & 0      & int & 1: инициализация модели для планеты Марс          \\
\end{longtable}
\normalsize% возвращаем шрифт к нормальному
\endgroup
\section{Использование длинных таблиц с окружением \textit{longtblr} из~пакета \texttt{tabularray}}\label{app:B2a}

В таблице \cref{tab:test-functions} более книжный вариант длинной таблицы,
используя окружение \verb!longtblr! из~пакета \verb!tabularray! и разнообразные
разделители (\verb!toprule!, \verb!midrule!, \verb!bottomrule!) из~пакета
\verb!booktabs!.

Чтобы визуально таблица смотрелась лучше, можно использовать следующие
параметры.
Таблица задаётся на всю ширину, \verb!longtblr! позволяет делить ширину колонок
пропорционально "--- тут три колонки в~пропорции 1.1:1.1:4 "--- для каждой
колонки первый параметр в~описании \verb!X[]!.
Кроме того, в~таблице убраны отступы слева и справа с~помощью \verb!@{}!
в~преамбуле таблицы.
К~первому и~второму столбцу применяется модификатор

\verb!>{\setlength{\baselineskip}{0.7\baselineskip}}!,

\noindent который уменьшает межстрочный интервал для текста таблиц (иначе
заголовок второго столбца значительно шире, а двухстрочное имя
сливается с~окружающими). Для первой и второй колонки текст в ячейках
выравниваются по~центру как по~вертикали, так и по горизонтали "---
задаётся буквами \verb!m!~и~\verb!c!~в~описании столбца \verb!X[]!.

Так как формулы большие "--- используется окружение \verb!alignedat!,
чтобы отступ был одинаковый у всех формул "--- он сделан для всех, хотя
для большей части можно было и не использовать.  Чтобы формулы
занимали поменьше места в~каждом столбце формулы (где надо)
используется \verb!\textstyle! "--- он~делает дроби меньше, у~знаков
суммы и произведения "--- индексы сбоку. Иногда формула слишком большая,
сливается со следующей, поэтому после неё ставится небольшой
дополнительный отступ \verb!\vspace*{2ex}!. Для штрафных функций "---
размер фигурных скобок задан вручную \verb!\Big\{!, т.\:к. не~умеет
\verb!alignedat! работать с~\verb!\left! и~\verb!\right! через
несколько строк/колонок.

В примечании к таблице наоборот, окружение \verb!cases! даёт слишком
большие промежутки между вариантами, чтобы их уменьшить, в конце
каждой строчки окружения использовался отрицательный дополнительный
отступ \verb!\\[-0.5em]!.

\DefTblrTemplate{contfoot-text}{default}{\small\slshape продолжение следует} % переделали default шаблон, используемый по-умолчанию
\DefTblrTemplate{conthead-text}{default}{\small\slshape (продолжение)} % переделали normal default, используемый по-умолчанию
\DefTblrTemplate{capcont}{default}{\centering\UseTblrTemplate{conthead-text}{default}\par} % для работы центрирования обязательно \par
\DefTblrTemplate{caplast}{default}{\small\slshape (окончание)}
\DefTblrTemplate{lasthead}{default}{\centering\UseTblrTemplate{caplast}{default}\par} % для работы центрирования обязательно \par
\DefTblrTemplate{firsthead}{default}{% правим шаблон у первого заголовка, чтобы считывал настройки из пакета caption
    % https://tex.stackexchange.com/a/628973
    \addtocounter{table}{-1}%
    \IfTokenListEmpty{\InsertTblrText{entry}}{% важно, чтобы не дублировались записи в списке таблиц
        \captionof{table}{\InsertTblrText{caption}}%
    }{%
        \captionof{table}[\InsertTblrText{entry}]{\InsertTblrText{caption}}%
    }% если будет запись в entry, то она пойдет в список таблиц, см. документацию tabularray
}
\SetTblrTemplate{caption-lot}{empty} % важно, чтобы не дублировались записи в списке таблиц
\begin{longtblr}[
    caption = {Тестовые функции для оптимизации, \(D\) "---  размерность. Для всех функций значение в точке глобального минимума равно нулю.},
    label = {tab:test-functions},
    ]{
    colspec = {%
    @{}>{\setlength{\baselineskip}{0.7\baselineskip}}X[1.1,m,c]%
    >{\setlength{\baselineskip}{0.7\baselineskip}}X[1.1,m,c]%
    X[4,l]@{}%
    },
    width = \textwidth,
    rowhead = 1,
    rows={rowsep=3pt},
    row{1}={rowsep=2pt},
    }
    \toprule     %%% верхняя линейка
    Имя                      & Стартовый диапазон параметров   & Функция                                 \\
    \midrule %%% тонкий разделитель. Отделяет названия столбцов. Обязателен по ГОСТ 2.105 пункт 4.4.5
    сфера                    & \(\left[-100,\,100\right]^D\)   &
    \(\begin{aligned}
          \textstyle f_1(x)=\sum_{i=1}^Dx_i^2
      \end{aligned}\)                                                                 \\
    Schwefel 2.22            & \(\left[-10,\,10\right]^D\)     &
    \(\begin{aligned}
          \textstyle f_2(x)=\sum_{i=1}^D|x_i|+\prod_{i=1}^D|x_i|
      \end{aligned}\)                                              \\
    Schwefel 1.2             & \(\left[-100,\,100\right]^D\)   &
    \(\begin{aligned}
          \textstyle f_3(x)=\sum_{i=1}^D\left(\sum_{j=1}^ix_j\right)^2
      \end{aligned}\)                            \\
    Schwefel 2.21            & \(\left[-100,\,100\right]^D\)   &
    \(\begin{aligned}
          \textstyle f_4(x)=\max_i\!\left\{\left|x_i\right|\right\}
      \end{aligned}\)                                           \\
    Rosenbrock               & \(\left[-30,\,30\right]^D\)     &
    \(\begin{aligned}
          \textstyle f_5(x)=
          \sum_{i=1}^{D-1}
          \left[100\!\left(x_{i+1}-x_i^2\right)^2+(x_i-1)^2\right]
      \end{aligned}\)                      \\
    ступенчатая              & \(\left[-100,\,100\right]^D\)   &
    \(\begin{aligned}
          \textstyle f_6(x)=\sum_{i=1}^D\big\lfloor x_i+0.5\big\rfloor^2
      \end{aligned}\)                                      \\
    зашумлённая квартическая & \(\left[-1.28,\,1.28\right]^D\) &
    \(\begin{aligned}
          \textstyle f_7(x)=\sum_{i=1}^Dix_i^4+rand[0,1)
      \end{aligned}\)\vspace*{2ex}                                                      \\
    Schwefel 2.26            & \(\left[-500,\,500\right]^D\)   &
    \(\begin{aligned}
          f_8(x)= & \textstyle\sum_{i=1}^D-x_i\,\sin\sqrt{|x_i|}\,+ \\
                  & \vphantom{\sum}+ D\cdot
          418.98288727243369
      \end{aligned}\)                                          \\
    Rastrigin                & \(\left[-5.12,\,5.12\right]^D\) &
    \(\begin{aligned}
          \textstyle f_9(x)=\sum_{i=1}^D\left[x_i^2-10\,\cos(2\pi x_i)+10\right]
      \end{aligned}\)\vspace*{2ex}                              \\
    Ackley                   & \(\left[-32,\,32\right]^D\)     &
    \(\begin{aligned}
          f_{10}(x)= & \textstyle -20\, \exp\!\left(
          -0.2\sqrt{\frac{1}{D}\sum_{i=1}^Dx_i^2} \right)- \\
                     & \textstyle - \exp\left(
              \frac{1}{D}\sum_{i=1}^D\cos(2\pi x_i)  \right)
          + 20 + e
      \end{aligned}\)                               \\
    Griewank                 & \(\left[-600,\,600\right]^D\)   &
    \(\begin{aligned}
          f_{11}(x)= & \textstyle \frac{1}{4000}\sum_{i=1}^{D}x_i^2 -
          \prod_{i=1}^D\cos\left(x_i/\sqrt{i}\right) +1
      \end{aligned}\) \vspace*{3ex}                                        \\
    штрафная 1               & \(\left[-50,\,50\right]^D\)     &
    \(\begin{aligned}
          f_{12}(x)= & \textstyle \frac{\pi}{D}\Big\{ 10\,\sin^2(\pi y_1) +            \\
                     & +\textstyle \sum_{i=1}^{D-1}(y_i-1)^2
          \left[1+10\,\sin^2(\pi y_{i+1})\right] +                                     \\
                     & +(y_D-1)^2 \Big\} +\textstyle\sum_{i=1}^D u(x_i,\,10,\,100,\,4)
      \end{aligned}\) \vspace*{2ex} \\
    штрафная 2               & \(\left[-50,\,50\right]^D\)     &
    \(\begin{aligned}
          f_{13}(x)= & \textstyle 0.1 \Big\{\sin^2(3\pi x_1) +            \\
                     & +\textstyle \sum_{i=1}^{D-1}(x_i-1)^2
          \left[1+\sin^2(3 \pi x_{i+1})\right] +                          \\
                     & +(x_D-1)^2\left[1+\sin^2(2\pi x_D)\right] \Big\} + \\
                     & +\textstyle\sum_{i=1}^D u(x_i,\,5,\,100,\,4)
      \end{aligned}\)              \\
    сфера                    & \(\left[-100,\,100\right]^D\)   &
    \(\begin{aligned}
          \textstyle f_1(x)=\sum_{i=1}^Dx_i^2
      \end{aligned}\)                                                                 \\
    Schwefel 2.22            & \(\left[-10,\,10\right]^D\)     &
    \(\begin{aligned}
          \textstyle f_2(x)=\sum_{i=1}^D|x_i|+\prod_{i=1}^D|x_i|
      \end{aligned}\)                                              \\
    Schwefel 1.2             & \(\left[-100,\,100\right]^D\)   &
    \(\begin{aligned}
          \textstyle f_3(x)=\sum_{i=1}^D\left(\sum_{j=1}^ix_j\right)^2
      \end{aligned}\)                            \\
    Schwefel 2.21            & \(\left[-100,\,100\right]^D\)   &
    \(\begin{aligned}
          \textstyle f_4(x)=\max_i\!\left\{\left|x_i\right|\right\}
      \end{aligned}\)                                           \\
    Rosenbrock               & \(\left[-30,\,30\right]^D\)     &
    \(\begin{aligned}
          \textstyle f_5(x)=
          \sum_{i=1}^{D-1}
          \left[100\!\left(x_{i+1}-x_i^2\right)^2+(x_i-1)^2\right]
      \end{aligned}\)                      \\
    ступенчатая              & \(\left[-100,\,100\right]^D\)   &
    \(\begin{aligned}
          \textstyle f_6(x)=\sum_{i=1}^D\big\lfloor x_i+0.5\big\rfloor^2
      \end{aligned}\)                                      \\
    зашумлённая квартическая & \(\left[-1.28,\,1.28\right]^D\) &
    \(\begin{aligned}
          \textstyle f_7(x)=\sum_{i=1}^Dix_i^4+rand[0,1)
      \end{aligned}\)\vspace*{2ex}                                                      \\
    Schwefel 2.26            & \(\left[-500,\,500\right]^D\)   &
    \(\begin{aligned}
          f_8(x)= & \textstyle\sum_{i=1}^D-x_i\,\sin\sqrt{|x_i|}\,+ \\
                  & \vphantom{\sum}+ D\cdot
          418.98288727243369
      \end{aligned}\)                                          \\
    Rastrigin                & \(\left[-5.12,\,5.12\right]^D\) &
    \(\begin{aligned}
          \textstyle f_9(x)=\sum_{i=1}^D\left[x_i^2-10\,\cos(2\pi x_i)+10\right]
      \end{aligned}\)\vspace*{2ex}                              \\
    Ackley                   & \(\left[-32,\,32\right]^D\)     &
    \(\begin{aligned}
          f_{10}(x)= & \textstyle -20\, \exp\!\left(
          -0.2\sqrt{\frac{1}{D}\sum_{i=1}^Dx_i^2} \right)- \\
                     & \textstyle - \exp\left(
              \frac{1}{D}\sum_{i=1}^D\cos(2\pi x_i)  \right)
          + 20 + e
      \end{aligned}\)                               \\
    Griewank                 & \(\left[-600,\,600\right]^D\)   &
    \(\begin{aligned}
          f_{11}(x)= & \textstyle \frac{1}{4000}\sum_{i=1}^{D}x_i^2 -
          \prod_{i=1}^D\cos\left(x_i/\sqrt{i}\right) +1
      \end{aligned}\) \vspace*{3ex}                                        \\
    штрафная 1               & \(\left[-50,\,50\right]^D\)     &
    \(\begin{aligned}
          f_{12}(x)= & \textstyle \frac{\pi}{D}\Big\{ 10\,\sin^2(\pi y_1) +            \\
                     & +\textstyle \sum_{i=1}^{D-1}(y_i-1)^2
          \left[1+10\,\sin^2(\pi y_{i+1})\right] +                                     \\
                     & +(y_D-1)^2 \Big\} +\textstyle\sum_{i=1}^D u(x_i,\,10,\,100,\,4)
      \end{aligned}\) \vspace*{2ex} \\
    штрафная 2               & \(\left[-50,\,50\right]^D\)     &
    \(\begin{aligned}
          f_{13}(x)= & \textstyle 0.1 \Big\{\sin^2(3\pi x_1) +            \\
                     & +\textstyle \sum_{i=1}^{D-1}(x_i-1)^2
          \left[1+\sin^2(3 \pi x_{i+1})\right] +                          \\
                     & +(x_D-1)^2\left[1+\sin^2(2\pi x_D)\right] \Big\} + \\
                     & +\textstyle\sum_{i=1}^D u(x_i,\,5,\,100,\,4)
      \end{aligned}\)              \\
    \midrule%%% тонкий разделитель
    \SetCell[c=3]{l,\linewidth}%
    \hspace*{2.5em}% абзацный отступ - требование ГОСТ 2.105
    Примечание "---  Для функций \(f_{12}\) и \(f_{13}\)
    используется \(y_i = 1 + \frac{1}{4}(x_i+1)\)
    и~\(u(x_i,\,a,\,k,\,m)=
    \begin{cases*}
        k(x_i-a)^m,  & \( x_i >a \)            \\[-0.5em]
        0,           & \( -a\leq x_i \leq a \) \\[-0.5em]
        k(-x_i-a)^m, & \( x_i <-a \)
    \end{cases*}
    \)                                                                                                   \\
    \bottomrule %%% нижняя линейка
\end{longtblr}
