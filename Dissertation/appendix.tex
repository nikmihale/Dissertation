\chapter{Постоянная наблюдаемая величина}\label{app:A}

Как было показано в  \cite{wref13}, наблюдаемая  $(\hat{L_z})^2=\hbar^2\frac{\partial^2}{\partial \phi^2}$ сохраняется для квантового биллиарда в круговом секторе. 
В частности, решению $\psi(r,\phi) = J_{\lambda_p}(\frac{\alpha_{\lambda_{p,n}}}{r_0}r) \sin \lambda_p \phi$ 
уравнения Шрёдингера в секторе $0\leq r \leq r_0, 0 \leq \phi \leq \theta_s$ соответствует величина $$\langle \psi, (\hat{L_z})^2\psi\rangle = \hbar^2 \lambda_p^2.$$

В наших условиях интегрируемость классического биллиарда следует из сохраняемой величины, а именно, произведение угловых моментов относительно фокусов.


Определим два оператора  $\hat L_{\pm c} = (x \pm c)\frac{\partial}{\partial y} - y\frac{\partial}{\partial x}$ угловых моментов относительно фокусов  $(x,y) = (\mp c, 0)$.


Рассмотрим оператор  $\hat A = \frac{-1}{2}(L_c L_{-c} + L_{-c} L_c + c^2 \nabla ^2)$. Пусть функция $\psi(\phi, \rho) = \Phi(\zeta, q, \phi) R(\zeta, q, \rho)$ является собственной функцией для оператора $\nabla^2$. %, причем $<\psi, \psi>=1$.
%Положим $\Phi_1(a, q, \phi), \Phi_2(a, q, \phi)$ -- независимые решения уравнения $\frac{\partial^2}{\partial \phi^2}\Phi + (a - 2q\cos{2\phi})\Phi = 0$, $R_1(a, q, \rho), R_2(a, q, \rho)$ -- независимые решения уравнения $\frac{\partial^2}{\partial \rho^2}R - (a - 2q\cosh{2\rho})R = 0$. 

Тогда
$\hat A \phi = \zeta \phi,$
в частности,
\begin{equation}
\langle  \psi, \hat A \psi\rangle = \zeta \langle\psi, \psi\rangle. \label{eq:cons}
\end{equation}


$\hat A$ в эллиптических координатах можно записать в виде 
\begin{align*}
\hat A & = \frac{2}{\cos 2\phi - \cosh 2\rho}(\sinh{\rho}^2 \dfrac{\partial^2}{\partial \phi^2}-{}\notag\\ 
&\qquad{}-\sin\phi^2\dfrac{\partial^2}{\partial \rho^2})  -\frac{c^2}{2} \nabla^2 =\notag\\
&= \frac{\cos 2\phi \dfrac{\partial^2}{\partial \rho^2} + \cosh 2\rho \dfrac{\partial^2}{\partial \phi^2}}{\cos 2 \phi - \cosh 2 \rho}.
\end{align*}
Угловая  $\Phi(\zeta, q, \phi)$ и радиальная   $R(\zeta, q, \rho)$ составляющие являются решениями соответствующих функций Матьё с параметрами  $\zeta, q$.
Таким образом,
$\frac{\partial^2}{\partial \phi ^2} \Phi(\phi) = -(\zeta-2q\cos\phi)\Phi(\phi)$ и $\frac{\partial^2}{\partial \rho ^2} R(\rho) = (\zeta-2q\cosh\rho)R(\rho)$. 
Поэтому,
\begin{align*}
A \psi &= R(\rho)\Phi(\phi)\times {}\notag\\
&{}\times\frac{\zeta(\cos 2\phi -\cosh2\rho) + (2q-2q)\cos2\phi\cosh2\rho}{\cos 2 \phi - \cosh 2 \rho} = {}\notag \\
&{}=\zeta \psi.
\end{align*}

