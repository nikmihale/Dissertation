\chapter{Самосопряженные расширения оператора Лапласа на ветвящихся многообразиях.}\label{ch:ch5}

\section{Введение.}\label{sec:ch5/sect1}
Будем называть $\Gamma$ ветвящимся многообразием, если $\Gamma$ определяется как объединение $N$ областей $\Gamma_j, j \in \{1, 2, \ldots, N\}$: $\Gamma = \cup_{j=1}^N \Gamma_j$. Предполагается, что для каждого $j$ область $\Gamma_j$ является $d_j$--мерной ограниченной областью в пространстве $\mathbb{R}^{d_j}$ с гладкой границей $\eta_j$ размерности $d_j-1$. Граница многообразия $\Gamma$ определяется как объединение $N$ границ составляющих областей $\partial \Gamma = \eta = \cup_{j=1}^N\eta_j$, где $\eta_j = \partial \Gamma_j$. 

Здесь и далее будем использовать запись 
$\left. u \right|_\eta = 
\begin{pmatrix}
\left. u_1 \right|_{\eta_1} \\
\ldots \\
\left. u_N \right|_{\eta_N}
\end{pmatrix}
$ для функции $u$ на границе $\eta$ и 

$\left. \frac{\partial u}{\partial n} \right|_\eta = 
\begin{pmatrix}
\left. \frac{\partial u_1}{\partial n_1} \right|_{\eta_1} \\
\ldots \\
\left. \frac{\partial u_N}{\partial n_N} \right|_{\eta_N}
\end{pmatrix}
$ для значения нормальной производной функции $u$, где $\left. \frac{\partial u_j}{\partial n_j} \right|_{\eta_j}$ -- производная в направлении внешней относительно $\Gamma_j$ нормали $n_j$ к границе $\eta_j$.


\section{Условия самосопряженности оператора Лапласа на ветвящихся многообразиях.}\label{sec:ch5/sect2}
Введем Гильбертово пространство $$h = L_2(\Gamma) = \bigoplus L_2(\Gamma_j)$$.
Положим пространство граничных значений $G = h^\frac{3}{2} \oplus h^\frac{1}{2}$, где 
$ h^\frac{3}{2} = \oplus_{j=1}^N W_2^\frac{3}{2}(\eta_j)$ и $ h^\frac{1}{2} = \oplus_{j=1}^N W_2^\frac{1}{2}(\eta_j)$.
Граничное значение $\left. u \right|_\eta$ функции $u \in W_2^2(\Gamma)$ является элементом пространства $h^\frac{3}{2}$ \cite{Jakovlev_1967}, в то же время граничное значение нормальной производной $\left. \frac{\partial u}{\partial n} \right|_\eta$ принадлежит пространству $h^\frac{1}{2}$. 

\begin{theorem}
Пусть $A$ -- линейный оператор на пространстве $h$ с плотной областью определения $h^\frac{3}{2}$ с множеством значений в линейном многообразии $h^\frac{1}{2}$. Пусть $D_A$ -- линейное многообразие функций $u \in W_2^2(\Gamma)$ и значения функции $u$ на границе $\eta$ связаны с значениями нормальной производной $u$ соотношением 
$$\left. \frac{\partial u}{\partial n} \right|_\eta = A \left. u \right|_\eta.$$
Тогда для самосопряженности оператора $\mathbf{L}_A = \left. ( \mathbf{L}_0^* ) \right|_{D_A}$ необходимо и достаточно выполнение равенства $A=A^*$.
\end{theorem}
\begin{proof}
Пусть $ u \in D(\mathbf{L}_A)$ и $ v \in D(\mathbf{L}_0^*)$, тогда справедливо равенство
$$\left( \mathbf{L}_A u , \ v \right)_h - \left(  u , \  \mathbf{L}_0^* v \right)_h  = 
\left( 
\left. u \right|_\eta , \ A^* \left. v \right|_\eta - \left. \frac{\partial v}{\partial n} \right|_\eta 
\right).$$
В силу произвольности  $\left. u \right|_\eta$, равенство $\left. \frac{\partial v}{\partial n} \right|_\eta = A^* \left. v \right|_\eta$ является необходимым и достаточным условием для включения $v \in D(\mathbf{L}_A^*)$. Поскольку область определения оператора $\mathbf{L}^A$ задана уравнением $\left. \frac{\partial u}{\partial n} \right|_\eta = A \left. u \right|_\eta$, тогда самосопряженность оператора $\mathbf{L}_A = \mathbf{L}_A^*$ равносильна тождеству $A=A^*$.
\end{proof}

\begin{theorem}
Пусть область определения оператора $\mathbf{L}$ задается соотношением $Z \left. \frac{\partial u}{\partial n} \right|_\eta + A \left. u \right|_\eta = 0$. Тогда оператор $\mathbf{L}$ самосопряжен тогда и только тогда, когда $A \ Z^* =Z \ A^*$.
\end{theorem} 
\begin{proof}
Запишем решения системы $Z \left. \frac{\partial u}{\partial n} \right|_\eta + A \left. u \right|_\eta = 0$ в следующем виде:
\begin{equation*}
\begin{pmatrix}
\left. u \right|_\eta \\
\left. \frac{\partial u}{\partial n} \right|_\eta 
\end{pmatrix}_{2n \times 1} = \Phi_{2n \times n} \xi_{n\times 1},
\end{equation*}
где $\Phi$ -- фундаментальная матрица системы, $\xi$ -- элемент пространства граничных значений $G$. 
Из граничных значений получим условие $\left( A \ Z \right) \Phi_{2n \times n} = 0_{n \times n}$ или, что эквивалентно, 
\begin{equation}
\Phi_{n \times 2n}^T \begin{pmatrix}
A^T \\
Z^T
\end{pmatrix}_{2n \times n} = 0_{n \times n}.
\label{eq:ch5:boundaryCond}
\end{equation}
Для элементов $u \in D(\mathbf{L})$ и $v \in D(\mathbf{L}_0^*)$ выполняется равенство
\begin{multline}
\left( L u, \ v \right) - \left( x, \ (\mathbf{L}_0^*) v \right) = 
\left( \left. \frac{\partial u}{\partial n} \right|_\eta ,\ \left. v \right|_\eta \right) - 
\left( \left. u \right|_\eta, \ \left. \frac{\partial v}{\partial n} \right|_\eta \right) =  \\ 
\left(
\begin{pmatrix}
	\left. u \right|_\eta \\
	\left. \frac{\partial u}{\partial n} \right|_\eta
\end{pmatrix}, \ 
\begin{pmatrix}
	0	& -1 \\
	1	& 0
\end{pmatrix}
\begin{pmatrix}
	\left. v \right|_\eta \\
	\left. \frac{\partial v}{\partial n} \right|_\eta
\end{pmatrix} 
\right) = 
\left( \Phi_{2n\times n} \xi_{n \times 1} , \ 
\begin{pmatrix}
	0	& -1 \\
	1	& 0
\end{pmatrix}
\begin{pmatrix}
	\left. v \right|_\eta \\
	\left. \frac{\partial v}{\partial n} \right|_\eta
\end{pmatrix} 
\right) = \\
\left( \xi, \ \Phi_{n\times 2n}^T 
\begin{pmatrix}
	0	& -1 \\
	1	& 0
\end{pmatrix}
\begin{pmatrix}
	\left. v \right|_\eta \\
	\left. \frac{\partial v}{\partial n} \right|_\eta
\end{pmatrix} 
\right)
\label{eq:ch5:branchingAZ}
\end{multline}

Поскольку $v \in D(\mathbf{L}_0^*)$, должно выполняться равенство 
\begin{equation}
\left( \mathbf{L} u, \ v \right) - \left(u, \   \mathbf{L}_0^* v \right) = 0,
\label{eq:ch5:adjointV}
\end{equation}
справедливое в том числе для базисных элементов $v$ из $D(\mathbf{L}_0^*)$.
Положим $V = 
\begin{pmatrix}
	\left. v_1 \right|_\eta					   	& \ldots	& \left. v_N \right|_\eta	\\
	\left. \frac{\partial v_1}{\partial n} \right|_\eta.	& \ldots	& \left. \frac{\partial v_N}{\partial n} \right|_\eta
\end{pmatrix}$ -- матрица, в которой по столбцам записаны базисные элементы линейного $G$, тогда 
из \eqref{eq:ch5:branchingAZ}, \eqref{eq:ch5:adjointV} получаем
$$\Phi^T \begin{pmatrix}
			0	&	-1	\\
			1	&	0
		\end{pmatrix}
	V_{2n \times n} = 0_{n \times n}.$$
В силу равенства \eqref{eq:ch5:boundaryCond} заключаем, что матрицу $\mathbf{V}$ можно выбрать в виде
$$V = \begin{pmatrix}
-Z^* \\
A^*
\end{pmatrix}.$$
\textcolor{red}{мне что-то сомневательно от доказательства, но я не знаю что меня смущает в нем}
\end{proof}

\section{Спектр оператора Лапласа на ветвящихся многообразиях.}\label{sec:ch5/sect3}
Как описать в общем случае -- я не знаю. Только в общих чертах если уравнение расщепляется, то могу повертеть руками и пару слов сказать, но в более сложных ситуациях спектр непонятно как описать. Рассмотрим примеры начиная с трилистника из прямоугольников.
\subsection{Пример: трилистник из прямоугольников}\label{sec:ch5/sect3/subsect1}
Положим $\Gamma = \Gamma_1 \cup \Gamma_2 \cup \Gamma_3$ -- ветвящееся многообразие, образованное прямоугольниками $\Gamma_j = [0, x_j] \times [0, y_0]$, соединенными по ребру $\{x=0\} \times [0, y_0]$.

На каждом $\Gamma_j$ уравнение на собственные функции оператора Лапласа $\frac{\partial^2 u}{\partial x^2} + \frac{\partial^2 u}{\partial y^2} + \varkappa^2 u=0$ расщепляется в декартовых координатах $(x, y)$, а граница $\eta_j$ представляет собой объединение координатных линий $\{x=x_j\} \cup \{y=y_0\} \cup \{x=0\} \cup \{y=0\}$. Следовательно, для $u(x,y) = \xi(x) \zeta(y)$ поиск обращающихся в нуль на $\eta = \partial \Gamma = \cup_j \left\{ \eta_j \setminus \{x=0\} \right\}$ собственных функций сводится к решению системы дифференциальных уравнений 
\begin{equation}
\left\{
\begin{array}{lll}
\dfrac{\partial^2 \xi_j}{\partial x^2}  + \nu^2 \xi_j = 0 , & \xi_j(x_j) = 0, & j=1,2,3; 	\\
\dfrac{\partial^2 \zeta_j}{\partial y^2}  + \mu^2 \zeta_j = 0 , & \zeta_j(y_0) = \zeta_j(0) = 0, & j=1,2,3,
\end{array}
\right.
\label{eq:rectangularSystem}
\end{equation}
где $\xi_j(x) = \left. \xi(x) \right|_{\Gamma_j}$, $\zeta_j(x) = \left. \zeta(x) \right|_{\Gamma_j}$, $\nu^2 + \mu^2 = \varkappa^2$.
На общем сегменте границы $\{x=0\}$ потребуем, чтобы выполнялось условие вида  $Z \left. \frac{\partial u}{\partial n} \right|_\eta + A \left. u \right|_\eta = 0$, а именно потребуем выполнения \textit{условий Кирхгофа}:
\begin{equation}
\begin{pmatrix}
0	&	0	&	0	\\
0	&	0	&	0	\\
1	&	1	&	1
\end{pmatrix} 
\begin{pmatrix}
-\xi_1' (0)	\zeta_1(y) \\
-\xi_2' (0)	\zeta_2(y) \\
-\xi_3' (0)	\zeta_3(y) 
\end{pmatrix} + 
\begin{pmatrix}
1	&	-1	&	0	\\
0	&	1	&	-1	\\
0	&	0	&	0
\end{pmatrix} 
\begin{pmatrix}
\xi_1(0)	\zeta_1(y) \\
\xi_2(0)	\zeta_2(y) \\
\xi_3(0)	\zeta_3(y) 
\end{pmatrix} = 
\begin{pmatrix}
0	\\	0	\\	0
\end{pmatrix} 
\label{eq:rectanglesKirchhoffCondition}
\end{equation}

Решениями системы уравнений без учета условия \eqref{eq:rectanglesKirchhoffCondition} являются функции $\xi_j(x) = \alpha_j \sin \nu (x-x_j) \zeta_j(y) = \beta_j \sin \mu y$, где $\mu = \dfrac{\pi m}{y_0}, m \in \mathbb{N}$. 
Заметим, что функции $\zeta_j(y)$ отличаются друг от друга только постоянным множителем $\beta_j$, который мы можем учесть в $\alpha_j$. 
Также поскольку $\zeta(y)$ не является тождественно нулевым решением, условие \eqref{eq:rectanglesKirchhoffCondition} может быть записано без общего множителя $\zeta(y)$. 
Таким образом, задача поиска $\xi(x)$ сводится к задаче на графе с одной точкой ветвления \eqref{sec:ch4/sect3}. Как следует из примера \eqref{sec:ch4/sect3/subsect1}, собственные значения $\nu^2$ являются нулями функции $F(\nu) = \nu \left( \sin{\nu x_1} \sin{\nu x_2} \cos{\nu x_3} +  \sin{\nu x_1} \cos{\nu x_2} \sin{\nu x_3} +  \cos{\nu x_1} \sin{\nu x_2} \sin{\nu x_3} \right)$

Таким образом, мы доказали следующее
\begin{statement}
Собственными значениями оператора Лапласа на ветвящемся многообразии $\Gamma$ являются $\varkappa^2 = \mu^2 + \nu^2$, где $\mu = \frac{\pi m}{y_0}, m \in \mathbb{N}$, а $\nu$ --- $n$-ый нуль функции 
\begin{equation*}
F(\nu) = \nu \left( \sin{\nu x_1} \sin{\nu x_2} \cos{\nu x_3} + \sin{\nu x_1} \cos{\nu x_2} \sin{\nu x_3} +  \cos{\nu x_1} \sin{\nu x_2} \sin{\nu x_3} \right).
\end{equation*}
Более того, если $x_1=x_2=x_3=x_0$, выражение для $\varkappa^2$ записывается в явном виде как
$$\varkappa^2 = \frac{\pi^2 m^2}{y_0^2} + \frac{\pi^2 n^2}{4x_0^2}, \  m, n \in \mathbb{N}.$$
\end{statement}

\subsection{Пример для частного случая вложения трилистника из прямоугольников}\label{sec:ch5/sect3/subsect2}
Рассмотрим задачу из предыдущего примера для конкретного вложения  $\Gamma$ в $\mathbb{R}^3$ и потребуем выполнение условия компланарности векторов $\left. \frac{\partial u}{\partial n_j} \right|_\eta, \ j = 1,2,3$ на множестве точек ветвления.

По аналогии с примером \eqref{sec:ch4/sect3/subsect2} предположим, что  $\Gamma$ (определение см. \eqref{sec:ch5/sect3/subsect1}) вложен в $\mathbb{R}^3$ таким образом, что 
$$
\angle( \Gamma_1, \Gamma_2) = \alpha, \quad
\angle( \Gamma_2, \Gamma_3) = \beta, \quad
\angle( \Gamma_3, \Gamma_1) = \gamma,
$$
где под $\angle( \Gamma_i, \Gamma_j)$ понимается двугранный угол между соответствующими многообразиями.

Повторяя соображения \eqref{sec:ch5/sect3/subsect1} получим, что искомые решения $u(x,y)$ уравнения $\Delta u + \varkappa^2 u=0$ на $\Gamma$ получаются решением системы \eqref{eq:rectangularSystem}. При этом двугранные углы между $\Gamma_i$ и $\Gamma_j$ будут иметь ту же величину, что и плоский угол, образованный векторами внешних нормалей $n_i$ и $n_j$. Таким образом 
 уравнение системы для $\xi(x)$ с учетом двугранных углов повторяет условие одномерной задачи из примера \eqref{sec:ch4/sect3/subsect2} с теми же величинами углов $\alpha, \beta, \gamma$. 
 
Следовательно, для решения уравнения $\Delta \xi + \nu^2 \xi = 0$ справедливо утверждение \ref{stat:1dimAnglesSpectrum}.
Повторяя соображения из предыдущего примера для $\Delta \zeta + \mu^2 \zeta = 0$, получим следующее:
\begin{statement}
Собственными значениями $\varkappa^2$ сформулированной задачи получаются как
$\varkappa^2 = \nu^2 + \mu^2$, где $\mu =  \frac{\pi m}{y_0}$, а $\nu$ является $n$--ым нулем функции
\begin{multline*}
F(\nu)=\nu \left(
\sin{\alpha} \sin{\nu x_1} \sin{\nu x_2} \cos{\nu x_3} + 
\sin{\gamma} \sin{\nu x_1} \cos{\nu x_2} \sin{\nu x_3} 
\right. \\ \left. + 
\sin{\beta} \cos{\nu x_1} \sin{\nu x_2} \sin{\nu x_3} 
 \right),
 \end{multline*}
\end{statement}

В случае $x_1=x_2=x_3=x_0$ имеем $$F(\nu) = \nu \cos \nu x_0 \sin^2(\nu x_0) \left(\sin \alpha + \sin \beta + \sin \gamma \right),$$ тогда $\varkappa^2 = \frac{\pi^2 m^2}{y_0^2} + \frac{\pi^2 n^2}{4x_0^2}, \ m,n \in \mathbb{N}$.

\clearpage
