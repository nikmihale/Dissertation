\chapter{Предварительные сведения}\label{ch:ch1}

\section{Квантовый бильярд}\label{sec:ch1/sec1}
Пусть $\Omega$ --- область в $\mathbb{R}^2(x,y)$. Рассмотрим стационарное уравнение Шрёдингера:
\begin{equation*}
\hat{H}\psi = \left(\frac{\hat{p}^2}{2M} + V(x,y)\right) \psi = \left( \frac{-\hbar^2}{2M}\nabla^2 + V(x, y)\right) \psi = E\psi,
\label{eq:schrodinger}
\end{equation*}
где потенциал $V(x,y)$ внутри области $\Omega$ равен нулю, а вне ее обращается в бесконечность. 

\begin{remark}
Такая задача равносильна поиску собственных функций и собственных значений оператора Лапласа в области $\Omega$ для функций, обращающихся в нуль на границе $\Omega$. 
\label{rem:schrodinger_equivalent}
\end{remark}

Заметим, что при обозначении $\varkappa^2 = \frac{2 M E}{\hbar^2}$ уравнение Шрёдингера можно записать в виде $\nabla^2  = -\varkappa^2 \psi$.

\subsection{Стационарное уравнение Шрёдингера в полярной системе координат}\label{sec:ch1/sec1/sub1}
В полярной системе координат $(x, y) = (r\cos \phi, r\sin \phi)$ оператор Гамильтона \eqref{eq:schrodinger} внутри области $\Omega$ имеет вид:
\begin{equation*}
\hat{H} = \frac{-\hbar^2}{2M} \left(\frac{1}{r^2}\frac{\partial^2}{\partial \phi^2} + \frac{1}{r}\frac{\partial}{\partial r} + \frac{\partial^2}{\partial r^2}\right) 
%\label{eq:polarSchrodinger}
\end{equation*}
Запишем функцию $\psi$ в виде $\psi(r, \phi) = R(r)\Phi(\phi)$, и пусть $E$ -- ее собственное значение. Тогда положим $\varkappa^2 = \frac{2M E}{\hbar^2}$, и уравнение $(\nabla^2  + \varkappa^2) \psi=0$ примет вид 
$$ \frac{R}{r^2}\frac{\partial^2\Phi}{\partial\phi^2} + \frac{\Phi}{r}\frac{\partial R}{\partial r} + \Phi  +\varkappa^2R\Phi = 0.$$
Обе части уравнения умножим на $\frac{r^2}{R\Phi}$:
$$ \frac{1}{\Phi}\frac{\partial^2\Phi}{\partial\phi^2} + \frac{r}{R}\frac{\partial R}{\partial r} + \frac{r^2}{R}\frac{\partial^2 R}{\partial r^2} + \varkappa^2r^2 = 0.$$
Введем разделяющий параметр $\lambda$, что позволяет рассмотреть исходное уравнение на $\psi(r, \phi) = R(r)\Phi(\phi)$, как систему из двух обыкновенных дифференциальных уравнений:
\begin{equation}
\left\{\begin{array}{rcll}
    \dfrac{\Phi''}{\Phi} &=&-\lambda^2, \\
    \dfrac{rR'}{R} + \dfrac{r^2R''}{R} + \varkappa^2r^2 &=& \lambda^2  \quad 	 &\textit{   (уравнение Бесселя)}. \\
\end{array}
\right.
\label{eq:Bessel_equations}
\end{equation}
решениями последнего уравнения являются функции Бесселя первого и второго рода $J_\lambda(\varkappa r)$ и $Y_\lambda(\varkappa r)$, соответственно. 


\subsection{Функции Бесселя}\label{sec:ch1/sec1/sub2}
Функция Бесселя первого рода $J_\alpha(x)$ является решением дифференциального уравнения второго порядка
$$x^2\frac{d^2y}{dx^2} + x\frac{dy}{dx} + (x^2 - \alpha^2)y=0,$$
где параметр $\alpha$ выбирается так, что разложение $J_\alpha(x)$ в точке $x=0$ имеет вид
$$J_\alpha(x) = \sum_{r=0}^\infty \frac{(-1)^r (\frac{x}{2})^{\alpha + 2r}}{r! \Gamma(\alpha+r+1)}.$$
Главная ветвь $J_\alpha(x)$ соответствует главному значению функции $(\frac{x}{2})^{\alpha + 2r}$ и является аналитической функцией в комплексной плоскости с вырезом вдоль интервала $(-\infty, 0]$. 
При $\alpha \in \mathbb{Z}$ функция $J_\alpha(x)$ является целой для $x \in \mathbb{C}$, причем $J_{-n}(x) = (-1)^nJ_n(x)$. Если $\alpha \notin \mathbb{Z}$, то функции $J_\alpha$ и $J_{-\alpha}$ линейно независимы.
Если $\alpha \in \mathbb{R}$, то функции Бесселя $J_\alpha(x)$ имеют счетное число положительных вещественных нулей $\{ j_{\alpha, k} \}_{k=1}^\infty$. Традиционно их нумеруют по возрастанию 
$$j_{\alpha, 1} < j_{\alpha, 2} < j_{\alpha, 3} < ...$$
Для функций Бесселя справедливы связующие соотношения:
$$Y_\alpha(x) = \frac{J_\alpha(x) \cos \pi \alpha - J_{-\alpha}(x)}{\sin \pi \alpha}$$ и рекуррентные соотношения, справедливые для $F=J$ и $F=Y$.
$$ \frac{2\alpha}{x}F_\alpha(x) = F_{\alpha-1}(x)+F_{\alpha+1}(x), \quad 
2\frac{d F_\alpha(x)}{d x} = F_{\alpha-1}(x)-F_{\alpha+1}(x).$$ 
Также существуют (см. \cite[\S~10.19]{wref5} асимптотики функций Бесселя для больших аргументов и для больших значений порядка $\alpha$:
$$J_\alpha(x) \approx \sqrt{2 \over \pi x}\left( \cos(x - \frac{\alpha \pi}{2} - \frac{\pi}{4} \right), \quad J_\alpha(x) \approx \sqrt{1 \over 2 \pi \alpha}\left(e x  \over 2 \alpha\right)^\alpha.$$

Для нулей $j_{\alpha, m}, \quad y_{\alpha, m}$ функций Бесселя первого и второго рода выведена~\cite{wref3} их асимптотика при $m \to \infty$:
$$j_{\alpha, m}, y_{\alpha, m} \approx a - {\mu - 1 \over 8 a} - {4(\mu - 1)(7\mu - 31) \over 3(8a)^3} - {32(\mu - 1)(83\mu^2 - 982\mu + 3779 \over 15(8a)^5}+\dots,$$
где $\mu = 4 \alpha^2$, $a=(m+{1\over2}\alpha - {1\over4})\pi$ для $j_{\alpha, m}$ или $a=(m+{1\over2}\alpha - {3\over4})\pi$ для $y_{\alpha, m}$.
Для произведений функций Бесселя вида $Y_\nu(x) J_\nu(\lambda x) - Y_\nu(\lambda x) J_\nu(x)$ асимптотика $m$-го положительного нуля $\alpha_{\nu, m}$ также известна  \cite[\S\ 9, с.~358]{wref2}:
 $$\alpha_{\nu, m} = \sigma + \frac{\chi}{\sigma} + \frac{\omega-\chi^2}{\sigma^3} + \frac{\eta-4\chi \omega +2\chi^3}{\sigma^5} + \dots,$$
 где $\mu=4\nu^2$,
$\sigma=\frac{\pi m}{\lambda - 1}$,
 $\chi = \frac{\mu - 1}{8 \lambda}$,
 $\omega = \frac{(\mu-1)(\mu-25)(\lambda^3 - 1)}{6(4\lambda)^3(\lambda-1)}$,
 $\eta = \frac{(\mu-1)(\mu^2-114\mu+1073)(\lambda^5-1)}{5(4\lambda)^5(\lambda-1)}$.


\section{Квантовый биллиард в области $\Omega$, ограниченной координатными линиями полярной системы координат}\label{sec:ch1/sec2}

\subsection{Квантовый биллиард в круге}\label{sec:ch1/sec2/sub1}
Пусть $\Omega = D^2$ --- двумерный диск радиуса $\rho_0$. Для этой системы гамильтониан имеет выглядит как в уравнении \eqref{eq:schrodinger}, где 
% в декартовых координатах записывается как указано в уравнении 
%$$\hat{H} = \frac{\hat{p}^2}{2M} + V(x, y) = \frac{-\hbar^2}{2M}\left(\frac{\partial^2 }{\partial x^2} + \frac{\partial^2}{\partial y^2}\right)  + V(x, y) = \frac{-\hbar^2}{2M} \nabla^2  + V(x, y), $$
потенциал $V(x,y)$ определяется как
\[
    V(x, y) = 
    \Bigg\{
    \begin{array}{cc}
        0, \sqrt{x^2+y^2} < \rho_0 \\
        \infty, \sqrt{x^2+y^2} \geq \rho_0. \\
    \end{array}
\] 
В силу замечания \ref{rem:schrodinger_equivalent}, для решения задачи достаточно найти такие собственные функции и собственные значения оператора Лапласа в области $\Omega$, которые обращаются в нуль на границе $\Omega$.
\begin{statement}
В области $D^2$ собственные функции $\psi_{k, m}(r, \phi)$ и собственные значения $E_{k,m}$ оператора $\hat{H}$ имеют вид 
$$\psi_{k, m}(r, \phi) = J_k\left(\frac{\alpha_{k, m}r}{\rho_0}\right)\sin{k \phi}, \hspace{15pt} E_{k,m} = \frac{\hbar^2 \alpha_{k, m}^2}{2M\rho_0^2}, \hspace{15pt} k, m \in \mathbb{Z},$$
где $\alpha_{k, m}$ - $m$-ый ноль функции Бесселя первого рода $J_k(x)$.
\label{st:sec1_stat1}
\end{statement}
\begin{remark}
Вместо $\sin$ можно использовать $\cos$.
\end{remark}
\begin{proof}
Для $r < \rho_0$ получаем дифференциальное уравнение второго порядка в частных производных $(\nabla^2 + \varkappa^2)\psi = 0$, где $\varkappa^2 = \frac{2M E}{\hbar^2}$. Запишем искомую функцию в виде $\psi(r, \phi) = R(r)\Phi(\phi)$. Тогда, в силу соображений из подраздела \ref{sec:ch1/sec1/sub1}, $\Phi(\phi) = A\cos{k\phi} + B\sin{k\phi}$ и $R(r) = J_k(\varkappa r)$, где $J_k(x)$ - функция Бесселя первого рода. Таким образом, $\psi(r, \phi) = J_k(\varkappa r) (\widetilde{A}\cos{k\phi} + \widetilde{B}\sin{k\phi} )$.

Из граничного условия заметим, что $$\psi(r, \phi) |_{\partial D^2} = 0 \implies J_k(\varkappa r)|_{r=\rho_0} = 0 \implies \varkappa \rho_0 \in \{\alpha: J_k(\alpha) = 0\},$$
откуда допустимые значения $\varkappa \in \{ \frac{\alpha}{\rho_0} : J_k(\alpha)=0 \}$. Значения $k$ должны быть такими, чтоы выполнялось равенство $\Phi(\phi) = \Phi(\phi+2\pi)$, справедливое при условии $k \in \mathbb{Z}$. Учитывая также, что $\varkappa^2 = \frac{2M E}{\hbar^2}$, допустимыми значениями $E$ являются
$$E = \frac{\hbar^2\varkappa^2}{2M} \in \left\{ \frac{\hbar^2\alpha^2}{2M\rho_0^2}: J_k(\alpha)=0 \right\}, k \in \mathbb{Z}.$$

Тогда, выбирая из множества $\alpha$ один конкретный $\alpha_{k, m}$ для некоторого $m \in \mathbb{N}$, получим вид собственных функций системы (с точностью до умножения на константу) 
$$\psi_{k, m}(r, \phi) = J_k\left(\frac{\alpha_{k, m}r}{\rho_0}\right) \sin{k\phi},$$
и их собственными значениями будут
$$E_{k, m} = \frac{\hbar^2\alpha_{k, m}^2}{2M\rho_0^2}, \hspace{15pt} k, m \in \mathbb{Z},$$
где $\alpha_{k, m}$ - $m$-ый ноль функции $J_k(x)$.
\end{proof}

\subsection{Квантовый биллиард в круговом кольце и накрытии кругового кольца}\label{sec:ch1/sec2/sub2}
Пусть  $\Omega$ --- область, $p$-листно накрывающая кольцо, 
ограниченное двумя концентрическими окружностями радиусов $0 < r_0 < r_1$. Случай $p=1$ относится к классической теории колебаний (см. \cite{wref11}). Будем считать, что обе окружности имеют центр в начале координат.
В  области $\Omega$ удобно рассматривать аналог полярных координат --- расстояние $r$ до начала координат и угол $\phi$, определенный  $\mod 2\pi p$. 
В $\Omega$ рассмотрим стационарное уравнение Шрёдингера \eqref{eq:schrodinger},
%$$\hat{H}\psi = \left(\frac{-\hbar^2}{2M} \nabla^2  + V(r)\right)\psi = E\psi,$$,
где 
\[
    V(x, y) = 
    \Bigg\{
    \begin{array}{cc}
        0, \rho_0 < \sqrt{x^2+y^2} < \rho_1 \\
        \infty, \rho_1 < \sqrt{x^2+y^2} \text{ или } \sqrt{x^2+y^2} < \rho_0.\\
    \end{array}
\] 
 $V(r)$ внутри области $\Omega$ равен нулю, а вне ее обращается в бесконечность. Задача равносильна поиску собственных функций и собственных значений оператора Лапласа в области $\Omega$ для функций, обращающихся в нуль на границе $\Omega$.
Положим  $\varkappa^2 = \frac{2 M E}{\hbar^2}$. Далее $J_\nu$  и $Y_\nu$ --- функции Бесселя первого и  второго рода соответственно.

\begin{theorem}{\normalfont (для $p=1$ см. \cite[с.~165.]{wref10})}.

В  области $\Omega$ ($p$-листном накрытии кругового кольца)  собственные функции $\psi_{k,m}(r,\phi)$ и собственные значения $E_{k,m}$ оператора $\hat{H}$ имеют вид
\begin{multline*}
\psi_{k,m}(r,\phi) = \biggl[ Y_\nu(\alpha_{\nu, m}) J_\nu\biggl(\frac{\alpha_{\nu, m}r}{r_0}\biggr) - Y_\nu\biggl(\frac{\alpha_{\nu, m}r}{r_0}\biggr) J_\nu(\alpha_{\nu, m}) \biggr] \cos{(\nu \phi+\phi_0)}, \\
\quad E_{k,m}= \frac{\varkappa^2_{k,m}\hbar^2}{2M},
\end{multline*}
где 
$\nu=\frac{k}{p}$,  $\lambda = \frac{r_1}{r_0}$,
$\varkappa^2_{k,m}=\frac{\alpha_{\nu, m}^2}{r_0^2},  k, m \in \mathbb{N}$, $\alpha_{\nu, m}$ --- $m$-ый нуль функции $f(x) = Y_\nu(x) J_\nu(\lambda x) - Y_\nu(\lambda x) J_\nu(x)$.
\label{th:sect1_theorem1}
\end{theorem}
\begin{proof}
%Запишем искомую функцию в виде $\psi(r, \phi) = R(r)\Phi(\phi)$, тогда уравнение $(\nabla^2 + \varkappa^2)\psi = 0$ приобретет вид
%$ \frac{R(r)\Phi''(\phi)}{r^2} + \frac{R'(r)\Phi(\phi)}{r} + \Phi(\phi) R''(r) +\varkappa^2R(r)\Phi(\phi) = 0$. Умножим обе части уравнения на $\frac{r^2}{R(r)\Phi(\phi)}$:
%$$\frac{\Phi''(\phi)}{\Phi(\phi)} + \frac{rR'(r)}{R(r)} + \frac{r^2R''(r)}{R(r)} + \varkappa^2r^2 = 0.$$
%
%Введем разделяющий параметр  $\nu$ и получим два уравнения (далее переменные явно не указываем, подразумевая, что $\Phi = \Phi(\phi), R=R(r)$):
%\[
%\left\{\begin{array}{cc}
%    \dfrac{\Phi''}{\Phi} =-\nu^2, \\[10pt]
%    \dfrac{rR'}{R} + \dfrac{r^2R''}{R} + \varkappa^2r^2 = \nu^2.
%\end{array}
%\right.
%\]
Стационарное уравнение Шрёдингера \eqref{eq:schrodinger}   в полярной системе координат приводит к системе вида  \eqref{eq:Bessel_equations}.
Решением углового уравнения  является функция $\Phi(\phi)  = \cos{(\nu \phi + \phi_0)}$ для некоторого вещественного значения $\phi_0$. Из условия периодичности $\Phi(0) = \Phi(2\pi p)$ следует, что 
$\nu = \frac{k}{p}$, где $k$ --- произвольное неотрицательное целое число. 

Решение радиального уравнения ищется в виде линейной  комбинации функций Бесселя первого и второго рода \cite[\S\ 9, с.~358]{wref2}:
$$R(r) = A J_\nu(\varkappa r) +B Y_\nu(\varkappa r).$$
Из граничного условия $R(r_0) = 0$ установим значения констант: $A = Y_\nu(\varkappa r_0), B = -J_\nu(\varkappa r_0)$ (либо пропорциональные им). 

Теперь рассмотрим функцию  $f(x)=Y_\nu(x) J_\nu(\lambda x) - Y_\nu(\lambda x) J_\nu(x)$, где $\lambda = \frac{r_1}{r_0}$.
Тогда граничное условие $R(r_1) = Y_\nu(\varkappa r_0) J_\nu(\varkappa r_1) -J_\nu(\varkappa r_0) Y_\nu(\varkappa r_1) =0$ можно записать в виде
$f(\varkappa r_0)=0$.
Обозначим $m$-й положительный нуль этой функции через $\alpha_{\nu, m}$. Тогда  $\varkappa r_0 = \alpha_{\nu, m} $ для какого-то значения $m$,
откуда следует, что $\varkappa$ может принимать только значения $\varkappa^2_{k,m}$, приведенные в формулировке теоремы \ref{th:sect1_theorem1}.
\end{proof}

\subsection{Квантовый биллиард в секторе круга $D^2: \phi \in (0, \theta)$}\label{sec:ch1/sec2/sec3}
Зафиксируем угол $\theta \in (0, 2\pi)$ и в полярных координатах $(r, \phi)$ в качестве $\Omega$ рассмотрим область 
$S^2 = \{r, \phi : 0 \leq r \leq \rho_0 , 0 \leq \phi \leq \theta\}$.
\[
    V(x, y) = 
    \Bigg\{
    \begin{array}{cc}
        0, r \in (0, \rho_0), \phi \in (0, \theta) \\
        \infty, \text{ иначе}.\\
    \end{array}
\] 

\begin{statement} \cite[p.~4]{wref13}
В области $S^2$ собственные функции $\psi_{k, m}(r, \phi)$ и собственные значения $E_{k, m}$ оператора $\hat{H}$ имеют вид 
$$\psi_{k, m}(r, \phi) = J_\lambda\left(\frac{\alpha_{\lambda, m}r}{\rho_0}\right)\sin{\lambda\phi}, \hspace{15pt} E_{k, m} = \frac{\hbar^2 \alpha_{\lambda, m}^2}{2M\rho_0^2}, \hspace{15pt} k, m \in \mathbb{Z},$$
где $\lambda = \frac{\pi k}{\theta}$, $\alpha_{\lambda, m}$ - $m$-ый ноль функции Бесселя первого рода $J_\lambda(x)$.
\label{st:sect1_stat1}
\end{statement}
\begin{proof}
Ищем решение в виде $\psi(r, \phi) = R(r)\Phi(\phi)$. Повторяя соображения из предыдущего пункта, получим систему дифференциальных уравнений для биллиарда в круге 
\[
\left\{\begin{array}{cc}
    \dfrac{\Phi''}{\Phi} =-\lambda^2, \\
    \dfrac{rR'}{R} + \dfrac{r^2R''}{R} + \varkappa^2r^2 = \lambda^2,
\end{array}
\right.
\]
решение которой должно удовлетворять новому граничному условию $\Phi(0)=\Phi(\theta)=0$. Поэтому $\Phi(\theta) = \sin(\lambda \theta) \in \{\pi k, k \in \mathbb{Z}\} \implies \lambda = \frac{\pi k}{\theta}, k \in \mathbb{Z}$.

В силу соображений из предыдущего пункта, решение системы дифференциальных уравнений имеет вид:
$$\psi_{k, m}(r, \phi) = J_\lambda\left(\frac{\alpha_{\lambda, m}r}{\rho_0}\right)\sin{\lambda\phi} , \quad \lambda=\frac{\pi k}{\theta}.$$
Аналогичными рассуждениями можно обнаружить, что собственные значения в этом случае принимают вид 
$E_{k, m} = \frac{\hbar^2 \alpha_{\lambda, m}^2}{2M\rho_0^2}$ для $\lambda=\frac{\pi k}{\theta}.$
\end{proof}

\begin{consequence} \cite[p.~4]{wref13}
Нетрудно заметить, что индексы $\lambda_m = \frac{m \pi}{\theta}$, соответствующие сектору $0 \leq \phi \leq \theta$ связаны с индексами $\mu_n = \frac{n \pi}{2\pi - \theta}$ ($\mu_n$ соответствуют решениям в дополнении сектора, т.е. для $\theta \leq \phi \leq 2\pi$) следующим образом:
$$\lambda_m = \frac{\mu_n}{2\mu_n - n}m, \hspace{15pt} \mu_n = \frac{\lambda_m}{2\lambda_m - m}n$$
\end{consequence}
\begin{proof}
Действительно, имеем 
$\frac{n \pi}{2\pi - \theta} = \mu_n$, $\frac{m \pi}{\theta} = \lambda_m$, тогда из равенства $2\pi = \theta + (2\pi - \theta) = \frac{m \pi}{\lambda_m} + \frac{n \pi}{\mu_n}$ получаются равенства из утверждения \ref{st:sect1_stat1}.
\end{proof}

\section{Оператор Лапласа в эллиптической системе координат.}\label{sec:ch1/sec3}

%---стационарное уравнение Шрёдингера в эллиптических координатах---
Рассмотрим область, ограниченную эллипсом с большой и малой полуосями, соответственно равными $w$ и $h$.
Обозначим половину расстояния между фокусами эллипса через  $\delta = \sqrt{w^2 - h^2}$.
Введем эллиптические координаты $\rho, \phi$, $\rho\ge 0, 0\le\phi \le 2\pi$, где 
$$(x, y) = (\delta\cosh{\rho}\cos{\phi}, \delta\sinh{\rho}\sin{\phi}). $$
Они регулярны вне отрезка, соединяющего фокусы $(\pm\delta,0)$.
 Рассматриваемая область задается неравенством $0 \leq \rho \leq \text{arccosh} (\frac{w}{\delta})$. 
При фиксированном $w=r_0$ и  $\delta\to 0$
область ``стремится'' к кругу радиуса $r_0$.
 
 В этой системе координат оператор Лапласа имеет вид
$$\nabla^2 = \frac{\partial^2}{\partial x^2} + \frac{\partial^2}{\partial y^2} = \frac{\frac{\partial^2}{\partial \rho^2} + \frac{\partial^2}{\partial \phi^2}}{\delta^2(\cosh^2{\rho} - \cos^2{\phi})} = 
\frac{\frac{\partial^2}{\partial \rho^2} + \frac{\partial^2}{\partial \phi^2}}{\frac{\delta^2}{2}(\cosh{2\rho} - \cos{2\phi})}.$$
%тогда оператор Шрёдингера имеет вид $\hat{H} = \frac{-\hbar^2}{2m} \nabla^2.$

 
Стационарное уравнение Шрёдингера 
переписывается как
$$ \nabla^2 \psi +\varkappa^2\psi =  0, \text{  где  }\varkappa^2 =\frac{2ME}{\hbar^2}$$
с условием, что $\psi$ на границе области обращается в нуль.
Разделяя переменные $\psi(\rho,\phi) = R(\rho)\Phi(\phi)$, приведем уравнение к виду
$$\Phi\frac{\partial^2}{\partial \rho^2}R + R\frac{\partial^2}{\partial \phi^2}\Phi + \frac{(\varkappa \delta)^2}{2}(\cosh{2\rho} - \cos{2\phi})R\Phi = 0. $$
В скобках добавим и вычтем разделяющий параметр $\frac{2\zeta}{(\varkappa \delta)^2}$, получим \textit{уравнения Матьё}, в которых $q=\frac{(\varkappa \delta)^2}{4}$:
\begin{equation}
\left\{
\begin{array}{rcll}

			\frac{\partial^2}{\partial \phi^2}\Phi + (\zeta - 2q\cos{2\phi})\Phi &= &0 \quad 	 &\textit{   (угловое уравнение Матьё)}, \\
		\frac{\partial^2}{\partial \rho^2}R - (\zeta - 2q\cosh{2\rho})R &=& 0 	\quad	& \textit{   (радиальное уравнение Матьё)}.  
\end{array}
%\tag{1}
\right.
\label{eq:mathieusystem}
\end{equation}


\subsection{Функции Матьё.}\label{sec:ch1/sec3/sub1}
Рассмотрим угловое уравнение Матьё $\frac{d^2}{d z^2}\Phi(z) + (\zeta - 2q\cos{2 z})\Phi(z) =0$. 
Поскольку коэффициенты углового уравнения Матьё периодичны по $z$, по теореме Флоке \cite{wref2} существует решение в виде $F_\nu(z) = e^{i\nu z}P(z)$, где $\nu$ зависит от параметров $\zeta$ и $q$, а функция $P(z)$ имеет тот же период $\pi$, что и коэффициенты уравнения. Постоянную $\nu$ называют характеристической экспонентой. При $\nu \notin \mathbb{Z}$ функции $F_\nu(z)$ и $F_\nu(-z)$ являются независимыми решениями дифференциального уравнения. При $\nu \in \mathbb{Z}$ функции $F_\nu(z)$ и $F_\nu(-z)$ являются пропорциональными и имеют период $\pi$ или $2 \pi$ (см. \cite{wref2}).

%При $\nu = \frac{m_1}{m_2}$ оба решения являются периодическими и имеют период не более $2\pi m_2$,
%см. [2].

Согласно теории Штурма при $q \neq 0$ возможно существование не более чем одного периодического решения с периодом $\pi$ или $2 \pi$. В зависимости от четности и периода этого решения параметр\footnote[2]{
		Можно рассмотреть уравнение Матьё как задачу на собственные значения и собственные функции оператора $D(y) = \dfrac{d^2y}{dx^2} - 2q\cos(2x) y$ (или оператора $D(y) = \dfrac{d^2y}{dx^2} - 2q\cosh(2x) y$). Поэтому в литературе  $\zeta$ часто называют собственными значениями.
}   $\zeta$ относится к одному из двух типов:
\[
\zeta = \left[
\begin{array}{cccc}
	a_\nu(q), 					& \nu \in \{0\} \cup \mathbb{N}; \\
	b_{-\nu}(q), 					& -\nu \in \mathbb{N},
\end{array}
\right.
\]
более точно, для $n\in\{0\}\cup\mathbb{N}$ (см. табл. \ref{tab:table1}).

%\chapter{Асимптотика собственных значений в квантовых эллиптических биллиардах}\label{ch:ch2}

%\section{Предварительные сведения}\label{sec:ch2/sec1}
\begin{table} [htbp]%
    \centering
    \caption{Периодические функции Матьё целого порядка}%
    \label{tab:table1}% label всегда желательно идти после caption
	%    \renewcommand{\arraystretch}{1.5}%% Увеличение расстояния между рядами, для улучшения восприятия.
    \begin{SingleSpace}
	\begin{tabular}{||c | c | c | c||} 
            \toprule     %%% верхняя линейка
            $\zeta$	&   \begin{tabular}{c}Периодическое решение\\ углового уравнения Матьё\footnotemark[3]\end{tabular} &   Период  & Четность функции \\
            \midrule 
		$a_{2n}(q)$                   &   $ce_{2n}(z, q)$               & период $\pi$     & четная \\ \hline
		$a_{2n+1}(q)$                 &   $ce_{2n+1}(z, q)$             & антипериод\footnotemark[4] $\pi$ & четная \\ \hline
		$b_{2n+1}(q)$                 &   $se_{2n+1}(z, q)$             & антипериод $\pi$ & нечетная \\ \hline          
		$b_{2n+2}(q)$                 &   $se_{2n+2}(z, q)$             & период $\pi$     & нечетная \\ 
		            \bottomrule %%% нижняя линейка
        \end{tabular}%
    \end{SingleSpace}
\end{table}
\footnotetext[3]{В табл. 1 приведены только собственные функции периода $\pi$ или $2\pi$.}
\footnotetext[4]{Антипериод $\pi$: $f(x+\pi) = -f(x)$.}

Для $\zeta=a_n(q)$ соответствующие непериодичные нечетные решения обозначают как $fe_n(z, q)$. Аналогично, для $\zeta=b_n(q)$ непериодичные четные решения обозначают как $ge_n(z, q)$. 
Выделяют также третий тип: $\zeta = \lambda_\nu(q), \nu \notin \mathbb{Z}$, которому соответствуют функции Матьё нецелого порядка $ce_\nu(z, q), se_\nu(z, q)$. В общем случае при $\nu \notin \mathbb{Q}$ обе функции являются непериодическими, однако для $\nu \in \mathbb{Q} \setminus \mathbb{Z}, \nu = \frac{n}{p}$, обе имеют период не более $2\pi p$. Таблица \ref{tab:table1} для $\nu = \frac{n}{p}$ может быть продолжена (см. табл. \ref{tab:table2}).

\begin{table}[htbp]
    \centering
        \caption{Периодические функции Матьё нецелого порядка $\nu$}%
    \label{tab:table2}% label всегда желательно идти после caption
	%    \renewcommand{\arraystretch}{1.5}%% Увеличение расстояния между рядами, для улучшения восприятия.
    \begin{SingleSpace}
	\begin{tabular}{||c | c | c | c||} 
            \toprule     %%% верхняя линейка
		 $\zeta$	&   \begin{tabular}{c}Периодическое решение\\ углового уравнения Матьё\end{tabular} &   Период  & Четность функции \\ [0.5ex] 
            \midrule 
		$\lambda_\nu(q)$                   &   $ce_\nu(z, q)$               & период $\pi p$     & четная \\ \hline
		$\lambda_\nu(q)$                   &   $se_\nu(z, q)$               & антипериод $\pi p$     & нечетная \\

	    \bottomrule %%% нижняя линейка
	\end{tabular}
    \end{SingleSpace}
\end{table}



Смысл параметра $\nu$ становится понятным при подстановке $q=0$ в угловое уравнение Матьё \eqref{eq:mathieusystem}. В этом случае угловая функция получается той же, что и в случае диска, следовательно, $\lambda_\nu(0) = \nu^2, ce_\nu(z, 0) = \cos(\nu z), se_\nu(z, 0) = \sin(\nu z)$.

Ряды Фурье для угловых функций Матьё сходятся равномерно и абсолютно на всех компактных множествах в комплексной плоскости. В приведенных ниже формулах предполагается, что $n \in \{0\} \cup \mathbb{N}$, $\nu \in \mathbb{R} \setminus \mathbb{Z}$:

{\small
\[
\begin{array}{llll}
	ce_{2n}(z, q) &= \sum_{m=0}^\infty A_{2m}^{2n}(q) \cos{2mz}, &
	ce_{2n+1}(z, q) &= \sum_{m=0}^\infty A_{2m+1}^{2n+1}(q) \cos{(2m+1)z}, \\
	se_{2n+1}(z, q) &= \sum_{m=0}^\infty B_{2m+1}^{2n+1}(q) \sin{(2m+1)z}, \ \ \ \ &
	se_{2n+2}(z, q) &= \sum_{m=0}^\infty B_{2m+2}^{2n+2}(q) \sin{(2m+2)z}, \\
    ce_\nu(z, q) &= \sum_{m=-\infty}^\infty c_{2m}^\nu(q) \cos{(\nu + 2m)z}, & 
    se_\nu(z, q) &= \sum_{m=-\infty}^\infty c_{2m}^\nu(q) \sin{(\nu + 2m)z}.
\end{array}
\]
}
Коэффициенты $A_k^l, B_k^l, c_k^\nu$ удовлетворяют определенным рекуррентным соотношениям (см. \cite{wref2} ).

Похожие разложения для функций $fe_n(z,q)$ и $ge_n(z,q)$ не приводятся, поскольку для $n \geq 2$ разность между $ce_n(z,q)$ и $ge_n(z,q)$ (и между $se_n(z,q)$ и $fe_n(z,q)$) имеет порядок $o(q)$, чего достаточно для целей настоящей работы. Заметим, что для $\nu = 1$ разложения для $fe_1(z,q)$ и $ge_1(z,q)$ соответствуют особым случаям, выходящим за рамки исследования.

Теперь обратимся к радиальным функциям Матьё. Радиальные функции Матьё первого рода и целого порядка определяются как 
$Ce_n(z, q) = ce_n(\pm i z, q)$, $Se_n(z, q) = \mp i se_n(\pm i z, q)$ (см., например, \cite{mclachlan}).
Для них и для радиальной функции нецелого порядка $M_\nu^{(1)}(z, q)$ имеют место разложения по функциям Бесселя первого рода (здесь равенство понимается с точностью до умножения на не зависящую от $z$ постоянную, которая не представляет интереса для настоящей работы):
{\small
\begin{equation}
\begin{array}{l}
\ Se_{2n+1}(z, q)  \propto \tanh{z}\sum\limits_{m=1}^\infty (-1)^{m} (2m+1) B_{2m+1}^{2n+1}(q) J_{2m+1}(x), \\
\begin{array}{ll}

	Se_{2n}(z, q) \propto \tanh{z}\sum\limits_{m=1}^\infty (-1)^m 2m B_{2m}^{2n}(q) J_{2m}(x), & \\
	Ce_{2n}(z, q) \propto	 \sum\limits_{m=0}^\infty (-1)^m A_{2m}^{2n}(q) J_{2m}(x), &
	Ce_{2n+1}(z, q)  \propto \sum\limits_{m=0}^\infty (-1)^{m+1} A_{2m+1}^{2n+1}(q) J_{2m+1}(x), \\
	M_\nu^{(1)}(z, q) \propto \sum\limits_{m=-\infty}^\infty (-1)^m c_{2m}^\nu(q)J_{\nu+2m}(x), &
        \text{везде для краткости } x = 2\sqrt{q}\cosh z. 
\end{array}
\end{array}
%\tag{2}
\label{eq:mathieuRadial}
\end{equation}
}
Здесь  коэффициенты $A_k^l, B_k^l, c_k^\nu$ те же, что и в разложении функций $ce_l(z, q), se_l(z, q), ce_\nu(z, q)$ в ряды Фурье (см. \cite[гл. VIII, с.~158--169]{mclachlan}). 
Заменой функций Бесселя первого рода $J_m(x)$ на функции Бесселя второго рода $Y_m(x)$ в вышеизложенных формулах можно получить независимые решения соответствующих уравнений. Так, для радиальных функций первого рода целого порядка $Ce_n(z, q)$ имеем второе решение $Fey_n(z, q)$, а для функций $Se_n(z, q)$ такое второе решение обозначают как $Gey_n(z, q)$  (см. \cite[гл. VIII, \S\ 8.11--13, с.~158--162]{mclachlan}). Из тех же соображений применительно к $M_\nu^{(1)}(z, q)$ появляется независимое решение $M_\nu^{(2)}(z, q)$ для случая нецелого порядка.
%---Теория Флоке, характеристическая экспонента---



\section{Квантовый биллиард в областях $\Omega$, ограниченных софокусными квадриками}\label{sec:ch1/sec4}
\subsection{Квантовый биллиард в эллипсе}\label{sec:ch1/sec4/sub1}
Рассмотрим спектр оператора Лапласа в области, ограниченной эллипсом с большой и малой полуосями, равными $a$ и $b$. В соответствии с \ref{sec:ch1/sec3}, при половине расстояния между фокусами $\delta = \sqrt{a^2-b^2}$ в эллиптической системе координат $(x, y) = (\delta\cosh{\rho}\cos{\phi}, \delta\sinh{\rho}\sin{\phi})$ этот эллипс задается неравенством $0 \leq \rho \leq \rho_0 = \text{arccosh} (\frac{a}{\delta})$. 


Эллиптическая система координат имеет особенности на соединяющем фокусы отрезке, который целиком содержится во внутренности рассматриваемой области. Пусть  $J$ -- отрезок, содержащий особые точки эллиптической системы координат.
Собственная функция $\psi(\rho, \phi)$ оператора Лапласа в эллипсе и ее производная должны удовлетворять условиям непрерывности на  $J$:
\begin{align}
& \psi(0,\phi) = \psi(0,-\phi)\notag \\
 &   \qquad\qquad\qquad\qquad     \text{\em (непрерывность сдвига через  $J$)} \label{eq:disp}, \\[10pt]
 &   \frac{\partial}{\partial \rho} \psi( \rho, \phi)|_{\rho \to 0} = - \frac{\partial}{\partial \rho} \psi(\rho, - \phi)|_{\rho \to 0} \notag \\
 &\qquad\qquad\qquad\qquad   \text{\em (непрерывность производной через $J$)}\label{eq:grad}, 
\end{align}
также см. \cite[XVI p.~294]{mclachlan}.

Используя уравнения (\ref{eq:disp}) и (\ref{eq:grad}), мы покажем, что если собственная функция  $\psi(\rho,\phi)$ оператора $\hat{H}$ в содержащей особые точки эллиптической системы координат области представлена в виде произведения функций одного переменного $\psi(\rho,\phi ) =  R(\rho) \Phi(\phi)$, тогда  функции  $R(\rho), \Phi(\phi)$ имеют одинаковую четность.


Общее решение радиального и углового уравнений Матьё~\eqref{eq:mathieusystem} представимо в виде
\[
\begin{array}{cc}
\Phi(\zeta, q, \phi) = A_\phi \Phi_{even}(\zeta, q, \phi) + B_\phi \Phi_{odd}(\zeta, q, \phi) \\
R(\zeta, q, \rho) = A_\rho R_{even}(\zeta, q, \rho) + B_\rho R_{odd}(\zeta, q, \rho).
\end{array}
\]
Очевидно, $\psi(0,\pm\phi) = \Phi(\pm \phi) A_\rho R_{even}(0)$.
Из \textit{непрерывности сдвига}  (\ref{eq:disp}) мы получаем
\begin{equation} 
\Phi(\phi) A_\rho = \Phi(-\phi) A_\rho.\label{eq:plus}\end{equation}


Также, $\frac{\partial}{\partial \rho} \psi( \rho,\pm \phi)|_{\rho \to 0} = \Phi(\pm\phi) B_\rho R'_{odd}(0)$.

Условие \textit{непрерывности производной} (\ref{eq:grad}) влечет
\begin{equation}\label{eq:minus}
 \Phi(\phi) B_\rho = -\Phi(-\phi) B_\rho
\end{equation}

Наконец, из уравнений (\ref{eq:plus}), (\ref{eq:minus}) следует, что одна из постоянных $A_\rho$ и $B_\rho$ равна нулю.
Таким образом, возможны два случая:
\[ \psi(\rho,\phi) = \Phi_{even}(\phi) R_{even}(\rho)  \] 
или
\[ \psi(\rho,\phi) = \Phi_{odd}(\phi) R_{odd}(\rho) .  \]
%\hfill
%$\Box$
\subsubsection{Собственные функции}\label{sec:ch1/sec4/sub1/sub1}
В случае эллипса на угловую функцию Матьё накладывается уловие периодичности: $\Phi(\phi) = \Phi(\phi+ 2 \pi)$. 
Тогда, как следует из табл. \ref{tab:table1}, четные периодические функции $\Phi_{even}(\phi)$ соответствуют случаю $\zeta = a_m(q)$, при этом $\Phi_{even}(\phi) = ce_m(\phi, q)$. 
Нечетные периодические решения $\Phi_{odd}(\phi)$ -- это функции 
$se_m(\phi, q)$, соответствующие случаю $\zeta = b_m(q)$ для некоторого целого $m$. Таким образом, нами доказано
    
\begin{statement}
Собственные функции $\psi(\rho, \phi)$ оператора $\hat{H}$ в эллипсе имеют вид:
\[
    \psi_{k,m}(\rho, \phi) = \left[
    \begin{array}{ccc}
    Ce_k(\rho, q_{k,m})ce_k(\phi, q_{k,m}), & \zeta = a_k(q_{k, m}); \\
    Se_k(\rho, q_{k,m})se_k(\phi, q_{k,m}), & \zeta = b_k(q_{k, m}),
    \end{array}
    \right.
\]
$k, m \in \mathbb{Z}$. При этом собственное значение  $q_{k, m}$ -- $m$-ый нуль функции $Ce_k(\rho_0, q)$ или $Se_k(\rho_0, q)$ как функции от $q$.
Значение параметра $q_{k,m}$ связано с соответствующим собственным значением $E_{k,m}$ формулой  $q_{k,m} = \frac{M E_{k,m}\delta^2}{2\hbar^2} = \frac{\varkappa_{k,m}^2\delta^2}{4}$. 
\end{statement}

\subsubsection{Асимптотика собственных значений}\label{sec:ch1/sec4/sub1/sub2}
\begin{theorem}
Собственные значения $\varkappa^2_{k,m}(\delta), k, m \in \mathbb{N}$, оператора $\hat{H}$ в эллипсе зависят от половины фокусного расстояния $\delta$ с точностью до $o(\delta^2)$ следующим образом:
{
\begin{equation}
\varkappa^2_{k,m}(\delta) = \left[
\begin{array}{ccc}
\dfrac{\alpha_{k, m}^2}{a^2} +  \delta^2 
\dfrac{\alpha_{k, m}^3}{8 a^4}
\left. \frac{\frac{1}{k-1} J_{k-2}(u) - \frac{1}{k+1}J_{k+2}(u)}{\frac{\partial J_{k} (u)}{\partial u}}\right|_{u=\alpha_{k, m}} , 
 & \zeta = a_k(q), & k \geq 2; \\
\dfrac{\alpha_{1, m}^2}{a^2} -  \delta^2 \dfrac{\alpha_{1, m}^3}{16 a^4}\left.\dfrac{J_{3}(u)}{\frac{\partial J_{1} (u)}{\partial u}}\right|_{u=\alpha_{1, m}}, & \zeta = a_1(q); \\
\dfrac{\alpha_{0, m}^2}{a^2} -  \delta^2 \dfrac{\alpha_{0, m}^3}{4 a^4}\left.\dfrac{J_{2}(u)}{\frac{\partial J_{0} (u)}{\partial u}}\right|_{u=\alpha_{0, m}},
 & \zeta = a_0(q); \\
\dfrac{\alpha_{k, m}^2}{a^2} +  \delta^2 
\dfrac{\alpha_{k, m}^3}{8 k a^4}
\left.\frac{\frac{k-2}{k-1} J_{k-2}(u) - \frac{k+2}{k+1} J_{k+2}(u)}{\frac{\partial J_{k} (u)}{\partial u}}\right|_{u=\alpha_{k, m}} , & \zeta = b_k(q), & k > 2; \\
\dfrac{\alpha_{2, m}^2}{a^2} -  \delta^2 \dfrac{\alpha_{2, m}^3}{12 a^4}\left.\dfrac{J_{4}(u)}{\frac{\partial J_{2} (u)}{\partial u}}\right|_{u=\alpha_{2, m}},
 & \zeta = b_2(q); \\
\dfrac{\alpha_{1, m}^2}{a^2} -  \delta^2 \dfrac{3\alpha_{1, m}^3}{16 a^4}\left.\dfrac{ J_{3}(u)}{\frac{\partial J_{1} (u)}{\partial u}}\right|_{u=\alpha_{1, m}},
 & \zeta = b_1(q), \\
\end{array}
\right.
\label{eq:valEllipse}:
\end{equation}
}
где $\alpha_{k, m}$ -- $m$-ый нуль функции $J_k(x)$. 
\label{th:stat4}
\end{theorem}

Приведем две леммы, с помощью которых докажем \ref{eq:valEllipse}:

Пусть $\zeta = a_k(q_{k, m})$, $q=\frac{\varkappa^2 \delta^2}{4}$, и пусть $\psi_{k,m}(\rho, \phi) = Ce_k(\rho, q)ce_k(\phi, q)$ -- произведение решений радиального и углового уравнений Матьё с указанными параметрами $\zeta, q$. Для малых $q$ справедливо разложение (см. \cite[\S~2.2, с.~122---124]{wref12}):
$$ce_k(\phi, q) = c_k \cos{k \phi} + q c_{k+2} \cos{(k+2) \phi} +q c_{k-2} \cos{(k-2) \phi} + o(q).$$ 
\begin{lemma}
Пусть $\zeta = a_k(q_{k, m})$, тогда $\varkappa^2$ при малых $\delta$ имеет вид
\begin{multline*}
\varkappa_{k, m}^2 = 
\frac{\alpha_{k, m}^2}{a^2} +  \delta^2 \frac{\alpha_{k, m}^3}{2 a^4}\frac{1}{\left.\frac{\partial J_{k} (u)}{\partial u}\right|_{u=\alpha_{k, m}}} 
\biggl(
\frac{c_{k-2}}{c_k} J_{k-2}(\alpha_{k, m}) + \frac{c_{k+2} }{c_k} J_{k+2}(\alpha_{k, m})
\biggr) + o(\delta^2),
\end{multline*}
где $\alpha_{k, m}$ -- $m$-ый нуль функции Бесселя первого рода $J_{k}(x)$.
\label{th:lemEllipse1}
\end{lemma}
\begin{proof}
Возможные значения $\varkappa^2$ определяются из условия обращения в нуль радиальной функции Матьё $Ce_k(\rho, q)$ на граничном эллипсе $\rho = \rho_0 = \text{arccosh} (\frac{a}{\delta})$.
Коэффициенты $\{c_k\}$ связаны \cite{wref2} с точностью до постоянного множителя с разложением радиальной функции Матьё $Ce_k(\rho, q)$ в бесконечную сумму функций Бесселя следующим образом:
\begin{multline*}
Ce_k(\rho, q) \propto 
	c_k J_k(2\sqrt{q}\cosh{\rho}) - \\
	- q c_{k-2} J_{k-2}(2\sqrt{q}\cosh{\rho}) -
	q c_{k+2} J_{k+2}(2\sqrt{q}\cosh{\rho}) + o(q).
\end{multline*}
Заметим, что из-за равенства $q = \frac{\varkappa^2\delta^2}{4}$ аргументы имеют вид $2 \sqrt{q} \cosh{\rho_0} = 2 \sqrt{\frac{\varkappa^2 \delta^2}{4}} \frac{a}{\delta} = \varkappa a$. 
Запишем для краткости $u = \varkappa a$ и перепишем граничное условие $Ce_k(\rho_0, q) = 0$:
$$0 = Ce_k(\rho_0, q) =
	c_k J_k(u) 
	- q c_{k-2} J_{k-2}(u) -
	q c_{k+2} J_{k+2}(u) + o(q).$$
Пусть $u = u_0 + q u_1 + o(q)$. Граничное условие может быть переписано с группировкой слагаемых при различных степенях $q$ следующим образом:
$$0 =
	c_k J_k(u_0) + q \biggl(
	c_k u_1 \left.\frac{\partial  J_k(u)}{\partial u}\right|_{u=u_0}
	-  c_{k-2} J_{k-2}(u_0) - c_{k+2} J_{k+2}(u_0) 
	\biggr)+ o(q).$$
Из равенства нулю коэффициентов при каждой степени $q$ следует
\begin{align*}
&u_0 = \alpha_{k, m}, \\
&u_1 = \frac{1}{\left.\frac{\partial J_{k} (u)}{\partial u}\right|_{u=\alpha_{k, m}}} 
\biggl(
\frac{c_{k-2}}{c_k} J_{k-2}(\alpha_{k, m}) + \frac{c_{k+2} }{c_k} J_{k+2}(\alpha_{k, m})
\biggr),
\end{align*}
где $\alpha_{k, m}$ -- $m$-ый нуль функции $J_k(x)$.  

Из равенства $q=\frac{\varkappa^2 a^2 \delta^2}{4 a^2}=\frac{u^2 \delta^2}{4a^2}$ можно получить выражение для собственного значения
$$\varkappa_{k, m}^2 = \frac{u^2}{a^2} = \frac{u_0^2}{a^2} + \frac{2 q u_0 u_1}{a^2} + o(q)= \frac{u_0^2}{a^2} +  \delta^2 \frac{u_0^3 u_1}{2 a^4} + o(\delta^2).$$ 
Непосредственной подстановкой $u_0, u_1$ в полученную формулу, получаем
$$\varkappa_{k, m}^2 = 
\frac{\alpha_{k, m}^2}{a^2} +  \delta^2 \frac{\alpha_{k, m}^3}{2 a^4}\frac{1}{\left.\frac{\partial J_{k} (u)}{\partial u}\right|_{u=\alpha_{k, m}}} 
\biggl(
\frac{c_{k-2}}{c_k} J_{k-2}(\alpha_{k, m}) + \frac{c_{k+2} }{c_k} J_{k+2}(\alpha_{k, m})
\biggr) + o(\delta^2).
$$ 
\end{proof}

Пусть $\zeta = b_k(q_{k,m})$, $q=\frac{\varkappa^2 \delta^2}{4}$, и пусть 
$\psi_{k,m}(\rho, \phi) = Se_k(\rho, q)se_k(\phi, q)$ -- произведение решений радиального и углового уравнений Матьё с указанными параметрами $\zeta, q$. Для малых $q$ справедливо разложение (см. \cite[\S~2.2, с.~122---124]{wref12}):
$$se_k(\phi, q) = c_k \sin{k \phi} + q c_{k+2} \sin{(k+2) \phi} +q c_{k-2} \sin{(k-2) \phi} + o(q).$$ 
\begin{lemma}
Пусть $\zeta = b_k(q_{k, m})$, тогда $\varkappa^2$ при малых $\delta$ имеет вид
\begin{multline*}
\varkappa_{k, m}^2 = 
\frac{\alpha_{k, m}^2}{a^2} +  \delta^2 \frac{\alpha_{k, m}^3}{2 a^4}\frac{1}{\left.\frac{\partial J_{k} (u)}{\partial u}\right|_{u=\alpha_{k, m}}} \times \\ \times
\biggl(
\frac{(k-2)c_{k-2}}{k c_k} J_{k-2}(\alpha_{k, m}) + \frac{(k+2)c_{k+2} }{k c_k} J_{k+2}(\alpha_{k, m})
\biggr) + o(\delta^2),
\end{multline*}
где $\alpha_{k, m}$ -- $m$-ый нуль функции Бесселя первого рода $J_{k}(x)$.
\label{th:lemEllipse2}
\end{lemma}
\begin{proof}
Возможные значения $\varkappa^2$ определяются из условия обращения в нуль радиальной функции Матьё $Se_k(\rho, q)$ на граничном эллипсе $\rho = \rho_0$.
Коэффициенты $\{c_k\}$ связаны \cite{wref2} с точностью до постоянного множителя с разложением радиальной функции Матьё $Se_k(\rho, q)$ в бесконечную сумму функций Бесселя следующим образом:
\begin{multline*}
Se_k(\rho, q) \propto 
	k c_k J_k(2\sqrt{q}\cosh{\rho}) - \\
	- q (k-2) c_{k-2} J_{k-2}(2\sqrt{q}\cosh{\rho}) -
	q (k+2) c_{k+2} J_{k+2}(2\sqrt{q}\cosh{\rho}) + o(q).
\end{multline*}
Заметим, что из-за равенства $q = \frac{\varkappa^2\delta^2}{4}$ аргументы имеют вид $2 \sqrt{q} \cosh{\rho_0} = 2 \sqrt{\frac{\varkappa^2 \delta^2}{4}} \frac{a}{\delta} = \varkappa a$. 
Запишем для краткости $u = \varkappa a$ и перепишем граничное условие $Se_k(\rho_0, q) = 0$:
$$0 = Se_k(\rho_0, q) =
	k c_k J_k(u) 
	- q (k-2) c_{k-2} J_{k-2}(u) -
	q (k+2) c_{k+2} J_{k+2}(u) + o(q).$$
Пусть $u = u_0 + q u_1 + o(q)$. Граничное условие может быть переписано с группировкой слагаемых при различных степенях $q$ следующим образом:
\begin{multline*}
0 =
	k c_k J_k(u_0) + q \biggl(
	k c_k u_1 \left.\frac{\partial  J_k(u)}{\partial u}\right|_{u=u_0} -\\
	-  (k-2) c_{k-2} J_{k-2}(u_0) - (k+2) c_{k+2} J_{k+2}(u_0) 
	\biggr)+ o(q).
\end{multline*}
Из равенства нулю коэффициентов при каждой степени $q$ следует
\begin{align*}
&u_0 = \alpha_{k, m}, \\
&u_1 = \frac{1}{\left.\frac{\partial J_{k} (u)}{\partial u}\right|_{u=\alpha_{k, m}}} 
\biggl(
\frac{(k-2)c_{k-2}}{k c_k} J_{k-2}(\alpha_{k, m}) + \frac{(k+2)c_{k+2} }{k c_k} J_{k+2}(\alpha_{k, m})
\biggr),
\end{align*}
где $\alpha_{k, m}$ -- $m$-ый нуль функции $J_k(x)$.  

Из равенства $q=\frac{\varkappa^2 a^2 \delta^2}{4 a^2}=\frac{u^2 \delta^2}{4a^2}$ можно получить выражение для собственного значения
$$\varkappa_{k, m}^2 = \frac{u^2}{a^2} = \frac{u_0^2}{a^2} + \frac{2 q u_0 u_1}{a^2} + o(q)= \frac{u_0^2}{a^2} +  \delta^2 \frac{u_0^3 u_1}{2 a^4} + o(\delta^2).$$ 
Непосредственной подстановкой $u_0, u_1$ в полученную формулу, получаем
\begin{multline*}
\varkappa_{k, m}^2 = 
\frac{\alpha_{k, m}^2}{a^2} +  \delta^2 \frac{\alpha_{k, m}^3}{2 a^4}\frac{1}{\left.\frac{\partial J_{k} (u)}{\partial u}\right|_{u=\alpha_{k, m}}} \times \\ \times
\biggl(
\frac{(k-2)c_{k-2}}{k c_k} J_{k-2}(\alpha_{k, m}) + \frac{(k+2)c_{k+2} }{k c_k} J_{k+2}(\alpha_{k, m})
\biggr) + o(\delta^2).
\end{multline*}
\end{proof}

Вернемся к доказательству утверждения \ref{th:stat4}.
\begin{proof}
\textit{Случай} 1: $\zeta = a_k(q)$.
Для малых $q$ справедливо (см. \cite{wref2}) представление четного решения углового уравнения Матьё $ce_k(\phi, q)$ в виде 
{
\[
ce_k(\phi, q) = 
\left[
\begin{array}{ll}
	\cos{k\phi} + 
	\frac{q}{4(k-1)} \cos{(k-2)\phi} - 
	\frac{q}{4(k+1)} \cos{(k+2)\phi} + o(q), \ \ & k \geq 2;\\
	\cos{\phi} - \frac{q}{8} \cos{3 \phi} + o(q), & k = 1; \\
	\frac{1}{\sqrt{2}} - \frac{1}{\sqrt{2}}\frac{q}{2}\cos{2 \phi} + o(q), & k = 0. 
\end{array}
\right.
\]
}

Пусть $k \geq 2$. Тогда $c_k = 1, c_{k+2}=\frac{-1}{4(k+1)},  c_{k-2}=\frac{1}{4(k-1)}$. Применим лемму \ref{th:lemEllipse1}:
\begin{multline*}
\varkappa_{k, m}^2 = 
\frac{\alpha_{k, m}^2}{a^2} +  \delta^2 \frac{\alpha_{k, m}^3}{2 a^4}\frac{1}{\left.\frac{\partial J_{k} (u)}{\partial u}\right|_{u=\alpha_{k, m}}} \times \\ \times
\biggl(
\frac{1}{4(k-1)} J_{k-2}(\alpha_{k, m}) - \frac{1}{4(k+1)}J_{k+2}(\alpha_{k, m})
\biggr) + o(\delta^2).
\end{multline*}

Пусть $k =1$. Тогда $c_k = 1, c_{k+2}=\frac{-1}{8},  c_{k-2}=0$, и из леммы \ref{th:lemEllipse1} получаем
$$
\varkappa_{1, m}^2 = 
\frac{\alpha_{1, m}^2}{a^2} -  \delta^2 \frac{\alpha_{1, m}^3}{16 a^4}\frac{J_{3}(\alpha_{1, m})}{\left.\frac{\partial J_{1} (u)}{\partial u}\right|_{u=\alpha_{1, m}}}  + o(\delta^2).
$$

Пусть $k =0$. Тогда $c_k = \frac{1}{\sqrt{2}}, c_{k+2}=\frac{-1}{2\sqrt{2}},  c_{k-2}=0$, и по лемме \ref{th:lemEllipse1} имеем
$$
\varkappa_{0, m}^2 = 
\frac{\alpha_{0, m}^2}{a^2} -  \delta^2 \frac{\alpha_{0, m}^3}{4 a^4}\frac{J_{2}(\alpha_{0, m})}{\left.\frac{\partial J_{0} (u)}{\partial u}\right|_{u=\alpha_{0, m}}} 
 + o(\delta^2).
$$

\textit{Случай} 2: $\zeta = b_k(q)$.
Для малых $q$ справедливо (см. \cite{wref2}) представление нечетного решения углового уравнения Матьё $se_k(\phi, q)$ в виде 
{
\[
se_k(\phi, q) = 
\left[
\begin{array}{ll}
	\sin{k\phi} + 
	\frac{q}{4(k-1)} \sin{(k-2)\phi} - 
	\frac{q}{4(k+1)} \sin{(k+2)\phi} + o(q), \ \ & k > 2;\\
	\sin{2\phi} - \frac{q}{12} \sin{4 \phi} + o(q), & k = 2; \\
	\sin{\phi} - \frac{q}{8}\sin{3 \phi} + o(q), & k = 1. 
\end{array}
\right.
\]
}

Пусть $k > 2$. Тогда $c_k = 1, c_{k+2}=\frac{-1}{4(k+1)},  c_{k-2}=\frac{1}{4(k-1)}$. Применим лемму \ref{th:lemEllipse2}:
\begin{multline*}
\varkappa_{k, m}^2 = 
\frac{\alpha_{k, m}^2}{a^2} +  \delta^2 \frac{\alpha_{k, m}^3}{2 a^4}\frac{1}{\left.\frac{\partial J_{k} (u)}{\partial u}\right|_{u=\alpha_{k, m}}} \times \\ \times
\biggl(
\frac{(k-2)}{4k(k-1) } J_{k-2}(\alpha_{k, m}) - \frac{(k+2)}{4k(k+1) } J_{k+2}(\alpha_{k, m})
\biggr) + o(\delta^2).
\end{multline*}

Пусть $k =2$. Тогда $c_k = 1, c_{k+2}=\frac{-1}{12},  c_{k-2}=0$, и из леммы \ref{th:lemEllipse2} получаем
$$
\varkappa_{2, m}^2 = 
\frac{\alpha_{2, m}^2}{a^2} -  \delta^2 \frac{\alpha_{2, m}^3}{12 a^4}\frac{J_{4}(\alpha_{2, m})}{\left.\frac{\partial J_{2} (u)}{\partial u}\right|_{u=\alpha_{2, m}}} 
 + o(\delta^2).
$$

Пусть $k =1$. Тогда $c_k = 1, c_{k+2}=\frac{-1}{8},  c_{k-2}=0$, и по лемме \ref{th:lemEllipse2} имеем
$$
\varkappa_{1, m}^2 = 
\frac{\alpha_{1, m}^2}{a^2} -  \delta^2 \frac{3\alpha_{1, m}^3}{16 a^4}\frac{ J_{3}(\alpha_{1, m})}{\left.\frac{\partial J_{1} (u)}{\partial u}\right|_{u=\alpha_{1, m}}}  + o(\delta^2).
$$
\end{proof}



\subsection{Квантовый биллиард в эллиптическом кольце и накрытии эллиптического кольца}\label{sec:ch1/sec4/sub2}
%\subsection{Рассматриваемые области}\label{sec:ch2/sec3/sub1}

Рассмотрим область (``эллиптическое кольцо''), ограниченную двумя эллипсами с длинными полуосями $0 < r_0 < r_1$ и с общими фокусами в точках $(\pm \delta, 0)$. 
В эллиптических координатах $(\rho, \phi)$ эта область   задается неравенствами $\rho_0 = \text{arccosh} (\frac{r_0}{\delta}) \leq \rho \leq \text{arccosh} (\frac{r_1}{\delta}) = \rho_1, \hspace{5pt} 0 \leq \phi \leq 2 \pi$. 
Для $p$-листного накрытия  $\Omega_\delta$ эллиптического кольца неравенство на угловую координату другое: $0 \leq \phi \leq 2 \pi p$. Для удобства введем $\varkappa^2 = \frac{2 M E}{\hbar^2}$.

\subsubsection{Собственные функции}\label{sec:ch1/sec4/sub2/sub1}
Мы хотим получить решения стационарного уравнения
Шредингера в $p$-листном накрытии  $\Omega_\delta$, а также асимптотику соответствующих уровней энергии при фокусном расстоянии $2\delta$, стремящемся к нулю. %Члены нулевого порядка должны совпадать с  $E_{k,m}$ из формулировки теоремы 1.

%Здесь и далее используется обозначение $W_{a,b}(u) = Y_a(u)J_b(\lambda u) - Y_a(\lambda u)J_b(u)$.


\begin{theorem}
В области $\Omega_\delta$ ($p$-листном накрытии эллиптического кольца) собственные функции $\psi_{k, m}(\rho, \phi)$ и собственные значения $E_{k,m}$ оператора $\hat{H}$ имеют вид
{\small
\begin{equation}
\psi_{k,m}(\rho,\phi) = 
\left[ 
\begin{array}{cll}

\begin{multlined}[t]
    \left.
    \biggl[ Ce_\nu(\rho_0, q) Fey_\nu(\rho, q) - Ce_\nu(\rho, q) Fey_\nu(\rho_0, q) \biggr] ce_\nu(\phi, q)
    \right|_{q=\beta_{\nu, m}}, \\
    E_{k,m}= \frac{\varkappa^2_{k,m}\hbar^2}{2M}, \quad \nu \in \{0\} \cup \mathbb{N} ;
\end{multlined}
\\
\begin{multlined}[t]
    \left.
    \biggl[ Se_{-\nu}(\rho_0, q) Gey_{-\nu}(\rho, q) - Se_{-\nu}(\rho, q) Gey_{-\nu}(\rho_0, q) \biggr] se_{-\nu}(\phi, q)
    \right|_{q=\beta_{\nu, m}}, \\
    E_{k,m}= \frac{\varkappa^2_{k,m}\hbar^2}{2M},  \quad -\nu \in \mathbb{N} ;
\end{multlined}
\\
\begin{multlined}[t]
\biggl[ M_\nu^{(1)}(\rho_0, q) M_\nu^{(2)}(\rho, q) - M_\nu^{(1)}(\rho, q) M_\nu^{(2)}(\rho_0, q) \biggr] \times \\ 
\times \left. (A_1 ce_\nu(\phi, q) + A_2 se_\nu(\phi, q)) \right|_{q=\beta_{\nu, m}}, E_{k,m}= \frac{\varkappa^2_{k,m}\hbar^2}{2M}, \quad \text{иначе},
\end{multlined}
\end{array}
\right.
\label{eq:funcRing}
\end{equation}
где $\nu=\frac{k}{p}$,  
$\varkappa^2_{k,m}=\frac{4 \beta_{\nu, m}}{\delta^2},  k, m \in \mathbb{N}$, 
$\beta_{\nu, m}$ --- $m$-й нуль функции 
\begin{equation}
f(q)= 
\left[ \begin{array}{ccc}
Ce_\nu(\rho_0, q) Fey_\nu(\rho_1, q) - Ce_\nu(\rho_1, q) Fey_\nu(\rho_0, q), & \nu \in \{0\} \cup \mathbb{N}; \\
Se_{-\nu}(\rho_0, q) Gey_{-\nu}(\rho_1, q) - Se_{-\nu}(\rho_1, q) Gey_{-\nu}(\rho_0, q), & -\nu \in \mathbb{N}; \\
M_\nu^{(1)}(\rho_0, q) M_\nu^{(2)}(\rho_1, q) - M_\nu^{(1)}(\rho_1, q) M_\nu^{(2)}(\rho_0, q) & \text{иначе}.
\end{array}
\right.
%\tag{3}
\label{eq:funcF}
\end{equation}
}

\label{th:sect2_th3}
\end{theorem}
\begin{proof}

Будем искать решение уравнения $\frac{-\hbar^2}{2M} \nabla^2\psi = E\psi$  в виде  $\psi(\rho, \phi)=R(\rho)\Phi(\phi)$, где  $\rho$ и $\phi$ --- эллиптические координаты.
Тогда $R$ и $\Phi$ являются решениями уравнений Матьё
\begin{equation}
	\Bigg\{
	\begin{array}{ccc}
		\frac{\partial^2}{\partial \phi^2}\Phi + (\zeta - 2q\cos{2\phi})\Phi & = 0, \\
		\frac{\partial^2}{\partial \rho^2}R - (\zeta - 2q\cosh{2\rho})R &= 0,
	\end{array}
\label{eq:ma}
\end{equation}
где
$q=\frac{(\varkappa \delta)^2}{4}$ и 
$\zeta$ --- разделяющая переменная. Для начала рассмотрим угловое уравнение Матьё и определим, при каких $\zeta$ условие $\Phi(0)=\Phi(2\pi p)$ выполнено.

По теореме Флоке для некоторого $\nu$ существует решение $\Phi_\nu(\phi)$ уравнения Матьё, такое, что $\Phi_\nu(\phi+2\pi p) =e^{2 i \pi p \nu}\Phi_\nu(\phi)$. В случае $p$-листного накрытия необходимо наложить условие периодичности $\Phi_\nu(0) = \Phi_\nu(2 \pi p)$. Следовательно, $e^{2 i \pi p \nu} = 1$. Откуда $ p \nu = k \in \mathbb{Z}$ и, таким образом, $ \nu = \frac{k}{p}$, где $k \in \mathbb{Z}$. 
Положим $\Phi(\phi) = \Phi_\nu(\phi)$.

Обозначим через $R_1(\rho, q), R_2(\rho, q)$ два независимых решения радиального уравнения Матьё~(\ref{eq:ma}), которые зависят от параметра $q$. Решением \eqref{eq:ma} является и их линейная комбинация $R(\rho, q) = A R_1(\rho, q) + B R_2(\rho, q)$. Из условия $R(\rho_0, q)=0$ установим значения констант: $A=R_2(\rho_0, q), B=-R_1(\rho_0, q)$ (либо значения, пропорциональные им). 
Теперь рассмотрим функцию $f(q) = R_2(\rho_0, q) R_1(\rho_1, q) -R_1(\rho_0, q) R_2(\rho_1, q)$, в зависимости от значения $\nu$ это одна из функций \eqref{eq:funcF}. Тогда условие $R(\rho_1, q)=0$ можно записать как $f(q)=0$. Обозначим $m$-й положительный нуль этой функции через $\beta_{\nu, m}$, тогда $q=\beta_{\nu, m}$ для какого-то значения $m$, откуда следует, что $\varkappa^2 = \frac{4q}{\delta^2}$ может принимать только значения $\varkappa^2_{k,m}$, приведенные в формулировке теоремы \ref{th:sect2_th3}.

В завершение доказательства остается только привести явный вид функций $\Phi(\phi), R(\rho)$. В зависимости от значения $\nu=\frac{k}{p}$ разделяющий параметр $\zeta$ в системе дифференциальных уравнений~(\ref{eq:ma}) относится к одному из трех типов:
\[
\zeta = \left[
\begin{array}{ll}
	a_\nu(q), 					& \nu \in \{0\} \cup \mathbb{N}; \\
    b_{-\nu}(q), 	\ \ \ \			& -\nu \in \mathbb{N}; \\
	\lambda_\nu(q) 	        & \text{иначе}.
\end{array}
\right.
\]
Периодическими угловыми решениями в первых двух случаях являются функции, описанные в табл.~\ref{tab:table1}, они и будут функциями $\Phi(\phi)$ в зависимости от значения $\nu$.

Радиальные функции получаются в виде линейных комбинаций радиальных функций Матьё целого порядка. В качестве $R_1(\rho, q)$ возьмем радиальные функции Матьё первого рода~\ref{eq:mathieuRadial}: $Ce_\nu(\rho, q)$ при $\nu \in \{0\} \cup \mathbb{N}$ и $Se_\nu(\rho, q)$ при $-\nu \in \mathbb{N}$. Независимыми решениями $R_2(\rho, q)$ для этих двух случаев являются $Fey_\nu(\rho, q)$ и $Gey_\nu(\rho, q)$ соответственно.

В случае $\nu = \frac{m_1}{m_2} \in \mathbb{Q} \setminus \mathbb{Z}$ обе угловые функции $ce_\nu(\phi, q), se_\nu(\phi, q)$  являются периодическими и имеют период не более $2\pi m_2$ (см. \cite{wref2}), поэтому в качестве $\Phi(\phi)$ подходит в том числе их линейная комбинация. Решение радиального уравнения Матьё представляется в виде линейной комбинацией функций $Ce_\nu(\rho, q)$ и $Se_\nu(\rho, q)$. Однако в следующей теореме  будет удобнее использовать линейную комбинацию функций $M_\nu^{(1)}(\rho, q), M_\nu^{(2)}(\rho, q)$~(см. \cite[\S\ 28.23]{wref5}; \cite[гл.~2, \S\ 2.4, с.~165]{wref12}), также образующих фундаментальную систему.
\end{proof}
\FloatBarrier
