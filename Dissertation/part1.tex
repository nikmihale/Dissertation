\chapter{Модельные задачи}\label{ch:ch1}

\section{Предварительные сведения}\label{sec:ch1/sec1}
Пусть $\Omega$ --- область, ограниченная координатными линиями в полярной системе координат $(x, y) = (r \cos \phi, r \sin \phi)$. В $\Omega$ рассмотрим стационарное уравнение Шредингера.
\begin{equation*}
\hat{H}\psi = \left( \frac{-\hbar^2}{2M}\nabla^2 + V(r)\right) \psi = E\psi,
%\label{eq:schrodinger}
\end{equation*}
где потенциал $V(r)$ внутри области $\Omega$ равен нулю, а вне ее обращается в бесконечность. Такая задача равносильна поиску собственных функций и собственных значений оператора Лапласа в области $\Omega$ для функций, обращающихся в нуль на границе $\Omega$. Заметим, что при обозначении $\varkappa^2 = \frac{2 M E}{\hbar^2}$ уравнение Шредингера можно записать в виде $\nabla^2  = -\varkappa^2 \psi.$

\subsection{Стационарное уравнение Шредингера в полярных координатах}\label{subsec:ch1/sec1/sub1}
В полярной системе координат $(x, y) = (r\cos \phi, r\sin \phi)$ оператор Гамильтона внутри области $\Omega$ имеет вид:
\begin{equation*}
\hat{H} = \frac{-\hbar^2}{2M} \nabla^2  + V(x, y) = 
\frac{-\hbar^2}{2M} \left(\frac{1}{r^2}\frac{\partial^2}{\partial \phi^2} + \frac{1}{r}\frac{\partial}{\partial r} + \frac{\partial^2}{\partial r^2}\right) 
%\label{eq:polarSchrodinger}
\end{equation*}
Запишем функцию $\psi$ в виде $\psi(r, \phi) = R(r)\Phi(\phi)$, и пусть $E$ -- ее собственное значение. Тогда положим $\varkappa^2 = \frac{2M E}{\hbar^2}$, и уравнение $(\nabla^2  + \varkappa^2) \psi=0$ примет вид 
$$ \frac{R}{r^2}\frac{\partial^2\Phi}{\partial\phi^2} + \frac{\Phi}{r}\frac{\partial R}{\partial r} + \Phi  +\varkappa^2R\Phi = 0.$$
Обе части уравнения умножим на $\frac{r^2}{R\Phi}$:
$$ \frac{1}{\Phi}\frac{\partial^2\Phi}{\partial\phi^2} + \frac{r}{R}\frac{\partial R}{\partial r} + \frac{r^2}{R}\frac{\partial^2 R}{\partial r^2} + \varkappa^2r^2 = 0.$$
Введем разделяющий параметр $\lambda$, что позволяет рассмотреть исходное уравнение на $\psi(r, \phi) = R(r)\Phi(\phi)$, как систему из двух обыкновенных дифференциальных уравнений:
\begin{equation*}
\left\{\begin{array}{rcll}
    \dfrac{\Phi''}{\Phi} &=&-\lambda^2, \\
    \dfrac{rR'}{R} + \dfrac{r^2R''}{R} + \varkappa^2r^2 &=& \lambda^2  \quad 	 &\textit{   (уравнение Бесселя)}. \\
\end{array}
\right.
%\label{eq:polarSystem}
\end{equation*}
решениями последнего уравнения являются функции Бесселя первого и второго рода $J_\lambda(\varkappa r)$ и $Y_\lambda(\varkappa r)$, соответственно. 


\subsection{Функции Бесселя}\label{subsec:ch1/sec1/sub2}
Функция Бесселя первого рода $J_\alpha(x)$ является решением дифференциального уравнения второго порядка
$$x^2\frac{d^2y}{dx^2} + x\frac{dy}{dx} + (x^2 - \alpha^2)y=0,$$
где параметр $\alpha$ выбирается так, что разложение $J_\alpha(x)$ в точке $x=0$ имеет вид
$$J_\alpha(x) = \sum_{r=0}^\infty \frac{(-1)^r (\frac{x}{2})^{\alpha + 2r}}{r! \Gamma(\alpha+r+1)}.$$
Главная ветвь $J_\alpha(x)$ соответствует главному значению функции $(\frac{x}{2})^{\alpha + 2r}$ и является аналитической функцией в комплексной плоскости с вырезом вдоль интервала $(-\infty, 0]$. 
При $\alpha \in \mathbb{Z}$ функция $J_\alpha(x)$ является целой для $x \in \mathbb{C}$, причем $J_{-n}(x) = (-1)^nJ_n(x)$. Если $\alpha \notin \mathbb{Z}$, то функции $J_\alpha$ и $J_{-\alpha}$ линейно независимы.
Если $\alpha \in \mathbb{R}$, то функции Бесселя $J_\alpha(x)$ имеют счетное число положительных вещественных нулей $\{ j_{\alpha, k} \}_{k=1}^\infty$. Традиционно их нумеруют по возрастанию 
$$j_{\alpha, 1} < j_{\alpha, 2} < j_{\alpha, 3} < ...$$
Для функций Бесселя справедливы связующие соотношения:
$$Y_\alpha(x) = \frac{J_\alpha(x) \cos \pi \alpha - J_{-\alpha}(x)}{\sin \pi \alpha}$$ и рекуррентные соотношения, справедливые для $F=J$ и $F=Y$.
$$ \frac{2\alpha}{x}F_\alpha(x) = F_{\alpha-1}(x)+F_{\alpha+1}(x), \quad 
2\frac{d F_\alpha(x)}{d x} = F_{\alpha-1}(x)-F_{\alpha+1}(x).$$ 
Также существуют (см. \cite[\S~10.19]{wref5} асимптотики функций Бесселя для больших аргументов и для больших значений порядка $\alpha$:
$$J_\alpha(x) \approx \sqrt{2 \over \pi x}\left( \cos(x - \frac{\alpha \pi}{2} - \frac{\pi}{4} \right), \quad J_\alpha(x) \approx \sqrt{1 \over 2 \pi \alpha}\left(e x  \over 2 \alpha\right)^\alpha.$$

Для нулей $j_{\alpha, m}, \quad y_{\alpha, m}$ функций Бесселя первого и второго рода выведена~\cite{wref3} их асимптотика при $m \to \infty$:
$$j_{\alpha, m}, y_{\alpha, m} \approx a - {\mu - 1 \over 8 a} - {4(\mu - 1)(7\mu - 31) \over 3(8a)^3} - {32(\mu - 1)(83\mu^2 - 982\mu + 3779 \over 15(8a)^5}+\dots,$$
где $\mu = 4 \alpha^2$, $a=(m+{1\over2}\alpha - {1\over4})\pi$ для $j_{\alpha, m}$ или $a=(m+{1\over2}\alpha - {3\over4})\pi$ для $y_{\alpha, m}$.
Для произведений функций Бесселя вида $Y_\nu(x) J_\nu(\lambda x) - Y_\nu(\lambda x) J_\nu(x)$ асимптотика $m$-го положительного нуля $\alpha_{\nu, m}$ также известна  \cite[\S\ 9, с.~358]{wref2}:
 $$\alpha_{\nu, m} = \sigma + \frac{\chi}{\sigma} + \frac{\omega-\chi^2}{\sigma^3} + \frac{\eta-4\chi \omega +2\chi^3}{\sigma^5} + \dots,$$
 где $\mu=4\nu^2$,
$\sigma=\frac{\pi m}{\lambda - 1}$,
 $\chi = \frac{\mu - 1}{8 \lambda}$,
 $\omega = \frac{(\mu-1)(\mu-25)(\lambda^3 - 1)}{6(4\lambda)^3(\lambda-1)}$,
 $\eta = \frac{(\mu-1)(\mu^2-114\mu+1073)(\lambda^5-1)}{5(4\lambda)^5(\lambda-1)}$.

\section{Квантовый биллиард в круге}\label{sec:ch1/sec2}

Рассмотрим квантовую частицу, свободно перемещающуюся в круге $D^2$ радиуса $\rho_0$. Для этой системы гамильтониан в декартовых координатах записывается следующим образом:
$$\hat{H} = \frac{\hat{p}^2}{2M} + V(x, y) = \frac{-\hbar^2}{2M}\left(\frac{\partial^2 }{\partial x^2} + \frac{\partial^2}{\partial y^2}\right)  + V(x, y) = \frac{-\hbar^2}{2M} \nabla^2  + V(x, y), $$
где $\rho_0$ - радиус круга $D^2$, и потенциал $V(x,y)$ определяется как
\[
    V(x, y) = 
    \Bigg\{
    \begin{array}{cc}
        0, \sqrt{x^2+y^2} < \rho_0 \\
        \infty, \sqrt{x^2+y^2} \geq \rho_0. \\
    \end{array}
\] 
Задача равносильна поиску собственных функций и собственных значений оператора Лапласа в области $\Omega$ с условием, что  функции обращаются в нуль на границе $\Omega$.
\begin{statement}
В области $D^2$ собственные функции $\psi_{k, m}(r, \phi)$ и собственные значения $E_{k,m}$ оператора $\hat{H}$ имеют вид 
$$\psi_{k, m}(r, \phi) = J_k\left(\frac{\alpha_{k, m}r}{\rho_0}\right)\sin{k \phi}, \hspace{15pt} E_{k,m} = \frac{\hbar^2 \alpha_{k, m}^2}{2M\rho_0^2}, \hspace{15pt} k, m \in \mathbb{Z},$$
где $\alpha_{k, m}$ - $m$-тый ноль функции Бесселя первого рода $J_k(x)$.
\label{st:sec1_stat1}
\end{statement}
\begin{remark}
Вместо $\sin$ можно использовать $\cos$.
\end{remark}
\begin{proof}
Для $r < \rho_0$ получаем дифференциальное уравнение второго порядка в частных производных $(\nabla^2 + \varkappa^2)\psi = 0$, где $\varkappa^2 = \frac{2M E}{\hbar^2}$. Запишем искомую функцию в виде $\psi(r, \phi) = R(r)\Phi(\phi)$. Тогда, в силу соображений из подраздела \ref{subsec:ch1/sec1/sub1}, $\Phi(\phi) = A\cos{k\phi} + B\sin{k\phi}$ и $R(r) = J_k(\varkappa r)$, где $J_k(x)$ - функция Бесселя первого рода. Таким образом, $\psi(r, \phi) = J_k(\varkappa r) (\widetilde{A}\cos{k\phi} + \widetilde{B}\sin{k\phi} )$.

Из граничного условия заметим, что $$\psi(r, \phi) |_{\partial D^2} = 0 \implies J_k(\varkappa r)|_{r=\rho_0} = 0 \implies \varkappa \rho_0 \in \{\alpha: J_k(\alpha) = 0\},$$
откуда допустимые значения $\varkappa \in \{ \frac{\alpha}{\rho_0} : J_k(\alpha)=0 \}$. Значения $k$ должны быть такими, чтоы выполнялось равенство $\Phi(\phi) = \Phi(\phi+2\pi)$, справедливое при условии $k \in \mathbb{Z}$. Учитывая также, что $\varkappa^2 = \frac{2M E}{\hbar^2}$, допустимыми значениями $E$ являются
$$E = \frac{\hbar^2\varkappa^2}{2M} \in \left\{ \frac{\hbar^2\alpha^2}{2M\rho_0^2}: J_k(\alpha)=0 \right\}, k \in \mathbb{Z}.$$

Тогда, выбирая из множества $\alpha$ один конкретный $\alpha_{k, m}$ для некоторого $m \in \mathbb{N}$, получим вид собственных функций системы (с точностью до умножения на константу) 
$$\psi_{k, m}(r, \phi) = J_k\left(\frac{\alpha_{k, m}r}{\rho_0}\right) \sin{k\phi},$$
и их собственными значениями будут
$$E_{k, m} = \frac{\hbar^2\alpha_{k, m}^2}{2M\rho_0^2}, \hspace{15pt} k, m \in \mathbb{Z},$$
где $\alpha_{k, m}$ - $m$-ый ноль функции $J_k(x)$.
\end{proof}

\section{Квантовый биллиард в круговом кольце и накрытии кругового кольца}\label{sec:ch1/sec3}
Пусть  $\Omega$ --- область, $p$-листно накрывающая кольцо, 
ограниченное двумя концентрическими окружностями радиусов $0 < r_0 < r_1$. Случай $p=1$ относится к классической теории колебаний (см. \cite{wref11}). Будем считать, что обе окружности имеют центр в начале координат.
В  области $\Omega$ удобно рассматривать аналог полярных координат --- расстояние $r$ до начала координат и угол $\phi$, определенный  $\mod 2\pi p$. 
В $\Omega$ рассмотрим стационарное уравнение Шрёдингера
$$\hat{H}\psi = \left(\frac{-\hbar^2}{2M} \nabla^2  + V(r)\right)\psi = E\psi,$$здесь потенциал $V(r)$ внутри области $\Omega$ равен нулю, а вне ее обращается в бесконечность. Задача равносильна поиску собственных функций и собственных значений оператора Лапласа в области $\Omega$ для функций, обращающихся в нуль на границе $\Omega$.
Положим  $\varkappa^2 = \frac{2 M E}{\hbar^2}$. Далее $J_\nu$  и $Y_\nu$ --- функции Бесселя первого и  второго рода соответственно.

\begin{theorem}{\normalfont (для $p=1$ см. \cite[с.~165.]{wref10})}.

В  области $\Omega$ ($p$-листном накрытии кругового кольца)  собственные функции $\psi_{k,m}(r,\phi)$ и собственные значения $E_{k,m}$ оператора $\hat{H}$ имеют вид
\begin{multline*}
\psi_{k,m}(r,\phi) = \biggl[ Y_\nu(\alpha_{\nu, m}) J_\nu\biggl(\frac{\alpha_{\nu, m}r}{r_0}\biggr) - Y_\nu\biggl(\frac{\alpha_{\nu, m}r}{r_0}\biggr) J_\nu(\alpha_{\nu, m}) \biggr] \cos{(\nu \phi+\phi_0)}, \\
\quad E_{k,m}= \frac{\varkappa^2_{k,m}\hbar^2}{2M},
\end{multline*}
где 
$\nu=\frac{k}{p}$,  $\lambda = \frac{r_1}{r_0}$,
$\varkappa^2_{k,m}=\frac{\alpha_{\nu, m}^2}{r_0^2},  k, m \in \mathbb{N}$, $\alpha_{\nu, m}$ --- $m$-ый нуль функции $f(x) = Y_\nu(x) J_\nu(\lambda x) - Y_\nu(\lambda x) J_\nu(x)$.
\label{th:sect1_theorem1}
\end{theorem}
\begin{proof}
Запишем искомую функцию в виде $\psi(r, \phi) = R(r)\Phi(\phi)$, тогда уравнение $(\nabla^2 + \varkappa^2)\psi = 0$ приобретет вид
$ \frac{R(r)\Phi''(\phi)}{r^2} + \frac{R'(r)\Phi(\phi)}{r} + \Phi(\phi) R''(r) +\varkappa^2R(r)\Phi(\phi) = 0$. Умножим обе части уравнения на $\frac{r^2}{R(r)\Phi(\phi)}$:
$$\frac{\Phi''(\phi)}{\Phi(\phi)} + \frac{rR'(r)}{R(r)} + \frac{r^2R''(r)}{R(r)} + \varkappa^2r^2 = 0.$$

Введем разделяющий параметр  $\nu$ и получим два уравнения (далее переменные явно не указываем, подразумевая, что $\Phi = \Phi(\phi), R=R(r)$):
\[
\left\{\begin{array}{cc}
    \dfrac{\Phi''}{\Phi} =-\nu^2, \\[10pt]
    \dfrac{rR'}{R} + \dfrac{r^2R''}{R} + \varkappa^2r^2 = \nu^2.
\end{array}
\right.
\]

Решением углового уравнения является функция $\Phi(\phi)  = \cos{(\nu \phi + \phi_0)}$ для некоторого вещественного значения $\phi_0$. Из условия периодичности $\Phi(0) = \Phi(2\pi p)$ следует, что 
$\nu = \frac{k}{p}$, где $k$ --- произвольное неотрицательное целое число. 

Решение радиального уравнения ищется в виде линейной  комбинации функций Бесселя первого и второго рода \cite[\S\ 9, с.~358]{wref2}:
$$R(r) = A J_\nu(\varkappa r) +B Y_\nu(\varkappa r).$$
Из граничного условия $R(r_0) = 0$ установим значения констант: $A = Y_\nu(\varkappa r_0), B = -J_\nu(\varkappa r_0)$ (либо пропорциональные им). 

Теперь рассмотрим функцию  $f(x)=Y_\nu(x) J_\nu(\lambda x) - Y_\nu(\lambda x) J_\nu(x)$, где $\lambda = \frac{r_1}{r_0}$.
Тогда граничное условие $R(r_1) = Y_\nu(\varkappa r_0) J_\nu(\varkappa r_1) -J_\nu(\varkappa r_0) Y_\nu(\varkappa r_1) =0$ можно записать в виде
$f(\varkappa r_0)=0$.
Обозначим $m$-й положительный нуль этой функции через $\alpha_{\nu, m}$. Тогда  $\varkappa r_0 = \alpha_{\nu, m} $ для какого-то значения $m$,
откуда следует, что $\varkappa$ может принимать только значения $\varkappa^2_{k,m}$, приведенные в формулировке теоремы \ref{th:sect1_theorem1}.
\end{proof}

\section{Квантовый биллиард в секторе круга $D^2: \phi \in (0, \theta)$}\label{sec:markup}
Зафиксируем угол $\theta \in (0, 2\pi)$ и в полярных координатах $(r, \phi)$ рассмотрим область 
$S^2 = \{r, \phi : 0 \leq r \leq \rho_0 , 0 \leq \phi \leq \theta\}$.
\begin{statement} \cite[p.~4]{wref13}
В области $S^2$ собственные функции $\psi_{k, m}(r, \phi)$ и собственные значения $E_{k, m}$ оператора $\hat{H}$ имеют вид 
$$\psi_{k, m}(r, \phi) = J_\lambda\left(\frac{\alpha_{\lambda, m}r}{\rho_0}\right)\sin{\lambda\phi}, \hspace{15pt} E_{k, m} = \frac{\hbar^2 \alpha_{\lambda, m}^2}{2M\rho_0^2}, \hspace{15pt} k, m \in \mathbb{Z},$$
где $\lambda = \frac{\pi k}{\theta}$, $\alpha_{\lambda, m}$ - $m$-ый ноль функции Бесселя первого рода $J_\lambda(x)$.
\label{st:sect1_stat1}
\end{statement}
\begin{proof}
Ищем решение в виде $\psi(r, \phi) = R(r)\Phi(\phi)$. Повторяя соображения из предыдущего пункта, получим систему дифференциальных уравнений для биллиарда в круге 
\[
\left\{\begin{array}{cc}
    \dfrac{\Phi''}{\Phi} =-\lambda^2, \\
    \dfrac{rR'}{R} + \dfrac{r^2R''}{R} + \varkappa^2r^2 = \lambda^2,
\end{array}
\right.
\]
решение которой должно удовлетворять новому граничному условию $\Phi(0)=\Phi(\theta)=0$. Поэтому $\Phi(\theta) = \sin(\lambda \theta) \in \{\pi k, k \in \mathbb{Z}\} \implies \lambda = \frac{\pi k}{\theta}, k \in \mathbb{Z}$.

В силу соображений из предыдущего пункта, решение системы дифференциальных уравнений имеет вид:
$$\psi_{k, m}(r, \phi) = J_\lambda\left(\frac{\alpha_{\lambda, m}r}{\rho_0}\right)\sin{\lambda\phi} , \quad \lambda=\frac{\pi k}{\theta}.$$
Аналогичными рассуждениями можно обнаружить, что собственные значения в этом случае принимают вид 
$E_{k, m} = \frac{\hbar^2 \alpha_{\lambda, m}^2}{2M\rho_0^2}$ для $\lambda=\frac{\pi k}{\theta}.$
\end{proof}

\begin{consequence} \cite[p.~4]{wref13}
Нетрудно заметить, что индексы $\lambda_m = \frac{m \pi}{\theta}$, соответствующие сектору $0 \leq \phi \leq \theta$ связаны с индексами $\mu_n = \frac{n \pi}{2\pi - \theta}$ ($\mu_n$ соответствуют решениям в дополнении сектора, т.е. для $\theta \leq \phi \leq 2\pi$) следующим образом:
$$\lambda_m = \frac{\mu_n}{2\mu_n - n}m, \hspace{15pt} \mu_n = \frac{\lambda_m}{2\lambda_m - m}n$$
\end{consequence}
\begin{proof}
Действительно, имеем 
$\frac{n \pi}{2\pi - \theta} = \mu_n$, $\frac{m \pi}{\theta} = \lambda_m$, тогда из равенства $2\pi = \theta + (2\pi - \theta) = \frac{m \pi}{\lambda_m} + \frac{n \pi}{\mu_n}$ получаются равенства из утверждения \ref{st:sect1_stat1}.
\end{proof}
\FloatBarrier
