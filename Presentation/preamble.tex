\begin{frame}[noframenumbering,plain]
    \setcounter{framenumber}{1}
    \maketitle
\end{frame}

%\begin{frame}
%    \frametitle{Положения, выносимые на защиту}
%\begin{itemize}[beginpenalty=10000] % https://tex.stackexchange.com/a/476052/104425
%  \item Доказана интегрируемость классического бильярда на софокусных столах, разделенных софокусными квадриками на области, заполненные изотропными средами с различными показателями `оптической плотности' при условии, что на границе раздела сред выполняется косинусный закон преломления.
% \item Исследованы и описаны слоения изоэнергетического многообразия на поверхности постоянного уровня дополнительного интеграла для двух `оптических систем' с косинусным законом преломления, т.е. для софокусных столов, разделенных софокусными квадриками на области, заполненные изотропными средами с различными показателями `оптической плотности'  при условии, что на границе раздела сред выполняется косинусный закон преломления.
%   \item Исследовано асимптотическое поведение уровней энергии свободной квантовой частицы в двух областях, ограниченных дугами гипербол и внешнего эллипса, которые при стремлении фокального расстояния к нулю принимают форму кругового сектора при стремлении фокального расстояния к нулю, и вычислен спектр наблюдаемой, соответствующей дополнительному интегралу классической системы.
%\end{itemize}
%\end{frame}
%\note{
%    Проговариваются вслух положения, выносимые на защиту
%}
%
%\begin{frame}
%    \frametitle{Содержание}
%    \tableofcontents
%\end{frame}
%\note{
%    Работа состоит из четырёх глав.
%
%    \medskip
%    В первой главе \dots
%
%    Во второй главе \dots
%
%    Третья глава посвящена \dots
%
%    В четвёртой главе \dots
%}
