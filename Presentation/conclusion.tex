%\begin{frame}
%    \frametitle{Научная новизна}
%    \begin{itemize}
%        \item Впервые реализован \dots
%        \item Разработана программа \dots
%        \item Впервые проведён анализ \dots
%        \item Предложена схема \dots
%    \end{itemize}
%\end{frame}
%\note{
%    Проговаривается вслух научная новизна
%}
%
%\begin{frame}
%    \frametitle{Научная и практическая значимость}
%    \begin{itemize}
%        \item Получены выражения для \dots.
%        \item Определены условия \dots.
%        \item Разработаны устройства \dots.
%    \end{itemize}
%\end{frame}
%\note{
%    Проговариваются вслух научная и практическая значимость
%}

\begin{frame}[t,allowframebreaks]
    \frametitle{Положения, выносимые на защиту}
\begin{itemize}
 \item Доказана интегрируемость классического бильярда на софокусных столах, разделенных софокусными квадриками на области, заполненные изотропными средами с различными показателями `оптической плотности' при условии, что на границе раздела сред выполняется косинусный закон преломления.
 \item Исследованы и описаны слоения изоэнергетического многообразия на поверхности постоянного уровня дополнительного интеграла для двух `оптических систем' с косинусным законом преломления, т.е. для софокусных столов, разделенных софокусными квадриками на области, заполненные изотропными средами с различными показателями `оптической плотности'  при условии, что на границе раздела сред выполняется косинусный закон преломления.
   \item Исследовано асимптотическое поведение уровней энергии свободной квантовой частицы в двух областях, ограниченных дугами гипербол и внешнего эллипса, которые при стремлении фокального расстояния к нулю принимают форму кругового сектора при стремлении фокального расстояния к нулю, и вычислен спектр наблюдаемой, соответствующей дополнительному интегралу классической системы.
\end{itemize}
\end{frame}

\begin{frame} % публикации на одной странице
% \begin{frame}[t,allowframebreaks] % публикации на нескольких страницах
    \frametitle{Основные публикации}
     \nocite{nikulin2023spektr611484954}%
    \nocite{nikulin2024asymptotic617844539}%
    \nocite{vestnikLatest}%
    \nocite{sbornikLatest}%


    \ifnumequal{\value{bibliosel}}{0}{
        \insertbiblioauthor
    }{
        \printbibliography%
%	\insertbiblioauthor
    }
\end{frame}
\note{
    Результаты работы опубликованы в N печатных изданиях,
    в~т.\:ч. M реферируемых изданиях.
}

\begin{frame}[t,allowframebreaks]
    \frametitle{Участие в конференциях}
    \begin{itemize}%[beginpenalty=10000]
\item Студенческая школа-конференция <<Математическая весна --  2023>>, Нижний Новгород, Россия, 27-30 марта 2023;
\item Ломоносовские чтения 2023, Россия, 4-14 апреля 2023;
\item \rom{30} Международная научная конференция студентов, аспирантов и молодых учёных <<Ломоносов-2023>>,  Москва, Россия, 10-21 апреля 2023;
\item Воронежская зимняя математическая школа С.Г. Крейна, Воронеж, Россия, 26-30 января 2024;
\item Студенческая школа-конференция <<Математическая весна -- 2024>>, Нижний Новгород, Россия, 25-28 марта 2024;
\item \rom{31} Международная конференция студентов, аспирантов и молодых ученых <<Ломоносов--2024>>, Москва, Россия, 12-26 апреля 2024;
%\item Современные геометрические и топологические методы, Сириус, Россия, 14-19 мая 2024;	%там не по диссеру был доклад
\item Семинар “Дифференциальная геометрия и приложения” под руководством акад. А.Т. Фоменко на механико-математическом факультете МГУ имени М.В. Ломоносова, 25 ноября 2024;
\item \rom{32} Международная научная конференция студентов, аспирантов и молодых ученых <<Ломоносов--2025>>, Москва, Россия, 11-25 апреля 2025.
\item \rom{5} Конференция математических центров России, Красноярск, 11-16 августа 2025
\end{itemize}

\end{frame}
\note{
    Работа была представлена на ряде конференций.
}

\begin{frame}[plain, noframenumbering] % последний слайд без оформления
    \begin{center}
        \Huge
        Спасибо за внимание!
    \end{center}
\end{frame}
