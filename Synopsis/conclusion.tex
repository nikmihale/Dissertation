\chapter*{Заключение}                   
В диссертации были получены новые важные результаты о квантовых и модифицированных классических бильярдах на софокусных столах.

Автором исследовано асимптотическое поведение уровней энергии свободной квантовой частицы в двух областях, ограниченных дугами гипербол и внешнего эллипса, которые при стремлении фокального расстояния к нулю принимают форму кругового сектора. Полученные аналитические выражения справедливы при стремлении фокального расстояния к нулю с точностью до второго порядка включительно, при этом коэффициент при нулевом порядке совпадает с энергетическим спектром свободной частицы в круговом секторе.
Заметим, что предложенная в диссертации техника может быть продолжена на более высокие порядки. 
Ожидается, что для коэффициентов таким образом могут быть получены аналитические выражения в терминах специальных функций.
Автором также вычислен спектр наблюдаемой, соответствующей дополнительному интегралу классической системы.

Также в работе доказана интегрируемость классического бильярда на софокусных столах, разделенных софокусными квадриками на области, заполненные изотропными средами с различными показателями `оптической плотности' в предположении, что на границе раздела сред выполняется косинусный закон преломления.
Приведена явная формула дополнительного интеграла. Отметим, что в зависимости от параметров `оптических плотностей' изотропных сред и параметров разделяющих их квадрик, дополнительный интеграл может принимать значения не только в вещественной прямой, но и, к примеру, в окружности. В общем случае в каждой точке пересечения софокусных квадрик дополнительный интеграл имеет по одной точке ветвления.

Автором исследованы и описаны слоения изоэнергетического многообразия на поверхности постоянного уровня дополнительного интеграла для двух таких систем. 
А именно, в четвертой и пятой главах рассмотрены два показательных примера: в одном случае дополнительный интеграл не имеет ветвления, а во втором --- имеет одну точку ветвления. В работе приведено описание  поверхностей регулярного значения дополнительного интеграла, а также их бифуркации для обоих примеров. Оказалось, что в рассматриваемых системах помимо торов возникают также поверхности более высокого рода с проколами. 
Получено описание поверхностей особых поверхностей, соответствующих одновременным разным бифуркациям в разных областях бильярдного стола. 
Приведена техника построения бифуркационных диаграмм, которые  одновременно учитывают все возможные значения 'оптических параметров` областей, в том числе для случая интеграла, имеющего одну точку ветвления.

Результаты диссертации могут быть полезны специалистам по теории интегрируемых систем, классической механике и математической физике.