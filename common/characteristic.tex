
{\actuality} 
Диссертация посвящена исследованию квантовых и классических бильярдов в областях, ограниченных софокусными квадриками.

Первая тема, которую мы рассмотрим, --- это асимптотика собственных значений оператора Шрёдингера для свободной квантовой частицы. 

В последние годы был достигнут значительный прогресс в понимании гипотезы Биркгофа об интегрируемых плоских бильярдах, например, см.~\cite{bm2022}.
С другой стороны, были найдены строгие доказательства неинтегрируемости некоторых бильярдов, таких как эллиптические бильярды в сильном магнитном поле~\cite{bm2020, bm2019}. Неинтегрируемые бильярды, как и бильярды в магнитном поле, исследуются давно, см.~\cite{berry1985, berry1986}.
Изменение бильярдной области с эллипса на, например, стадион, немедленно приводит к хаотической динамике, \cite{bunimovich1974, stockmann2000}.

Хорошо известно, что классический бильярд в ограниченной софокусными квадриками области интегрируем. Интегрируемость  этой системы следует из того, что помимо сохранения полной энергии существует другая постоянная величина, а именно произведение угловых моментов относительно общих фокусов граничных квадрик.
В случае, если фокусы совпадают, степень этого дополнительного интеграла можно понизить, поскольку угловой момент относительно центра круговой области сохраняется.
Однако, в работе \cite{wref13} было показано, что для бильярда в круговом секторе сохраняемой величиной является не  угловой момент, а квадрат углового момента относительно вершины угла.


Недавние работы А.\,Т.~Фоменко и В.\,В.~Ведюшкиной (см. \cite{wref6,wref7,wref8} , а также другие публикации этих авторов) вновь привлекли к этой теме внимание специалистов. 
В частности, в работе \cite{wref6} изучались особенности бильярда в кольце, ограниченном софокусными эллипсами (`эллиптическое кольцо'). 
В работе \cite{Fok15} в качестве плоских биллиардов рассматриваются также области (`эллиптический сектор'), ограниченные эллипсом с большой полуосью $0 <r_0$ с фокальным расстоянием $\delta$ и 
одной ветвью софокусной гиперболы  (область $A_\delta$), обозначенная как  $A_1$ в \cite{Fok15},
или двумя ветвями софокусных гипербол (область $B_\delta$), которая обозначалась как  $A_0'$ в \cite{Fok15},
Все упомянутые эллипсы и гиперболы имеют общие фокусы в точках $(\pm \delta, 0)$.

В настоящей работе рассматривается соответствующая квантовая система, а именно изучается спектр оператора Шрёдингера в этой области и в ее накрытиях. 
При стремлении $\delta$ к $0$, области $A_\delta$ и $B_\delta$  стремятся к круговому сектору с некоторым центральным углом. Аналогично, при стремлении $\delta$ к $0$ эллиптическое кольцо стремится к круговому кольцу. 
Возникает естественный вопрос об установлении асимптотики уровней энергии при $\delta \to 0$. 
Для приведенных типов областей получены точные решения стационарного уравнения Шрёдингера, а именно собственные функции и соответствующие уровни энергии. Наше исследование направлено на вычисление асимптотики последних при устремлении фокального расстояния $\delta$ к нулю в областях $A_\delta$, $B_\delta$ и в эллиптическом кольце.

Вторая тема, которую мы рассмотрим, --- это развитие теории математических бильярдов.
Одним из возможных направлений являются рассмотрение бильярда с неплоской метрикой. Здесь мы упомянем интегрируемые биллиарды на софокусных столах на плоскости с метрикой Минковского для свободной частицы, рассмотренные В. Драговичем, М. Раднович [127] и Е.Е.Каргиновой [128, 129]. В присутствии центрального потенциала типа Гука бильярд на софокусном столе также сохраняет интегрируемость: один из примеров был разобран в работе А.И.Скворцова и В.В.Ведюшкиной [130].

К ряду других содержательных результатов приводит ослабление условия на выпусклость углов: если предположить, что граница бильярдного стола может иметь углы $\frac{3\pi}{2}$, тогда результирующий псевдоинтегрируемый бильярд имеет совместные поверхности уровня первых интегралов, негомеоморфные торам. Упомянем здесь работы В.Драговича и М.Раднович [131-133], а также В.А.Москвина [134,135]. 

Перспективными направлениями также являются класс бильярдов с проскальзыванием вдоль границы стола, который был предложен А.Т.Фоменко в работе [63] и так называемые бильярдные игры [100]. Отметим также бильярды на склеенных из плоских столов $CW$- комплексах с перестановками [Ведюшкина].

Автором совместно с Ф.\,Ю.\,Попеленским был обнаружен новый класс интегрируемых бильярдов.
Пусть область $\Omega$ ограничивается набором софокусных квадрик и разбивается дугами квадрик того же семейства на области $\Omega_i$. Припишем каждой области $\Omega_i$ коэффициент $n_i$, имеющий смысл показателя преломления.
Рассмотрим движение материальной точки в области $\Omega$: будем считать, что на внешней границе $\Omega$ движение подчиняется закону `угол падения равен углу отражения', а на общей границе областей $\Omega_i$ и $\Omega_j$ выполняется соотношение $n_i \cos \theta_i = n_j \cos \theta_j$. Гдесь $\theta_i, \theta_j$ -- углы, которые образуют отрезки траектории с нормалью к кривой $C \subset (\partial \Omega_i \cap \partial \Omega_j)$, если $\theta_i$ и $\theta_j$ корректно определены. В работе для такой системы исследуется интеграл движения $\Xi$. Дополнительно в работе описаны слоения изоэнергетического многообразия на поверхности уровня интеграла $\Xi$ для двух разбиений областей $\Omega$. 

{\aim} 
Диссертационная работа преследует следующие цели:
\begin{enumerate}[beginpenalty=10000] % https://tex.stackexchange.com/a/476052/104425
  \item Вычисление асимптотики собственных значений оператора Шрёдингера в зависимости от расстояния между фокусами для:
  \begin{itemize}[beginpenalty=10000] % https://tex.stackexchange.com/a/476052/104425
  \item конечно-листного накрытия области, ограниченной двумя софокусными эллипсами (<<эллиптическое кольцо>>)
  \item симметричной относительно горизонтальной оси области, ограниченной дугой эллипса и ветвью софокусной  гиперболы (<<эллиптический сектор>> вида $A_\delta$)  (см. рис. \ref{fig:intro_quantum_domains}).
  \item области, ограниченной отрезком горизонтальной оси, дугой эллипса и ветвями двух софокусных с эллипсом гипербол (<<эллиптический сектор>> вида $B_\delta$)  (см. рис. \ref{fig:intro_quantum_domains}).
  \end{itemize}
   \begin{figure}[ht]
    \centerfloat{
        \hfill
        \subcaptionbox{<<эллиптический сектор>> вида $A_\delta$}{%
\includegraphics[width=3.5cm]{right1.pdf}}
        \hfill
        \subcaptionbox{<<эллиптический сектор>> вида $B_\delta$}{%
\includegraphics[width=3.5cm]{up1.pdf}}
        \hfill
    }
    \caption{Софокусные столы для квантовой задачи.}\label{fig:intro_quantum_domains}
\end{figure}
  
  \item Для бильярда с косинусным законом преломления на софокусных столах:
    \begin{itemize}[beginpenalty=10000]
	  \item Исследовать динамическую систему на интегрируемость. Доказать существование дополнительного интеграла движения $\Xi$.
	  \item Исследовать поверхности постоянного уровня первого интеграла $\Xi$ для двух разбиений бильярдного стола (см. рис. \ref{fig:intro_classical_domains}).
  \begin{figure}[ht]
    \centerfloat{
        \hfill
%        \subcaptionbox{Область для задачи А}
        \hfill
%        \subcaptionbox{Область для задачи Б}{%
\includegraphics[width=1.7cm]{images/ch4/section3_circular/domain.pdf}
%{
        \hfill
    }
    \caption{Разбиения софокусных столов для классической задачи.}\label{fig:intro_classical_domains}
\end{figure}
    \end{itemize}
\end{enumerate}

{\defpositions}
\begin{enumerate}[beginpenalty=10000] % https://tex.stackexchange.com/a/476052/104425
  \item Вычислены и приведены коэффициенты разложения собственных значений $E_{k,m}$ стационарного оператора Шрёдингера по степеням половины фокального расстояния $\delta$ для:
   \begin{itemize}[beginpenalty=10000] % https://tex.stackexchange.com/a/476052/104425
  \item конечно-листного накрытия области, ограниченной двумя софокусными эллипсами (<<эллиптическое кольцо>>) 
  \item симметричной относительно горизонтальной оси области, ограниченной дугой эллипса и ветвью софокусной  гиперболы (<<эллиптический сектор>> вида $A_\delta$, см. рис. \ref{fig:intro_quantum_domains})
  \item области, ограниченной отрезком горизонтальной оси, дугой эллипса и ветвями двух софокусных с эллипсом гипербол (<<эллиптический сектор>> вида $B_\delta$, см. рис. \ref{fig:intro_quantum_domains})
  \end{itemize}
  Коэффициенты получены для всех натуральных $k$ и $m$ с точностью до второго порядка включительно.

  \item Вычислена явная формула для интеграла движения $\Xi$ для бильярда в области, ограниченной эллипсом, подчиняющегося преломлениям согласно косинусному закону на дугах софокусных квадрик.
  \item Построены поверхности постоянного уровня дополнительного интеграла $\Xi$ для свободной частицы в эллипсе при преломлении траектории согласно косинусному закону на софокусном эллипсе (см. рис. \ref{fig:intro_classical_domains}). Поверхности построены для регулярных и критических значений интеграла $\Xi$.
   \item Построены поверхности постоянного уровня дополнительного интеграла $\Xi$ для свободной частицы в круге при преломлении траектории согласно косинусному закону на окружности меньшего радиуса и сегменте радиальной прямой (см. рис. \ref{fig:intro_classical_domains}). Поверхности построены для регулярных и критических значений интеграла $\Xi$.
\end{enumerate}
%В папке Documents можно ознакомиться с решением совета из Томского~ГУ
%(в~файле \verb+Def_positions.pdf+), где обоснованно даются рекомендации
%по~формулировкам защищаемых положений.


%настоящей работы является получение этой асимптотики с точностью до второго порядка. Ожидается, что в нулевом порядке уровни энергии будут совпадать с результатами~\cite{wref13} для кругового сектора и~\cite[\S~207, с.~276]{wref11} для накрытия кругового кольца кратности $p=1$.

% из первой статьи про косинусное преломление:
%Интерес к составным бильярдам возникает в связи с активно разрабатываемой  в школе А.~Т.~Фоменко теорией бильярдов со сложной топологией, таких как бильярдные книжки. Современное состояние этой теории см. в обзоре [1].  
%
%Рассмотрим ограниченную эллипсом область, разбитую дугой софокусной квадрики на две области с разными плотностями, но постоянными внутри каждой из областей. Тогда мы можем рассмотреть бильярдную траекторию, которая при пересечении границы раздела двух сред меняет направление по закону Снеллиуса [2]: отношение синусов углов падения и преломления равно обратному отношению плотностей сред. 
%Экспериментальная компьютерная проверка демонстрирует, что такой бильярд неинтегрируем. 
%
%С другой стороны, в физике известны законы преломления другого вида. 
%В задачах теплопроводности направление векторов плотностей теплового потока на границе раздела двух сред определяется отношением тангенсов [3, 4].
%В некоторых задачах оптики встречаются законы, формулируемые в терминах отношения косинусов углов падения и преломления [5, 6].
%
%В настоящей работе мы рассматриваем бильярд в эллипсе, разделенном дугами софокусных квадрик на несколько областей $\Omega_i$ с постоянными в них плотностями $n_i$, при этом закон преломления задан равенством $n_1 \cos{\theta_1} = n_2 \cos{\theta_2}$. Мы покажем, что в полученная система будет интегрируемой (см. пункты 5 и 6) и предъявим дополнительный интеграл. В некоторых случаях значения этого дополнительного интеграла принадлежат не прямой, а окружности, см. пункт 7.
%Для~достижения поставленной цели необходимо было решить следующие задачи:

%В настоящей работе  эта асимптотика получена с точностью до второго порядка. В нулевом порядке уровни энергии совпадают с результатами~\cite{wref13} для кругового сектора и~\cite[\S~207, с.~276]{wref11} для накрытия кругового кольца кратности $p=1$.


{\novelty} 
Все положения диссертации, выносимые на защиту, являются оригинальными и получены автором самостоятельно  или при равноценном вкладе с соавторами. Кроме того, диссертация содержит следующие вспомогательные результаты, которые также являются новыми:
  \begin{itemize}[beginpenalty=10000] % https://tex.stackexchange.com/a/476052/104425
  \item впервые исследован косинусный закон преломления бильярдной траектории на софокусных столах
  \item для этого закона приведена методика построения бифуркационных диаграмм нового типа, одновременно учитывающих все возможные значения <<оптических>> параметров областей
  \item впервые получены особые поверхности, соответствующие одновременным бифуркациям разных типов в разных частях бильярдного стола
  \end{itemize}

%Для уровней энергии получены явные асимптотические выражения, в частности, формулы ($\ref{eq:funcRing}$) и ($\ref{eq:valRing}$) для $p$-листного накрытия эллиптического кольца, а также формулы ($\ref{eq:fun}$) и ($\ref{eq:val}$) для области $A_{\fixme{\varepsilon}}$ и формулы  ($\ref{eq:funB}$) и ($\ref{eq:valB}$) для области  $B_{\fixme{\varepsilon}}$. Для особых случаев для областей $A_{\fixme{\varepsilon}}$ и $B_{\fixme{\varepsilon}}$ справедливы формулы (\ref{eq:valS1}) и (\ref{eq:valS2}). 
%Приведенные асимптотики для собственных значений в зависимости от расстояния между фокусами справедливы с точностью до второго порядка включительно и подтверждаются численными экспериментами.

%Интегрируемость классических бильярдов в областях $A_{\fixme{\varepsilon}}$ и $B_{\fixme{\varepsilon}}$ следует из существования сохраняющейся величины в дополнение к полной энергии.
%В приложении~\ref{app:A} приведен квантовый аналог этой величины, сохраняющейся для квантовых бильярдов в этих же областях.
%
%\begin{enumerate}[beginpenalty=10000] % https://tex.stackexchange.com/a/476052/104425
%  \item Впервые \ldots
%  \item Впервые \ldots
%  \item Было выполнено оригинальное исследование \ldots
%\end{enumerate}
%
{\methods} В работе используются элементы теории Штурма и теории специальных функций, методы теории краевых задач и математического анализа. В исследовании бильярда с косинусным законом преломления на софокусных квадриках применяются методы теории топологической классификации интегрируемых
гамильтоновых систем с одной и двумя степенями свободы, построенной А.Т. Фоменко, Х. Цишангом, А.В. Болсиновым и многими другими.

{\influence} 
Диссертация имеет теоретический характер.
Полученные результаты могут быть использованы при исследовании собственных функций и собственных значений оператор Лапласа в областях, ограниченных софокусными квадриками.
Разработанные методы позволяют получить асимптотику собственных значений в том числе для областей, деформация которых не удовлетворяет необходимым условиям вариационной формулы Адамара (например, если рассматривать для <<эллиптических секторов>> $A_\delta$ и $B_\delta$ в качестве параметра деформации величину $\delta^2$).

Ценность исследования бильярда с косинусным законом преломления заключается в расширении класса бильярдных задач, что позволяет строить новые интересные примеры интегрируемых систем, топология слоений Лиувилля для которых весьма нетривиальна. 
В рамках нового научного направления в теории биллиардов рассмотрена возможность неодновременных перестроек в областях бильярдного стола, соответствующих разным оптическим плотностям, а также одновременные неодинаковые перестройки в них.
Кроме того, получен динамический бильярд с первым интегралом, множество значений которого не является интервалом.

{\probation}
Основные результаты диссертации обоснованы в виде строгих математических доказательств и прошли апробацию на следующих научных конференциях и семинарах:

\begin{enumerate}%[beginpenalty=10000]
\item Студенческая школа-конференция <<Математическая весна --  2023>>, Нижний Новгород, Россия, 27-30 марта 2023;

\item Ломоносовские чтения 2023, Россия, 4-14 апреля 2023;

\item XXX Международная научная конференция студентов, аспирантов и молодых учёных <<Ломоносов-2023>>,  Москва, Россия, 10-21 апреля 2023;

\item Воронежская зимняя математическая школа С.Г. Крейна, Воронеж, Россия, 26-30 января 2024;

\item Студенческая школа-конференция <<Математическая весна -- 2024>>, Нижний Новгород, Россия, 25-28 марта 2024;

\item XXXI Международная конференция студентов, аспирантов и молодых ученых <<Ломоносов--2024>>, Москва, Россия, 12-26 апреля 2024;

%\item Современные геометрические и топологические методы, Сириус, Россия, 14-19 мая 2024;	%там не по диссеру был доклад
\item Семинар “Дифференциальная геометрия и приложения” под руководством акад. А.Т. Фоменко на механико-математическом факультете МГУ имени М.В. Ломоносова, 25 ноября 2024;


\item XXXII Международная научная конференция студентов, аспирантов и молодых ученых <<Ломоносов--2025>>, Москва, Россия, 11-25 апреля 2025.
\end{enumerate}
%{\contribution} Автор принимал активное участие \ldots


\ifnumequal{\value{bibliosel}}{0}
{%%% Встроенная реализация с загрузкой файла через движок bibtex8. (При желании, внутри можно использовать обычные ссылки, наподобие `\cite{vakbib1,vakbib2}`).
    {\publications} Основные результаты по теме диссертации изложены
    в четырех работах \nocite{nikulin2023spektr611484954, nikulin2024asymptotic617844539, vestnikLatest, sbornikLatest}
    X из которых изданы в журналах, рекомендованных ВАК,
    X "--- в тезисах докладов.
}%
{%%% Реализация пакетом biblatex через движок biber
    \begin{refsection}[bl-author, bl-registered]
        % Это refsection=1.
        % Процитированные здесь работы:
        %  * подсчитываются, для автоматического составления фразы "Основные результаты ..."
        %  * попадают в авторскую библиографию, при usefootcite==0 и стиле `\insertbiblioauthor` или `\insertbiblioauthorgrouped`
        %  * нумеруются там в зависимости от порядка команд `\printbibliography` в этом разделе.
        %  * при использовании `\insertbiblioauthorgrouped`, порядок команд `\printbibliography` в нём должен быть тем же (см. biblio/biblatex.tex)
        %
        % Невидимый библиографический список для подсчёта количества публикаций:
        \phantom{\printbibliography[heading=nobibheading, section=1, env=countauthorvak,          keyword=biblioauthorvak]%
        \printbibliography[heading=nobibheading, section=1, env=countauthorwos,          keyword=biblioauthorwos]%
        \printbibliography[heading=nobibheading, section=1, env=countauthorscopus,       keyword=biblioauthorscopus]%
        \printbibliography[heading=nobibheading, section=1, env=countauthorconf,         keyword=biblioauthorconf]%
        \printbibliography[heading=nobibheading, section=1, env=countauthorother,        keyword=biblioauthorother]%
        \printbibliography[heading=nobibheading, section=1, env=countregistered,         keyword=biblioregistered]%
        \printbibliography[heading=nobibheading, section=1, env=countauthorpatent,       keyword=biblioauthorpatent]%
        \printbibliography[heading=nobibheading, section=1, env=countauthorprogram,      keyword=biblioauthorprogram]%
        \printbibliography[heading=nobibheading, section=1, env=countauthor,             keyword=biblioauthor]%
        \printbibliography[heading=nobibheading, section=1, env=countauthorvakscopuswos, filter=vakscopuswos]%
        \printbibliography[heading=nobibheading, section=1, env=countauthorscopuswos,    filter=scopuswos]}%
        %
        \nocite{*}%
        %
        {\publications} Основные результаты по теме диссертации изложены в~\arabic{citeauthor}~печатных работах, 
        \arabic{citeauthorvak} из которых изданы в журналах, рекомендованных ВАК%
        \ifnum \value{citeauthorscopuswos}>0%
            , \arabic{citeauthorscopuswos} "--- в~периодических научных журналах, индексируемых Web of~Science и Scopus%
        \fi%
        \ifnum \value{citeauthorconf}>0%
            , \arabic{citeauthorconf} "--- в~тезисах докладов.
        \else%
            .
        \fi%
        \ifnum \value{citeregistered}=1%
            \ifnum \value{citeauthorpatent}=1%
                Зарегистрирован \arabic{citeauthorpatent} патент.
            \fi%
            \ifnum \value{citeauthorprogram}=1%
                Зарегистрирована \arabic{citeauthorprogram} программа для ЭВМ.
            \fi%
        \fi%
        \ifnum \value{citeregistered}>1%
            Зарегистрированы\ %
            \ifnum \value{citeauthorpatent}>0%
            \formbytotal{citeauthorpatent}{патент}{}{а}{}%
            \ifnum \value{citeauthorprogram}=0 . \else \ и~\fi%
            \fi%
            \ifnum \value{citeauthorprogram}>0%
            \formbytotal{citeauthorprogram}{программ}{а}{ы}{} для ЭВМ.
            \fi%
        \fi%
        % К публикациям, в которых излагаются основные научные результаты диссертации на соискание учёной
        % степени, в рецензируемых изданиях приравниваются патенты на изобретения, патенты (свидетельства) на
        % полезную модель, патенты на промышленный образец, патенты на селекционные достижения, свидетельства
        % на программу для электронных вычислительных машин, базу данных, топологию интегральных микросхем,
        % зарегистрированные в установленном порядке.(в ред. Постановления Правительства РФ от 21.04.2016 N 335)
    \end{refsection}%
    \begin{refsection}[bl-author, bl-registered]
        % Это refsection=2.
        % Процитированные здесь работы:
        %  * попадают в авторскую библиографию, при usefootcite==0 и стиле `\insertbiblioauthorimportant`.
        %  * ни на что не влияют в противном случае
        \nocite{vakbib2}%vak
        \nocite{patbib1}%patent
        \nocite{progbib1}%program
        \nocite{bib1}%other
        \nocite{confbib1}%conf
        \nocite{nikulin2023spektr611484954}
	\nocite{nikulin2024asymptotic617844539}
	\nocite{vestnikLatest}
	\nocite{sbornikLatest}

    \end{refsection}%
        %
        % Всё, что вне этих двух refsection, это refsection=0,
        %  * для диссертации - это нормальные ссылки, попадающие в обычную библиографию
        %  * для автореферата:
        %     * при usefootcite==0, ссылка корректно сработает только для источника из `external.bib`. Для своих работ --- напечатает "[0]" (и даже Warning не вылезет).
        %     * при usefootcite==1, ссылка сработает нормально. В авторской библиографии будут только процитированные в refsection=0 работы.
        \nocite{nikulin2023spektr611484954}
	\nocite{nikulin2024asymptotic617844539}
	\nocite{vestnikLatest}
	\nocite{sbornikLatest}
}


\ifsynopsis
{\volumeAndStructure} Диссертация состоит из~введения,
\fixme{XX} глав, заключения и~приложения. Полный объем диссертации
\fixme{ХХХ}~страниц с~\fixme{ХХ}~рисунками и~\fixme{5}~таблицами. Список
литературы содержит \fixme{ХХX}~наименование.
\else
\begin{refsection}[bl-author, bl-registered]
	% Это refsection=3.
	% Для подсчёта позиций списка литературы при группировке работ автора
	%
	% Невидимый библиографический список для подсчёта количества публикаций:
	\printbibliography[heading=nobibheading, section=0, env=counter, keyword=bibliofull]%
	%
	\nocite{*}%
	%% authorother
	%		\nocite{bib1}%
	%		\nocite{bib2}%
	%
	{\volumeAndStructure} Диссертация состоит из~введения,
	\formbytotal{totalchapter}{глав}{ы}{}{} и заключения.
	Полный объём диссертации составляет
	\formbytotal{TotPages}{страниц}{у}{ы}{}, включая
	\formbytotal{totalcount@figure}{рисун}{ок}{ка}{ков} и
	\formbytotal{totalcount@table}{таблиц}{у}{ы}{}.
	Список литературы содержит
	\formbytotal{citenum}{наименован}{ие}{ия}{ий}.
\end{refsection}%
\fi
