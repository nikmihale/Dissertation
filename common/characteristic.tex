
{\actuality} 
В последние годы был достигнут значительный прогресс в понимании гипотезы Биркгофа об интегрируемых плоских бильярдах, например, см.~\cite{bm2022}.
С другой стороны, были найдены строгие доказательства неинтегрируемости некоторых бильярдов, таких как эллиптические бильярды в сильном магнитном поле~\cite{bm2020, bm2019}. Неинтегрируемые бильярды, как и бильярды в магнитном поле, исследуются давно, см.~\cite{berry1985, berry1986}.
Изменение бильярдной области с эллипса на, например, стадион, немедленно приводит к хаотической динамике, \cite{bunimovich1974, stockmann2000}.

Хорошо известно, что классический бильярд в ограниченной софокусными квадриками области интегрируем. Интегрируемость  этой системы следует из того, что помимо сохранения полной энергии существует другая постоянная величина, а именно произведение угловых моментов относительно общих фокусов граничных квадрик.
В случае, если фокусы совпадают, степень этого дополнительного интеграла можно понизить, поскольку угловой момент относительно центра круговой области сохраняется.
Однако, в работе \cite{wref13} было показано, что для бильярда в круговом секторе сохраняемой величиной является не  угловой момент, а квадрат углового момента относительно вершины угла.


Недавние работы А.\,Т.~Фоменко и В.\,В.~Ведюшкиной (см. \cite{wref6,wref7,wref8} , а также другие публикации этих авторов) вновь привлекли к этой теме внимание специалистов. 
В частности, в работе \cite{wref6} изучались особенности бильярда в кольце, ограниченном софокусными эллипсами (`эллиптическое кольцо'). В настоящей работе рассматривается соответствующая квантовая система, а именно изучается спектр оператора Шрёдингера в этой области и в ее накрытиях. 
В работе также рассмотрена область (`эллиптический сектор'), ограниченная эллипсом с большой полуосью $0 <r_0$ и эксцентриситетом $\varepsilon$ и 
одной ветвью софокусной гиперболы  (область $A_\varepsilon$)
или двумя ветвями софокусных гипербол (область $B_\varepsilon$).
Все упомянутые эллипсы и гиперболы имеют общие фокусы в точках $(\pm r_0\varepsilon, 0)$.

Для обоих типов областей получены точные решения стационарного уравнения Шрёдингера, а именно собственные функции и соответствующие уровни энергии.

Области $A_\varepsilon$ и $B_\varepsilon$ связаны с бильярдами в круговых секторах следующим образом: при стремлении $\varepsilon$ к 0, обе области стремятся к круговому сектору с некоторым центральным углом. Аналогично, при стремлении $\varepsilon$ к 0 эллиптическое кольцо стремится к круговому кольцу. 

Таким образом, возникает естественный вопрос об установлении асимптотики уровней энергии при $\varepsilon\to 0$. 

{\aim} 
Диссертационная работа преследует следующие цели:
\begin{enumerate}[beginpenalty=10000] % https://tex.stackexchange.com/a/476052/104425
  \item Исследовать, разработать, вычислить и~т.\:д. и~т.\:п.
  \item Исследовать, разработать, вычислить и~т.\:д. и~т.\:п.
  \item Исследовать, разработать, вычислить и~т.\:д. и~т.\:п.
  \item Исследовать, разработать, вычислить и~т.\:д. и~т.\:п.
\end{enumerate}

%настоящей работы является получение этой асимптотики с точностью до второго порядка. Ожидается, что в нулевом порядке уровни энергии будут совпадать с результатами~\cite{wref13} для кругового сектора и~\cite[\S~207, с.~276]{wref11} для накрытия кругового кольца кратности $p=1$.

% из первой статьи про косинусное преломление:
%Интерес к составным бильярдам возникает в связи с активно разрабатываемой  в школе А.~Т.~Фоменко теорией бильярдов со сложной топологией, таких как бильярдные книжки. Современное состояние этой теории см. в обзоре [1].  
%
%Рассмотрим ограниченную эллипсом область, разбитую дугой софокусной квадрики на две области с разными плотностями, но постоянными внутри каждой из областей. Тогда мы можем рассмотреть бильярдную траекторию, которая при пересечении границы раздела двух сред меняет направление по закону Снеллиуса [2]: отношение синусов углов падения и преломления равно обратному отношению плотностей сред. 
%Экспериментальная компьютерная проверка демонстрирует, что такой бильярд неинтегрируем. 
%
%С другой стороны, в физике известны законы преломления другого вида. 
%В задачах теплопроводности направление векторов плотностей теплового потока на границе раздела двух сред определяется отношением тангенсов [3, 4].
%В некоторых задачах оптики встречаются законы, формулируемые в терминах отношения косинусов углов падения и преломления [5, 6].
%
%В настоящей работе мы рассматриваем бильярд в эллипсе, разделенном дугами софокусных квадрик на несколько областей $\Omega_i$ с постоянными в них плотностями $n_i$, при этом закон преломления задан равенством $n_1 \cos{\theta_1} = n_2 \cos{\theta_2}$. Мы покажем, что в полученная система будет интегрируемой (см. пункты 5 и 6) и предъявим дополнительный интеграл. В некоторых случаях значения этого дополнительного интеграла принадлежат не прямой, а окружности, см. пункт 7.
%Для~достижения поставленной цели необходимо было решить следующие задачи:

%В настоящей работе  эта асимптотика получена с точностью до второго порядка. В нулевом порядке уровни энергии совпадают с результатами~\cite{wref13} для кругового сектора и~\cite[\S~207, с.~276]{wref11} для накрытия кругового кольца кратности $p=1$.

{\methods} В работе используются элементы теории Штурма и теории специальных функций, методы теории краевых задач и математического анализа


{\novelty} Для уровней энергии получены явные асимптотические выражения, в частности, формулы ($\ref{eq:funcRing}$) и ($\ref{eq:valRing}$) для $p$-листного накрытия эллиптического кольца, а также формулы ($\ref{eq:fun}$) и ($\ref{eq:val}$) для области $A_\varepsilon$ и формулы  ($\ref{eq:funB}$) и ($\ref{eq:valB}$) для области  $B_\varepsilon$. Для особых случаев для областей $A_\varepsilon$ и $B_\varepsilon$ справедливы формулы (\ref{eq:valS1}) и (\ref{eq:valS2}). 
Приведенные асимптотики для собственных значений в зависимости от расстояния между фокусами справедливы с точностью до второго порядка включительно и подтверждаются численными экспериментами.

Интегрируемость классических бильярдов в областях $A_\varepsilon$ и $B_\varepsilon$ следует из существования сохраняющейся величины в дополнение к полной энергии.
В приложении~\ref{app:A} приведен квантовый аналог этой величины, сохраняющейся для квантовых бильярдов в этих же областях.

\begin{enumerate}[beginpenalty=10000] % https://tex.stackexchange.com/a/476052/104425
  \item Впервые \ldots
  \item Впервые \ldots
  \item Было выполнено оригинальное исследование \ldots
\end{enumerate}

{\defpositions}
\begin{enumerate}[beginpenalty=10000] % https://tex.stackexchange.com/a/476052/104425
  \item Первое положение
  \item Второе положение
  \item Третье положение
  \item Четвертое положение
\end{enumerate}
В папке Documents можно ознакомиться с решением совета из Томского~ГУ
(в~файле \verb+Def_positions.pdf+), где обоснованно даются рекомендации
по~формулировкам защищаемых положений.

{\valueT} полученных результатов обеспечивается \ldots \ Результаты находятся в соответствии с результатами, полученными другими авторами.


{\probation}
Основные результаты работы докладывались~на:
перечисление основных конференций, симпозиумов и~т.\:п.

%{\contribution} Автор принимал активное участие \ldots

\ifnumequal{\value{bibliosel}}{0}
{%%% Встроенная реализация с загрузкой файла через движок bibtex8. (При желании, внутри можно использовать обычные ссылки, наподобие `\cite{vakbib1,vakbib2}`).
    {\publications} Основные результаты по теме диссертации изложены
    в~XX~печатных изданиях,
    X из которых изданы в журналах, рекомендованных ВАК,
    X "--- в тезисах докладов.
}%
{%%% Реализация пакетом biblatex через движок biber
    \begin{refsection}[bl-author, bl-registered]
        % Это refsection=1.
        % Процитированные здесь работы:
        %  * подсчитываются, для автоматического составления фразы "Основные результаты ..."
        %  * попадают в авторскую библиографию, при usefootcite==0 и стиле `\insertbiblioauthor` или `\insertbiblioauthorgrouped`
        %  * нумеруются там в зависимости от порядка команд `\printbibliography` в этом разделе.
        %  * при использовании `\insertbiblioauthorgrouped`, порядок команд `\printbibliography` в нём должен быть тем же (см. biblio/biblatex.tex)
        %
        % Невидимый библиографический список для подсчёта количества публикаций:
        \phantom{\printbibliography[heading=nobibheading, section=1, env=countauthorvak,          keyword=biblioauthorvak]%
        \printbibliography[heading=nobibheading, section=1, env=countauthorwos,          keyword=biblioauthorwos]%
        \printbibliography[heading=nobibheading, section=1, env=countauthorscopus,       keyword=biblioauthorscopus]%
        \printbibliography[heading=nobibheading, section=1, env=countauthorconf,         keyword=biblioauthorconf]%
        \printbibliography[heading=nobibheading, section=1, env=countauthorother,        keyword=biblioauthorother]%
        \printbibliography[heading=nobibheading, section=1, env=countregistered,         keyword=biblioregistered]%
        \printbibliography[heading=nobibheading, section=1, env=countauthorpatent,       keyword=biblioauthorpatent]%
        \printbibliography[heading=nobibheading, section=1, env=countauthorprogram,      keyword=biblioauthorprogram]%
        \printbibliography[heading=nobibheading, section=1, env=countauthor,             keyword=biblioauthor]%
        \printbibliography[heading=nobibheading, section=1, env=countauthorvakscopuswos, filter=vakscopuswos]%
        \printbibliography[heading=nobibheading, section=1, env=countauthorscopuswos,    filter=scopuswos]}%
        %
        \nocite{*}%
        %
        {\publications} Основные результаты по теме диссертации изложены в~\arabic{citeauthor}~печатных изданиях,
        \arabic{citeauthorvak} из которых изданы в журналах, рекомендованных ВАК%
        \ifnum \value{citeauthorscopuswos}>0%
            , \arabic{citeauthorscopuswos} "--- в~периодических научных журналах, индексируемых Web of~Science и Scopus%
        \fi%
        \ifnum \value{citeauthorconf}>0%
            , \arabic{citeauthorconf} "--- в~тезисах докладов.
        \else%
            .
        \fi%
        \ifnum \value{citeregistered}=1%
            \ifnum \value{citeauthorpatent}=1%
                Зарегистрирован \arabic{citeauthorpatent} патент.
            \fi%
            \ifnum \value{citeauthorprogram}=1%
                Зарегистрирована \arabic{citeauthorprogram} программа для ЭВМ.
            \fi%
        \fi%
        \ifnum \value{citeregistered}>1%
            Зарегистрированы\ %
            \ifnum \value{citeauthorpatent}>0%
            \formbytotal{citeauthorpatent}{патент}{}{а}{}%
            \ifnum \value{citeauthorprogram}=0 . \else \ и~\fi%
            \fi%
            \ifnum \value{citeauthorprogram}>0%
            \formbytotal{citeauthorprogram}{программ}{а}{ы}{} для ЭВМ.
            \fi%
        \fi%
        % К публикациям, в которых излагаются основные научные результаты диссертации на соискание учёной
        % степени, в рецензируемых изданиях приравниваются патенты на изобретения, патенты (свидетельства) на
        % полезную модель, патенты на промышленный образец, патенты на селекционные достижения, свидетельства
        % на программу для электронных вычислительных машин, базу данных, топологию интегральных микросхем,
        % зарегистрированные в установленном порядке.(в ред. Постановления Правительства РФ от 21.04.2016 N 335)
    \end{refsection}%
    \begin{refsection}[bl-author, bl-registered]
        % Это refsection=2.
        % Процитированные здесь работы:
        %  * попадают в авторскую библиографию, при usefootcite==0 и стиле `\insertbiblioauthorimportant`.
        %  * ни на что не влияют в противном случае
        \nocite{vakbib2}%vak
        \nocite{patbib1}%patent
        \nocite{progbib1}%program
        \nocite{bib1}%other
        \nocite{confbib1}%conf
    \end{refsection}%
        %
        % Всё, что вне этих двух refsection, это refsection=0,
        %  * для диссертации - это нормальные ссылки, попадающие в обычную библиографию
        %  * для автореферата:
        %     * при usefootcite==0, ссылка корректно сработает только для источника из `external.bib`. Для своих работ --- напечатает "[0]" (и даже Warning не вылезет).
        %     * при usefootcite==1, ссылка сработает нормально. В авторской библиографии будут только процитированные в refsection=0 работы.
}

При использовании пакета \verb!biblatex! будут подсчитаны все работы, добавленные
в файл \verb!biblio/author.bib!. Для правильного подсчёта работ в~различных
системах цитирования требуется использовать поля:
\begin{itemize}
        \item \texttt{authorvak} если публикация индексирована ВАК,
        \item \texttt{authorscopus} если публикация индексирована Scopus,
        \item \texttt{authorwos} если публикация индексирована Web of Science,
        \item \texttt{authorconf} для докладов конференций,
        \item \texttt{authorpatent} для патентов,
        \item \texttt{authorprogram} для зарегистрированных программ для ЭВМ,
        \item \texttt{authorother} для других публикаций.
\end{itemize}
Для подсчёта используются счётчики:
\begin{itemize}
        \item \texttt{citeauthorvak} для работ, индексируемых ВАК,
        \item \texttt{citeauthorscopus} для работ, индексируемых Scopus,
        \item \texttt{citeauthorwos} для работ, индексируемых Web of Science,
        \item \texttt{citeauthorvakscopuswos} для работ, индексируемых одной из трёх баз,
        \item \texttt{citeauthorscopuswos} для работ, индексируемых Scopus или Web of~Science,
        \item \texttt{citeauthorconf} для докладов на конференциях,
        \item \texttt{citeauthorother} для остальных работ,
        \item \texttt{citeauthorpatent} для патентов,
        \item \texttt{citeauthorprogram} для зарегистрированных программ для ЭВМ,
        \item \texttt{citeauthor} для суммарного количества работ.
\end{itemize}
% Счётчик \texttt{citeexternal} используется для подсчёта процитированных публикаций;
% \texttt{citeregistered} "--- для подсчёта суммарного количества патентов и программ для ЭВМ.

Для добавления в список публикаций автора работ, которые не были процитированы в
автореферате, требуется их~перечислить с использованием команды \verb!\nocite! в
\verb!Synopsis/content.tex!.
