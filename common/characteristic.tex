
{\actuality} 
Диссертация посвящена исследованию квантовых и модифицированных классических бильярдов в областях, ограниченных софокусными квадриками.

В последние годы в проблеме нахождения интегрируемых модификаций эллиптического бильярда достигнут значительный прогресс. Основными здесь являются работы, выполненные группой исследователей, возглавляемых А.~Т.~Фоменко и В.~В.~Ведюшкиной (Фокичевой). 
Объем полученных этой группой результатов настолько значителен, что даже краткое их изложение  в рамках данного введения не представляется возможным, познакомиться с этими результатами можно по важному обзору \cite{FomVed23}.
Тем не менее, упомянем некоторые их результаты.
Была разработана теория бильярдных книжек --- бильярдных систем на  $CW$-комплексах, склеенных из плоских ограниченных софокусными квадриками столов \cite{VedFom19}.
В терминах бильярдных книжек был сформулирован ряд гипотез Фоменко о моделировании интегрируемых гамильтоновых систем и их особенностей.
В работе \cite{zbMATH07344445}  был введен класс бильярдов с проскальзыванием вдоль границы стола. Другому расширению интегрируемых бильярдов --- силовым (эволюционным) бильярдам посвящена недавняя статья \cite{FomVed22}.  
 
%Первая тема, которую мы рассмотрим, --- это развитие теории математических бильярдов.
%Подробный обзор актуальных исследований в этой области приведен в \cite{FomVed23}. Приведем ключевые из них.

Хорошо известно, что бильярд в эллипсе в присутствии постоянного магнитного поля, перпендикулярного плоскости стола, не интегрируем (физическая точка зрения отражена в большом количестве статей, в частности, см. Берри и Робник \cite{berry1985, berry1986}). 
Математически строгие утверждения, в том числе для бильярдных столов на поверхностях постоянной кривизны,  содержатся в работах \cite{bm2019, zbMATH06661562, bm2020}.
Однако, бильярд в магнитном поле в круге интегрируем и демонстрирует интересное поведение с точки зрения инвариантов интегрируемых гамильтоновых систем.

Интегрируемой является интересная система, введенная Драговичем и Раднович в \cite{DraGasRad22} (так называемая <<бильярдная игра>>), см. также недавнюю работу \cite{wire}.
Кроме того, имеется цикл работ про интегрируемые бильярды с потенциалом, см. например \cite{Dra1, Dra4}, инициированный работой В.~В.~Козлова \cite{Kozlov}.
Пример с центральным потенциалом типа Гука на софокусном столе был разобран в работе А.И.Скворцова и В.В.Ведюшкиной \cite{VedSkv22}.
Софокусные бильярды в эллипсе с метрикой Минковского были рассмотрены В. Драговичем, М. Раднович \cite{DraRad15} и Е.Е.Каргиновой \cite{Kar19, Kar20}. 

%К ряду других содержательных результатов приводит ослабление условия на выпусклость углов: если предположить, что граница бильярдного стола может иметь углы $\frac{3\pi}{2}$, тогда результирующий псевдоинтегрируемый бильярд имеет совместные поверхности уровня первых интегралов, негомеоморфные торам. Упомянем здесь работы В.Драговича и М.Раднович \cite{zbMATH06376857, DraRad151, zbMATH06467543} и  В.А.Москвина \cite{Mos18, Mos20}. 

Автором совместно с Ф.\,Ю.\,Попеленским был обнаружен новый класс интегрируемых бильярдов.
Пусть область $\Omega$ ограничивается набором софокусных квадрик и разбивается дугами квадрик того же семейства на области $\Omega_i$. Припишем каждой области $\Omega_i$ коэффициент $n_i$, имеющий смысл показателя преломления.
Рассмотрим движение материальной точки в области $\Omega$: будем считать, что на внешней границе $\Omega$ движение подчиняется закону `угол падения равен углу отражения', а на общей границе областей $\Omega_i$ и $\Omega_j$ выполняется соотношение $n_i \cos \theta_i = n_j \cos \theta_j$. Гдесь $\theta_i, \theta_j$ -- углы, которые образуют отрезки траектории с нормалью к кривой $C \subset (\partial \Omega_i \cap \partial \Omega_j)$, если $\theta_i$ и $\theta_j$ корректно определены. 
Для такой системы возникает интеграл движения $\Xi$, вообще говоря, многозначный. Для двух систем такого рода описаны слоения изоэнергетического многообразия на поверхности уровня дополнительного интеграла $\Xi$. 
\bigskip

Вторая тема, которая обсуждается в диссертации, --- это исследование асимптотического поведения уровней энергии свободной квантовой частицы на софокусных столах при устремлении к нулю расстояния между фокусами.

Данный вопрос тесно связан с общей проблемой изучения собственных значений оператора Лапласа в области $\Omega$ с условием Дирихле на границе $\partial \Omega$.
Для ряда областей ответ выражается в терминах специальных функций. Классическими являются работы Ламэ \cite{lame1852leccons} (равносторонний треугольник) и Рэлея \cite{wref13} (прямоугольник, круг, ограниченная эллипсом область и прямоугольный треугольник). 
Рэлей, в том числе, рассматривал зависимость частоты колебания `почти круговой' эллиптической мембраны от ее эксцентриситета.
В работе \cite{wref13} изучался энергетический спектр квантовой частицы в круговом секторе.
%Например, случай кругового сектора изучается в работе \cite{wref13}, равносторонний треугольник --- в , а для  прямоугольника, круга и прямоугольного треугольника --- в \cite{wref11}.
В общем же случае зависимость собственных значений оператора Лапласа при непрерывной деформации гладкой области изучается в теории возмущений \cite{kato2013perturbation, rellich1969perturbation}. 


В диссертации рассматривается квантовая система в двух областях, ограниченных дугами гипербол и внешнего эллипса, которые при стремлении фокального расстояния к нулю принимают форму кругового сектора.
Классические методы (например, формула Адамара) неприменимы, поскольку границы деформируемых и предельной областей имеют углы.
Для рассматриваемых двух областей получены точные решения в терминах специальных функций Матьё. Вычислено асимптотическое разложение уровней энергии для каждого состояния. У классического дополнительного интеграла имеется квантовый аналог --- наблюдаемая, коммутирующая с гамильтонианом. Для нее вычислен спектр.

{\aim} 
Диссертационная работа преследует следующие цели:
\begin{enumerate}[beginpenalty=10000] % https://tex.stackexchange.com/a/476052/104425
  \item Доказательство интегрируемости классического бильярда на софокусных столах, разделенных софокусными квадриками на области, заполненные изотропными средами с различными показателями `оптической плотности' при условии, что на границе раздела сред выполняется косинусный закон преломления.
 \item Исследовать слоения изоэнергетического многообразия на поверхности постоянного уровня дополнительного интеграла для двух `оптических систем' с косинусным законом преломления, т.е. для софокусных столов, разделенных софокусными квадриками на области, заполненные изотропными средами с различными показателями `оптической плотности'  при условии, что на границе раздела сред выполняется косинусный закон преломления.
   \item Исследование асимптотического поведения уровней энергии свободной квантовой частицы в двух областях, ограниченных дугами гипербол и внешнего эллипса, которые при стремлении фокального расстояния к нулю принимают форму кругового сектора при стремлении фокального расстояния к нулю, и нахождение спектра наблюдаемой, соответствующей дополнительному интегралу классической системы.
\end{enumerate}

{\defpositions}
\begin{enumerate}[beginpenalty=10000] % https://tex.stackexchange.com/a/476052/104425
  \item Доказана интегрируемость классического бильярда на софокусных столах, разделенных софокусными квадриками на области, заполненные изотропными средами с различными показателями `оптической плотности' при условии, что на границе раздела сред выполняется косинусный закон преломления.
 \item Исследованы и описаны слоения изоэнергетического многообразия на поверхности постоянного уровня дополнительного интеграла для двух `оптических систем' с косинусным законом преломления, т.е. для софокусных столов, разделенных софокусными квадриками на области, заполненные изотропными средами с различными показателями `оптической плотности'  при условии, что на границе раздела сред выполняется косинусный закон преломления.
   \item Исследовано асимптотическое поведение уровней энергии свободной квантовой частицы в двух областях, ограниченных дугами гипербол и внешнего эллипса, которые при стремлении фокального расстояния к нулю принимают форму кругового сектора при стремлении фокального расстояния к нулю, и вычислен спектр наблюдаемой, соответствующей дополнительному интегралу классической системы.

\end{enumerate}


{\novelty} 
Все положения диссертации, выносимые на защиту, являются оригинальными и получены автором самостоятельно  или при равноценном вкладе с соавторами. Кроме того, диссертация содержит следующие вспомогательные результаты, которые также являются новыми:
  \begin{itemize}[beginpenalty=10000] % https://tex.stackexchange.com/a/476052/104425
  \item впервые исследован косинусный закон преломления бильярдной траектории на софокусных столах
  \item для этого закона приведена методика построения бифуркационных диаграмм нового типа, одновременно учитывающих все возможные значения <<оптических>> параметров областей
  \item впервые получены особые поверхности, соответствующие одновременным бифуркациям разных типов в разных частях бильярдного стола
  \end{itemize}

%Для уровней энергии получены явные асимптотические выражения, в частности, формулы ($\ref{eq:funcRing}$) и ($\ref{eq:valRing}$) для $p$-листного накрытия эллиптического кольца, а также формулы ($\ref{eq:fun}$) и ($\ref{eq:val}$) для области $A_{\fixme{\varepsilon}}$ и формулы  ($\ref{eq:funB}$) и ($\ref{eq:valB}$) для области  $B_{\fixme{\varepsilon}}$. Для особых случаев для областей $A_{\fixme{\varepsilon}}$ и $B_{\fixme{\varepsilon}}$ справедливы формулы (\ref{eq:valS1}) и (\ref{eq:valS2}). 
%Приведенные асимптотики для собственных значений в зависимости от расстояния между фокусами справедливы с точностью до второго порядка включительно и подтверждаются численными экспериментами.

%Интегрируемость классических бильярдов в областях $A_{\fixme{\varepsilon}}$ и $B_{\fixme{\varepsilon}}$ следует из существования сохраняющейся величины в дополнение к полной энергии.
%В приложении~\ref{app:A} приведен квантовый аналог этой величины, сохраняющейся для квантовых бильярдов в этих же областях.
%
%\begin{enumerate}[beginpenalty=10000] % https://tex.stackexchange.com/a/476052/104425
%  \item Впервые \ldots
%  \item Впервые \ldots
%  \item Было выполнено оригинальное исследование \ldots
%\end{enumerate}
%
{\methods} В исследовании бильярда с косинусным законом преломления на софокусных квадриках применяются методы теории топологической классификации интегрируемых гамильтоновых систем с одной и двумя степенями свободы, построенной А.Т. Фоменко, Х. Цишангом, А.В. Болсиновым и другими. В исследовании асимптотического поведения уровней энергии квантового бильярда используются элементы теории Штурма и теории специальных функций, методы теории краевых задач. 

{\influence} 
Диссертация имеет теоретический характер.
Полученные результаты могут быть использованы при исследовании собственных функций и собственных значений оператор Лапласа в областях, ограниченных софокусными квадриками.
Разработанные методы позволяют получить асимптотику собственных значений в том числе для областей с углами на границе.

Результаты по теории бильярда с косинусным законом преломления могут быть использованы для расширения и исследования класса интегрируемых бильярдных систем. Ожидается нахождение новых интересных примеров интегрируемых систем с весьма нетривиальной топологией слоения Лиувилля.

%Ценность исследования бильярда с косинусным законом преломления заключается в расширении класса бильярдных задач, что позволяет строить новые интересные примеры интегрируемых систем, топология слоений Лиувилля для которых весьма нетривиальна. 
%Кроме того, получен динамический бильярд с первым интегралом, множество значений которого не является интервалом.

\bigskip
{\probation}
Основные результаты диссертации обоснованы в виде строгих математических доказательств и прошли апробацию на следующих научных конференциях и семинарах:

\begin{enumerate}%[beginpenalty=10000]
\item Студенческая школа-конференция <<Математическая весна --  2023>>, Нижний Новгород, Россия, 27-30 марта 2023;

\item Ломоносовские чтения 2023, Россия, 4-14 апреля 2023;

\item XXX Международная научная конференция студентов, аспирантов и молодых учёных <<Ломоносов-2023>>,  Москва, Россия, 10-21 апреля 2023;

\item Воронежская зимняя математическая школа С.Г. Крейна, Воронеж, Россия, 26-30 января 2024;

\item Студенческая школа-конференция <<Математическая весна -- 2024>>, Нижний Новгород, Россия, 25-28 марта 2024;

\item XXXI Международная конференция студентов, аспирантов и молодых ученых <<Ломоносов--2024>>, Москва, Россия, 12-26 апреля 2024;

%\item Современные геометрические и топологические методы, Сириус, Россия, 14-19 мая 2024;	%там не по диссеру был доклад
\item Семинар “Дифференциальная геометрия и приложения” под руководством акад. А.Т. Фоменко на механико-математическом факультете МГУ имени М.В. Ломоносова, 25 ноября 2024;


\item XXXII Международная научная конференция студентов, аспирантов и молодых ученых <<Ломоносов--2025>>, Москва, Россия, 11-25 апреля 2025.

\item V Конференция математических центров России, Красноярск, 11-16 августа 2025


\end{enumerate}
%{\contribution} Автор принимал активное участие \ldots


\ifnumequal{\value{bibliosel}}{0}
{%%% Встроенная реализация с загрузкой файла через движок bibtex8. (При желании, внутри можно использовать обычные ссылки, наподобие `\cite{vakbib1,vakbib2}`).
    {\publications} Основные результаты по теме диссертации изложены
    в четырех работах \nocite{nikulin2023spektr611484954, nikulin2024asymptotic617844539, vestnikLatest, sbornikLatest}
    X из которых изданы в журналах, рекомендованных ВАК,
    X "--- в тезисах докладов.
}%
{%%% Реализация пакетом biblatex через движок biber
    \begin{refsection}[bl-author, bl-registered]
        % Это refsection=1.
        % Процитированные здесь работы:
        %  * подсчитываются, для автоматического составления фразы "Основные результаты ..."
        %  * попадают в авторскую библиографию, при usefootcite==0 и стиле `\insertbiblioauthor` или `\insertbiblioauthorgrouped`
        %  * нумеруются там в зависимости от порядка команд `\printbibliography` в этом разделе.
        %  * при использовании `\insertbiblioauthorgrouped`, порядок команд `\printbibliography` в нём должен быть тем же (см. biblio/biblatex.tex)
        %
        % Невидимый библиографический список для подсчёта количества публикаций:
        \phantom{\printbibliography[heading=nobibheading, section=1, env=countauthorvak,          keyword=biblioauthorvak]%
        \printbibliography[heading=nobibheading, section=1, env=countauthorwos,          keyword=biblioauthorwos]%
        \printbibliography[heading=nobibheading, section=1, env=countauthorscopus,       keyword=biblioauthorscopus]%
        \printbibliography[heading=nobibheading, section=1, env=countauthorconf,         keyword=biblioauthorconf]%
        \printbibliography[heading=nobibheading, section=1, env=countauthorother,        keyword=biblioauthorother]%
        \printbibliography[heading=nobibheading, section=1, env=countregistered,         keyword=biblioregistered]%
        \printbibliography[heading=nobibheading, section=1, env=countauthorpatent,       keyword=biblioauthorpatent]%
        \printbibliography[heading=nobibheading, section=1, env=countauthorprogram,      keyword=biblioauthorprogram]%
        \printbibliography[heading=nobibheading, section=1, env=countauthor,             keyword=biblioauthor]%
        \printbibliography[heading=nobibheading, section=1, env=countauthorvakscopuswos, filter=vakscopuswos]%
        \printbibliography[heading=nobibheading, section=1, env=countauthorscopuswos,    filter=scopuswos]}%
        %
        \nocite{*}%
        %
        {\publications} Основные результаты по теме диссертации изложены в~\arabic{citeauthor}~печатных работах, 
        из которых \arabic{citeauthor} опубликованы в журналах, удовлетворяющих положению о присуждении учёных степеней в диссертационном совете МГУ. Список работ приведен в конце диссертации.
        
        %
%        \ifnum \value{citeauthorscopuswos}>0%
%            , \arabic{citeauthorscopuswos} "--- в~периодических научных журналах, индексируемых Web of~Science и Scopus%
%        \fi%
%        \ifnum \value{citeauthorconf}>0%
%            , \arabic{citeauthorconf} "--- в~тезисах докладов.
%        \else%
%            .
%        \fi%
%        \ifnum \value{citeregistered}=1%
%            \ifnum \value{citeauthorpatent}=1%
%                Зарегистрирован \arabic{citeauthorpatent} патент.
%            \fi%
%            \ifnum \value{citeauthorprogram}=1%
%                Зарегистрирована \arabic{citeauthorprogram} программа для ЭВМ.
%            \fi%
%        \fi%
%        \ifnum \value{citeregistered}>1%
%            Зарегистрированы\ %
%            \ifnum \value{citeauthorpatent}>0%
%            \formbytotal{citeauthorpatent}{патент}{}{а}{}%
%            \ifnum \value{citeauthorprogram}=0 . \else \ и~\fi%
%            \fi%
%            \ifnum \value{citeauthorprogram}>0%
%            \formbytotal{citeauthorprogram}{программ}{а}{ы}{} для ЭВМ.
%            \fi%
%        \fi%
        % К публикациям, в которых излагаются основные научные результаты диссертации на соискание учёной
        % степени, в рецензируемых изданиях приравниваются патенты на изобретения, патенты (свидетельства) на
        % полезную модель, патенты на промышленный образец, патенты на селекционные достижения, свидетельства
        % на программу для электронных вычислительных машин, базу данных, топологию интегральных микросхем,
        % зарегистрированные в установленном порядке.(в ред. Постановления Правительства РФ от 21.04.2016 N 335)
    \end{refsection}%
    \begin{refsection}[bl-author, bl-registered]
        % Это refsection=2.
        % Процитированные здесь работы:
        %  * попадают в авторскую библиографию, при usefootcite==0 и стиле `\insertbiblioauthorimportant`.
        %  * ни на что не влияют в противном случае
        \nocite{vakbib2}%vak
        \nocite{patbib1}%patent
        \nocite{progbib1}%program
        \nocite{bib1}%other
        \nocite{confbib1}%conf
        \nocite{nikulin2023spektr611484954}
	\nocite{nikulin2024asymptotic617844539}
	\nocite{vestnikLatest}
	\nocite{sbornikLatest}

    \end{refsection}%
        %
        % Всё, что вне этих двух refsection, это refsection=0,
        %  * для диссертации - это нормальные ссылки, попадающие в обычную библиографию
        %  * для автореферата:
        %     * при usefootcite==0, ссылка корректно сработает только для источника из `external.bib`. Для своих работ --- напечатает "[0]" (и даже Warning не вылезет).
        %     * при usefootcite==1, ссылка сработает нормально. В авторской библиографии будут только процитированные в refsection=0 работы.
        \nocite{nikulin2023spektr611484954}
	\nocite{nikulin2024asymptotic617844539}
	\nocite{vestnikLatest}
	\nocite{sbornikLatest}
}


\ifsynopsis
{\volumeAndStructure} Диссертация состоит из~введения,
%\fixme{XX} глав, заключения и~приложения. Полный объем диссертации
%\fixme{ХХХ}~страниц с~\fixme{ХХ}~рисунками и~\fixme{5}~таблицами. Список
%литературы содержит \fixme{ХХX}~наименование.
5 глав и заключения. Полный объем диссертации
132~страницы с~109~рисунков и~2~таблицами. Список
литературы содержит 34~наименования.
\else
\begin{refsection}[bl-author, bl-registered]
	% Это refsection=3.
	% Для подсчёта позиций списка литературы при группировке работ автора
	%
	% Невидимый библиографический список для подсчёта количества публикаций:
	\printbibliography[heading=nobibheading, section=0, env=counter, keyword=bibliofull]%
	%
	\nocite{*}%
	%% authorother
	%		\nocite{bib1}%
	%		\nocite{bib2}%
	%
	{\volumeAndStructure} Диссертация состоит из~введения,
	\formbytotal{totalchapter}{глав}{ы}{}{} и заключения.
	Полный объём диссертации составляет
	\formbytotal{TotPages}{страниц}{у}{ы}{}, включая
	\formbytotal{totalcount@figure}{рисун}{ок}{ка}{ков} и
	\formbytotal{totalcount@table}{таблиц}{у}{ы}{}.
	Список литературы содержит
	\formbytotal{citenum}{наименован}{ие}{ия}{ий}.
\end{refsection}%
\fi
