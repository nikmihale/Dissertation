%%% Основные сведения %%%
\newcommand{\thesisAuthorLastName}{Никулин}
\newcommand{\thesisAuthorOtherNames}{Михаил Александрович}
\newcommand{\thesisAuthorInitials}{М.\,А.}
\newcommand{\thesisAuthor}             % Диссертация, ФИО автора
{%
    \texorpdfstring{% \texorpdfstring takes two arguments and uses the first for (La)TeX and the second for pdf
        \thesisAuthorLastName~\thesisAuthorOtherNames% так будет отображаться на титульном листе или в тексте, где будет использоваться переменная
    }{%
        \thesisAuthorLastName, \thesisAuthorOtherNames% эта запись для свойств pdf-файла. В таком виде, если pdf будет обработан программами для сбора библиографических сведений, будет правильно представлена фамилия.
    }
}
\newcommand{\thesisAuthorShort}        % Диссертация, ФИО автора инициалами
{\thesisAuthorInitials~\thesisAuthorLastName}
\newcommand{\thesisUdk}                % Диссертация, УДК
{\fixme{xxx.xxx}}
\newcommand{\thesisTitle}              % Диссертация, название
{НЕКОТОРЫЕ СВОЙСТВА КВАНТОВЫХ БИЛЛИАРДОВ НА СОФОКУСНЫХ СТОЛАХ}
\newcommand{\thesisSpecialtyNumber}    % Диссертация, специальность, номер
{1.1.3}
\newcommand{\thesisSpecialtyTitle}     % Диссертация, специальность, название (название взято с сайта ВАК для примера)
{геометрия и топология}
%% \newcommand{\thesisSpecialtyTwoNumber} % Диссертация, вторая специальность, номер
%% {\fixme{XX.XX.XX}}
%% \newcommand{\thesisSpecialtyTwoTitle}  % Диссертация, вторая специальность, название
%% {\fixme{Теория и~методика физического воспитания, спортивной тренировки,
%% оздоровительной и~адаптивной физической культуры}}
\newcommand{\thesisDegree}             % Диссертация, ученая степень
{кандидата физико-математических наук}
\newcommand{\thesisDegreeShort}        % Диссертация, ученая степень, краткая запись
{канд. физ.-мат. наук}
\newcommand{\thesisCity}               % Диссертация, город написания диссертации
{Москва}
\newcommand{\thesisYear}               % Диссертация, год написания диссертации
{\the\year}
\newcommand{\thesisOrganization}       % Диссертация, организация
{
МОСКОВСКИЙ ГОСУДАРСТВЕННЫЙ УНИВЕРСИТЕТ  \\
имени М.В.ЛОМОНОСОВА \\
МЕХАНИКО-МАТЕМАТИЧЕСКИЙ ФАКУЛЬТЕТ}
% или так, но я так не хочу
%ФЕДЕРАЛЬНОЕ ГОСУДАРСТВЕННОЕ БЮДЖЕТНОЕ ОБРАЗОВАТЕЛЬНОЕ 
%УЧРЕЖДЕНИЕ ВЫСШЕГО ОБРАЗОВАНИЯ 
%«МОСКОВСКИЙ ГОСУДАРСТВЕННЫЙ УНИВЕРСИТЕТ  \\
%имени М.В.ЛОМОНОСОВА»	}
\newcommand{\thesisOrganizationShort}  % Диссертация, краткое название организации для доклада
{МГУ им. М.В.Ломоносова}

\newcommand{\thesisInOrganization}     % Диссертация, организация в предложном падеже: Работа выполнена в ...
{кафедре дифференциальной геометрии и приложений механико-математического факультета ФГБОУ ВО <<Московский государственный университет имени М.В. Ломоносова>>}

%% \newcommand{\supervisorDead}{}           % Рисовать рамку вокруг фамилии
\newcommand{\supervisorFio}              % Научный руководитель, ФИО
{Попеленский Фёдор Юрьевич}
\newcommand{\supervisorRegalia}          % Научный руководитель, регалии
{доцент, кандидат физико-математических наук}
\newcommand{\supervisorFioShort}         % Научный руководитель, ФИО
{Ф.\,Ю.~Попеленский}
\newcommand{\supervisorRegaliaShort}     % Научный руководитель, регалии
{доц.,к.ф.-м.н.}

%% \newcommand{\supervisorTwoDead}{}        % Рисовать рамку вокруг фамилии
%% \newcommand{\supervisorTwoFio}           % Второй научный руководитель, ФИО
%% {\fixme{Фамилия Имя Отчество}}
%% \newcommand{\supervisorTwoRegalia}       % Второй научный руководитель, регалии
%% {\fixme{уч. степень, уч. звание}}
%% \newcommand{\supervisorTwoFioShort}      % Второй научный руководитель, ФИО
%% {\fixme{И.\,О.~Фамилия}}
%% \newcommand{\supervisorTwoRegaliaShort}  % Второй научный руководитель, регалии
%% {\fixme{уч.~ст.,~уч.~зв.}}

\newcommand{\opponentOneFio}           % Оппонент 1, ФИО
{Тюрин Николай Андреевич}
\newcommand{\opponentOneRegalia}       % Оппонент 1, регалии
{доктор физико-математических наук, профессор РАН}
\newcommand{\opponentOneJobPlace}      % Оппонент 1, место работы
{Объединенный институт ядерных исследований, Лаборатория Теоретической Физики}
\newcommand{\opponentOneJobPost}       % Оппонент 1, должность
{начальник сектора}

\newcommand{\opponentTwoFio}           % Оппонент 2, ФИО
{Соколов Сергей Викторович}
\newcommand{\opponentTwoRegalia}       % Оппонент 2, регалии
{доктор физико-математических наук}
\newcommand{\opponentTwoJobPlace}      % Оппонент 2, место работы
{ФГАОУ ВО <<Московский физико-технический институт (национальный исследовательский университет)>>}
\newcommand{\opponentTwoJobPost}       % Оппонент 2, должность
{заведующий кафедрой теоретической механики}

 \newcommand{\opponentThreeFio}         % Оппонент 3, ФИО
 {Цветкова Анна Валерьевна}
 \newcommand{\opponentThreeRegalia}     % Оппонент 3, регалии
 {кандидат физико-математических наук}
 \newcommand{\opponentThreeJobPlace}    % Оппонент 3, место работы
 {Институт проблем механики им. А.Ю. Ишлинского РАН, Лаборатория механики природных катастроф}
 \newcommand{\opponentThreeJobPost}     % Оппонент 3, должность
 {научный сотрудник}

%\newcommand{\leadingOrganizationTitle} % Ведущая организация, дополнительные строки. Удалить, чтобы не отображать в автореферате
%{\fixme{Федеральное государственное бюджетное образовательное учреждение высшего
%профессионального образования с~длинным длинным длинным длинным названием}}

\newcommand{\defenseDate}              % Защита, дата
{\fixme{DD mmmmmmmm YYYY~года~в~XX часов YY минут}}
\newcommand{\defenseCouncilNumber}     % Защита, номер диссертационного совета
{\fixme{Д\,123.456.78}}
\newcommand{\defenseCouncilTitle}      % Защита, учреждение диссертационного совета
{ФГБОУ ВО <<Московский государственный университет имени М.В.Ломоносова>>}
\newcommand{\defenseCouncilAddress}    % Защита, адрес учреждение диссертационного совета
{Российская Федерация,
119234, Москва, ГСП-1, Ленинские горы, д. 1, МГУ имени М. В. Ломоносова, механико-математический факультет, аудитория 14-08}
\newcommand{\defenseCouncilPhone}      % Телефон для справок
{\fixme{+7~(0000)~00-00-00}}

\newcommand{\defenseSecretaryFio}      % Секретарь диссертационного совета, ФИО
{\fixme{Фамилия Имя Отчество}}
\newcommand{\defenseSecretaryRegalia}  % Секретарь диссертационного совета, регалии
{\fixme{д-р~физ.-мат. наук}}            % Для сокращений есть ГОСТы, например: ГОСТ Р 7.0.12-2011 + http://base.garant.ru/179724/#block_30000

\newcommand{\synopsisLibrary}          % Автореферат, название библиотеки
{Фундаментальной библиотеке ФГБОУ ВО МГУ имени М.В.Ломоносова по адресу: Москва, Ломоносовский проспект, д. 27}
\newcommand{\synopsisDate}             % Автореферат, дата рассылки
{\fixme{DD mmmmmmmm} \the\year~года}

% To avoid conflict with beamer class use \providecommand
\providecommand{\keywords}%            % Ключевые слова для метаданных PDF диссертации и автореферата
{}
